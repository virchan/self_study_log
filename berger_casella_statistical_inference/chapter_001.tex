\documentclass[12pt,letterpaper,reqno]{amsart}


\usepackage{lipsum} %separate the paragraphs of the dummy text into TEX-paragraphs.

\usepackage{tikz} %commutative diagrams

\usepackage{setspace} %Provides support for setting the spacing between lines in a document.

\usepackage{amsmath,amssymb} %provides miscellaneous enhancements for improving the information structure and printed output of documents containing mathematical formulas. e.g., \DeclareMathOperator, \sin, \lim

\usepackage{enumitem} %additional options for \enumerate

\usepackage{subfig} %provides support for the manipulation and reference of small or ‘sub’ figures and tables within a single figure or table environment.

\usepackage{framed} %box, shaded, put a left line around a region.

\usepackage{etoolbox} %provides LATEX frontends to some of the new primitives provided by e-TEX as well as some generic tools which are not strictly related to e-TEX but match the profile of this package.

\usepackage{bm} %Access bold symbols in maths mode

\usepackage{mdframed} %Framed environments that can split at page boundaries

\usepackage{mathrsfs} %Support for using RSFS fonts in maths

\usepackage{centernot} %Centred \not command

\usepackage{thmtools} %Extensions to theorem environments

\usepackage{thm-restate} %Restate Theorem

\usepackage{fancyvrb} %Sophisticated verbatim text

\usepackage{dsfont} %?

\usepackage{bbm} %Blackboard variants of Computer Modern fonts.

\usepackage{breqn} %Automatic line breaking of displayed equations

\usepackage{adjustbox} %The package provides several macros to adjust boxed content.

\usepackage{tabularx} %The package defines an environment tabularx, an extension of tabular which has an additional column designator, X, which creates a paragraph-like column whose width automatically expands so that the declared width of the environment is filled. (Two X columns together share out the available space between them, and so on.)

\usepackage{calligra} %Calligraphic font

\usepackage[
    backend=bibtex, 
    natbib=true, 
    bibstyle=verbose, citestyle=verbose,    % bibstyle extensively modifed below
    doi=true, url=true,                     % excluded from citations below
    citecounter=true, citetracker=true,
    block=space, 
    backref=false, backrefstyle=two,
    abbreviate=false,
    isbn=true, maxbibnames=4, maxcitenames=4,
    style=alphabetic
]{biblatex} %Adding Reference to your paper.

\usepackage[margin=1in]{geometry} %edit margin of your document

\usepackage[makeindex]{imakeidx} %making index

\usepackage{empheq} %The package provides a visual markup extension to amsmath.

\usepackage{longtable} %allow longtable

\usepackage[mathscr]{eucal}%change mathscr

\usepackage{xpatch} %fix skipbelow in mdframe

\usepackage{standalone}  %need this to include the tikz picture created in different files.

\usepackage{wasysym}

\usepackage{bashful}

\usepackage{hyperref,cleveref} %Extensive support for hypertext in LATEX (ALWAYS LOAD LAST!) %this loads the packages


\makeindex

\author{Virgil Chan}
\title{Casella-Berger \\ Statistical Inference Solution: \\ Chapter 1}
\date{July 27, 2022}


\usetikzlibrary{patterns,positioning,arrows,chains,matrix,positioning,scopes} %options for tikzpicture

\addbibresource{bibliography.bib} %put your reference here


\makeatletter
\patchcmd{\@maketitle}
  {\ifx\@empty\@dedicatory}
  {\ifx\@empty\@date \else {\vskip3ex \centering\footnotesize\@date\par\vskip1ex}\fi
   \ifx\@empty\@dedicatory}
  {}{}
\patchcmd{\@adminfootnotes}
  {\ifx\@empty\@date\else \@footnotetext{\@setdate}\fi}
  {}{}{}
\makeatother

\makeatletter
\xpatchcmd{\endmdframed}
  {\aftergroup\endmdf@trivlist\color@endgroup}
  {\endmdf@trivlist\color@endgroup\@doendpe}
  {}{}
\makeatother
%\topsep
\newtheoremstyle{break}
  {\topsep}% Space above
  {\topsep}% Space below
  {\it}% Body font
  {}% Indent amount
  {\bfseries}% Theorem head font
  {}% Punctuation after theorem head
  {\topsep}% Space after theorem head, ' ', or \newline
  {\thmname{#1}\thmnumber{ #2} \thmnote{(#3)}}% Theorem head spec (can be left empty, meaning `normal')

\mdfdefinestyle{box}{
     linecolor=white,
    skipabove=\topsep,
    skipbelow=\topsep,
    innerbottommargin=\topsep,
}  %

\theoremstyle{break}
\newmdtheoremenv[style=box]{theorem}{Theorem}[section]
\newmdtheoremenv[style=box]{lemma}[theorem]{Lemma}
\newmdtheoremenv[style=box]{proposition}[theorem]{Proposition}
\newmdtheoremenv[style=box]{corollary}[theorem]{Corollary}
\newmdtheoremenv[style=box]{definition}[theorem]{Definition}
\newmdtheoremenv[style=box]{addendum}[theorem]{Addendum}
\newmdtheoremenv[style=box]{conjecture}[theorem]{Conjecture}
\newmdtheoremenv[style=box]{question}[theorem]{Question}
\newmdtheoremenv[style=box]{condit}[theorem]{Condition}

\makeatletter
\def\th@plain{%
  \thm@notefont{}% same as heading font
  \normalfont % body font
}
\def\th@definition{%
  \thm@notefont{}% same as heading font
  \normalfont % body font
}
\makeatother

\newtheoremstyle{exampstyle}
{\topsep} % Space above
{\topsep} % Space below
{} % Body font
{} % Indent amount
{\bfseries} % Theorem head font
{} % Punctuation after theorem head
{\topsep} % Space after theorem head
{\thmname{#1}\thmnumber{ #2} \thmnote{(#3)}} % Theorem head spec (can be left empty, meaning `normal')

\theoremstyle{exampstyle}
\newtheorem{example}[theorem]{Example}
\newtheorem{remark}[theorem]{Remark}


\pdfpagewidth 8.5in
\pdfpageheight 11in

    
%\BEGIN{COMMAND}

%SHORT HAND NOTATION-------------------------------
\newcommand{\D}[1]{\mathbb{#1}} %short hand mathbb command

\newcommand{\Dm}[1]{\mathbbm{#1}} %mathbb for numbers

\newcommand{\mycal}[1]{\mathcal{{#1}}} %short hand mathcal

\newcommand{\scr}[1]{\mathscr{{#1}}} %short hand mathscr

\newcommand{\ve}{\varepsilon} %short hand epsilon

\newcommand\numberthis{\addtocounter{equation}{1}\tag{\theequation}}%label equation in \align*

\newcommand{\dund}[1]{\underline{\underline{{#1}}}}%Double underline

\newcommand{\Frechet}{Fr\'{e}chet}%Frechet space

\newcommand{\ol}[1]{\overline{{#1}}}%overline

\newcommand{\ul}[1]{\underline{{#1}}}%underline

\newcommand{\dul}[1]{\ul{\ul{{#1}}}}%double underline

\newcommand{\Cech}{\v{C}ech} %Cech

\newcommand{\Nother}{N\"{o}ther} %Nother

\newcommand{\Notherian}{N\"{o}ther} %Notherian

\newcommand{\wt}[1]{\widetilde{{#1}}} %short hand for \widetilde

\newcommand{\fun}[2]{{#1}\! \left( {#2} \right)}%short hand for function

\newcommand{\on}[1]{\operatorname{{#1}}} %short hand for \operatorname

\newcommand{\Kp}[1]{K^+_{{#1}}} %short hand for plus-construction

\newcommand{\Knp}[1]{K_{{#1}}} %short hand for before plus-construction

\newcommand{\btimes}{\boxtimes} %short hand for box tensor

\newcommand{\bplus}{\boxplus} %short hand for box sum

\newcommand{\myline}{\par\noindent\rule{\textwidth}{0.4pt}} %horizontal line

\newcommand{\wh}[1]{\widehat{{#1}}} %short hand for \widetilde

\newcommand{\myref}[2]{\hyperref[{#2}]{{#1}~\ref{#2}}} %short hand for \hyperref

\newcommand{\myeqref}[2]{\hyperref[{#2}]{{#1}~\eqref{#2}}} %short hand for \hyperref for equations

%SET THEORY----------------------------------------
\newcommand{\SET}[2]{\left\{ {#1} \ \middle| \ {#2} \right\}} %short hand notation for set

\newcommand{\SETT}[1]{\left\{ {#1} \right\}} %short hand notation for set without constraint

\newcommand{\floor}[1]{\left \lfloor{#1}\right \rfloor } %floor function

\newcommand{\Rng}[1]{\operatorname{Rng}\left(#1\right)} %range of a function

\newcommand{\id}[1]{\operatorname{id}_{{#1}}} %identity map

\newcommand\restr[2]{{% we make the whole thing an ordinary symbol
  \left.\kern-\nulldelimiterspace % automatically resize the bar with \right
  #1 % the function
  \vphantom{\big|} % pretend it's a little taller at normal size
  \right|_{#2} % this is the delimiter
  }} %restriction of a function

\newcommand{\Abs}[1]{\left\lvert{#1}\right\rvert} %absolute value

\newcommand{\nonnegint}{\D{N} \cup \left\{ 0 \right\}} %Non-negative integers

\newcommand{\Image}[1]{\operatorname{im}\left( {#1} \right)} %Image of a map

\newcommand{\Proj}[1]{\operatorname{proj}_{{#1}}}%projection map

\newcommand{\ev}[1]{\operatorname{ev}_{{#1}}}%Evaluation map

\makeatletter
\newbox\xrat@below
\newbox\xrat@above
\newcommand{\xrightarrowtail}[2][]{%
  \setbox\xrat@below=\hbox{\ensuremath{\scriptstyle #1}}%
  \setbox\xrat@above=\hbox{\ensuremath{\scriptstyle #2}}%
  \pgfmathsetlengthmacro{\xrat@len}{max(\wd\xrat@below,\wd\xrat@above)+.6em}%
  \mathrel{\tikz [>->,baseline=-.75ex]
                 \draw (0,0) -- node[below=-2pt] {\box\xrat@below}
                                node[above=-2pt] {\box\xrat@above}
                       (\xrat@len,0) ;}}
\makeatother %xrightarrowtail

\makeatletter
\newbox\xrat@below
\newbox\xrat@above
\newcommand{\xtwoheadrightarrow}[2][]{%
  \setbox\xrat@below=\hbox{\ensuremath{\scriptstyle #1}}%
  \setbox\xrat@above=\hbox{\ensuremath{\scriptstyle #2}}%
  \pgfmathsetlengthmacro{\xrat@len}{max(\wd\xrat@below,\wd\xrat@above)+.6em}%
  \mathrel{\tikz [->>,baseline=-.75ex]
                 \draw (0,0) -- node[below=-2pt] {\box\xrat@below}
                                node[above=-2pt] {\box\xrat@above}
                       (\xrat@len,0) ;}}
\makeatother %xtwoheadrightarrow

\newcommand{\dom}[1]{\operatorname{dom} \left( {#1} \right)} %Domain of a function

\newcommand{\codom}[1]{\operatorname{codom} \left( {#1} \right)} %Codomain of a function

\newcommand{\sseq}{\subseteq} %short hand for \subseteq
  
%ALGEBRA-------------------------------------------
\newcommand{\Aut}[2]{\operatorname{Aut}_{{#2}}\left({#1} \right)} %automorphism group

\newcommand{\Hom}[3]{\operatorname{Hom}_{{#3}}\left({#1},{#2}\right)} %Homomorphism group

\newcommand{\Char}[1]{\operatorname{char}\left( {#1} \right)} %characteristic of a ring/field

\newcommand{\Rad}[1]{\operatorname{Rad}\left({#1}\right)} %radical

\newcommand{\Nil}[1]{\operatorname{Nil}\left({#1}\right)} %nilradical

\newcommand{\Tors}[1]{\operatorname{Tors}\left({#1} \right)} %Torsion 

\newcommand{\Gal}[2]{\operatorname{Gal}\left({#1}/{#2}\right)} %Galois group

\newcommand{\pid}[1]{\left\langle {#1} \right\rangle} %Principal Ideal

\newcommand{\polynomial}[5]{\sum_{{#1}={#2}}^{{#3}} {#4}_{{#1}} {{#5}}^{{#1}}} %Polynomial

\newcommand{\mydeg}[1]{\deg \left( {#1} \right)} %Degree

\newcommand{\GL}[1]{GL \left({#1}\right)} %General linear group

\newcommand{\abelian}[1]{{#1}^{ab}} %Abelianisation

\newcommand{\Orb}[2]{\operatorname{Orb}_{{#1}}\left( {#2} \right)} %orbit of group action

\newcommand{\Stab}[2]{\operatorname{Stab}_{{#1}}\left( {#2} \right)} %stabiliser of group action

\newcommand{\GROUP}[2]{\left\langle {#1} \ \middle| \ {#2} \right\rangle} %group presentation

\newcommand{\ideal}{\trianglelefteq}%normal subgroup symbol

\newcommand{\properideal}{\triangleleft}%proper normal subgroup symbol

\newcommand{\groupaction}[3]{{#1} \rotatebox[origin=c]{180}{$\circlearrowright$}_{{#3}} {#2}}%group action

\newcommand{\SL}[1]{SL \left({#1}\right)} %special linear group

\newcommand{\iso}[1]{\operatorname{iso}\left( {#1} \right)}%Group of isomorphisms

\newcommand{\Gr}[1]{\operatorname{Gr}\left( {#1} \right)}%associated graded group

\newcommand{\bigboxplus}{
  \mathop{
    \vphantom{\bigoplus} 
    \mathchoice
      {\vcenter{\hbox{\resizebox{\widthof{$\displaystyle\bigoplus$}}{!}{$\boxplus$}}}}
      {\vcenter{\hbox{\resizebox{\widthof{$\bigoplus$}}{!}{$\boxplus$}}}}
      {\vcenter{\hbox{\resizebox{\widthof{$\scriptstyle\oplus$}}{!}{$\boxplus$}}}}
      {\vcenter{\hbox{\resizebox{\widthof{$\scriptscriptstyle\oplus$}}{!}{$\boxplus$}}}}
  }\displaylimits 
} %internal direct sum

\newcommand{\bideg}[1]{\text{bi-deg}\left( {#1} \right)}%bi-degree

\newcommand{\Stgroup}[1]{\operatorname{St}\left({#1} \right)} %Steinberg group

\newcommand{\groupp}[2]{\left\langle {#1} \ \middle| \ {#2} \right\rangle} %group presentation

%LINEAR ALGEBRA------------------------------------
\newcommand{\rank}[1]{\operatorname{{rank}}\left({#1}\right)} %rank

\newcommand{\diag}[1]{\operatorname{diag}\left({#1} \right)} %diagonal matrix

\newcommand{\Ker}[1]{\operatorname{ker}\left({#1} \right)} %Kernel of a map

\newcommand{\Tr}[1]{\operatorname{trace}\left({#1}\right)} %trace

\newcommand{\mat}[4]{\left[ \begin{array}{cc}
{#1} & {#2}  \\
{#3} & {#4}
\end{array} \right]} %2x2 matrix

\newcommand{\Range}[1]{\operatorname{Rng}\left( {#1} \right)} %Range of a map

\newcommand{\Span}[1]{\operatorname{span}\left( {#1} \right)} %span of a set

\newcommand{\Null}[1]{\operatorname{null}\left({#1}\right)} %nullity

\newcommand{\innerprod}[3]{\left\langle {#1}, {#2} \right\rangle_{{#3}}} %Inner product

\newcommand{\colvec}[2]{\left[ \begin{array}{c}
{#1} \\
\vdots \\
{#2}
\end{array} \right]}%column vector

\newcommand{\Dim}[2]{\dim_{{#2}} \left( {#1} \right)}%dimension

\newcommand{\Det}[1]{\det \left( {#1} \right)}%determinant

\newcommand{\coDim}[2]{\operatorname{codim}_{{#2}} \left( {#1} \right)}%codimension

%COMMUTATIVE ALGEBRA AND ALGEBRAIC GEOMETRY--------
\newcommand{\td}[1]{\operatorname{tr.deg.}\left( {#1} \right)} %transcendence degree

\newcommand{\Spec}[1]{\operatorname{Spec}\left({#1}\right)} %spectrum of a ring

\newcommand{\height}{\operatorname{height}} %height

\newcommand{\mSpec}[1]{\operatorname{mSpec}\left( {#1}\right)} %maximal ideals of a ring

\newcommand{\length}[1]{\operatorname{length}\left({#1} \right)} %length

\newcommand{\Pic}[1]{\operatorname{Pic}\left({#1}\right)}  %Piccard group

\newcommand{\Idem}[1]{\operatorname{Idem}\left( {#1} \right)}%Idempotent

\newcommand{\sections}[1]{\Gamma\left({#1} \right)}%sections of sheaf/bundle

\newcommand{\kerpre}[1]{\operatorname{ker}_{\operatorname{presheaf}}\left( {#1} \right)}%presheaf kernel

\newcommand{\cokerpre}[1]{\operatorname{coker}_{\operatorname{presheaf}}\left( {#1} \right)}%presheaf cokernel

\newcommand{\RES}[2]{\operatorname{res}_{{#1},{#2}}}%restriction of a presheaf

%Complex Analysis----------------------------------

\newcommand{\expi}[2]{e^{\frac{{#1}}{{#2}}}} %complex exponential with imaginary fraction exponent

\newcommand{\Res}[2]{\operatorname{Res} \left[ {#1},{#2} \right]} %residue

\newcommand{\myRe}[1]{\operatorname{Re}\left({#1} \right)} %Real part

\newcommand{\myIm}[1]{\operatorname{Im}\left({#1} \right)} %Imaginary part

\newcommand{\winding}[2]{\fun{\operatorname{Ind}_{{#1}}}{{#2}}}%winding number of a curve

\newcommand{\polydisc}[1]{\D{D}^{{#1}}_{\operatorname{poly}}}%poly-disc

\newcommand{\distbound}[1]{\partial^{\operatorname{dist}}\polydisc{{#1}}}%distinguished boundary

\newcommand{\Blaschke}[1]{\fun{\operatorname{Blaschke}}{{#1}}}%Blaschke product

%HOMOLOGICAL ALGEBRA AND CATEGORY THEORY-----------
\newcommand{\Mor}[2]{\operatorname{mor}_{{#2}}\left({#1}\right)} %Morphism class

\newcommand{\cat}[1]{\normalfont{\mathbf{#1}}} %notation for category

\newcommand{\Tor}[3]{\operatorname{Tor}_{{#3}}\left( {#1}, {#2} \right)} %Tor functor
\newcommand{\catset}{\operatorname{\scr{S}ets}} %category of sets

\newcommand{\Ring}{\operatorname{\scr{R}ings}} %category of rings

\newcommand{\Obj}[1]{\operatorname{obj}\left({#1} \right)} %Object class of category

\newcommand{\Ext}[2]{\operatorname{Ext}\left( {#1}, {#2} \right)} %Ext-functor

\newcommand{\coker}[1]{\operatorname{{coker}}\left( {#1} \right)} %cokernel

\makeatletter
\newcommand{\colim@}[2]{%
  \vtop{\m@th\ialign{##\cr
    \hfil$#1\operator@font colim$\hfil\cr
    \noalign{\nointerlineskip\kern1.5\ex@}#2\cr
    \noalign{\nointerlineskip\kern-\ex@}\cr}}%
}
\newcommand{\colim}[1]{%
  \mathop{\mathpalette\colim@{}}_{{#1}}
} %colimit

\renewcommand{\varprojlim}{%
  \mathop{\mathpalette\varlim@{\leftarrowfill@\scriptscriptstyle}}\nmlimits@
}
\renewcommand{\varinjlim}{%
  \mathop{\mathpalette\varlim@{\rightarrowfill@\scriptscriptstyle}}\nmlimits@
} %limit

\newcommand{\hocolim@}[2]{%
  \vtop{\m@th\ialign{##\cr
    \hfil$#1\operator@font hocolim$\hfil\cr
    \noalign{\nointerlineskip\kern1.5\ex@}#2\cr
    \noalign{\nointerlineskip\kern-\ex@}\cr}}%
}
\newcommand{\hocolim}[1]{%
  \mathop{\mathpalette\hocolim@{}}_{{#1}}
}%homotopy colimit

\newcommand{\holim@}[2]{%
  \vtop{\m@th\ialign{##\cr
    \hfil$#1\operator@font holim$\hfil\cr
    \noalign{\nointerlineskip\kern1.5\ex@}#2\cr
    \noalign{\nointerlineskip\kern-\ex@}\cr}}%
}
\newcommand{\holim}[1]{%
  \mathop{\mathpalette\holim@{}}_{{#1}}
}%homotopy limit

\newcommand{\tensor}[3]{{#1} \otimes_{{#3}} {#2}}%tensor product

\newcommand{\Eq}[1]{\operatorname{Eq} \left( {#1} \right)}%Equaliser

\newcommand{\coEq}[1]{\operatorname{coEq} \left( {#1} \right)}%Coequaliser

\newcommand{\catfgProj}[1]{\operatorname{\scr{P}roj}^{\operatorname{fg}}_{{#1}}} %category of finitely generated projective R-modules

\newcommand{\catmodule}[1]{\operatorname{\scr{M}odule}_{{#1}}} %category of R-modules

\newcommand{\catspace}{\operatorname{\scr{S}paces}} %category of spaces

\newcommand{\catspectra}{\operatorname{\scr{S}pectra}} %category of spectra

\newcommand{\catab}{\operatorname{\scr{A}belian}} %category of Abelian groups

\newcommand{\catringoid}{\operatorname{\scr{R}ingoids}} %category of ringoids

\newcommand{\catgroup}{\operatorname{\scr{G}roups}} %category of groups

\newcommand{\catfgfree}[1]{\operatorname{\scr{F}ree}^{\operatorname{fg}}_{{#1}}} %category of finitely generated free R-modules

\newcommand{\hormor}[1]{\operatorname{hor-mor}\left({#1}\right)} % horizontal morphism class

\newcommand{\vermor}[1]{\operatorname{ver-mor}\left({#1}\right)} % vertical morphism class

\newcommand{\bimor}[1]{\operatorname{bimor}\left({#1}\right)} % bi-morphism class

\newcommand{\catiso}[1]{\operatorname{iso}\left({#1}\right)}%category of isomorphisms

\newcommand{\SiS}[1]{\scr{S}_{{#1}}} %S-inverse-S-construction of category of isomorphisms

\newcommand{\CMA}[2]{\scr{C}_{{#2}}\left( {#1} \right)} %Pedersen-Weibel category

\newcommand{\catfinset}{\operatorname{\scr{F}in\scr{S}et}} %category of finite set

\newcommand{\catSMC}{\operatorname{\scr{S}ym\scr{M}on\scr{C}at}} %category of symmetric monoidal categories

\newcommand{\PiP}[1]{{\scr{P}_{{#1}}}} %S-inverse-S-construction of category of isomorphisms in the idempotent completion

\newcommand{\catoofree}[1]{\operatorname{\scr{F}ree}^{\mathbb{N}}_{{#1}}} %category of countably generated free R-modules

\newcommand{\catomegaspectra}{\operatorname{\Omega-\scr{S}pectra}} %category of omega-spectra

%TOPOLOGY------------------------------------------
\newcommand{\point}{\operatorname{point}} %point 

\newcommand{\Closure}[2]{\operatorname{Closure}_{{#1}}\left({#2} \right)} %Closure

\newcommand{\Int}[1]{\operatorname{Int}\left({#1} \right)} %Set of interior points

\newcommand{\Bd}[1]{\partial {#1}} %Boundary of a set

\newcommand{\sphere}[1]{\D{S}^{{#1}}} %sphere

\newcommand{\CP}[1]{\D{C}\D{P}^{{#1}}} %complex projective spaces

\newcommand{\RP}[1]{\D{R}\D{P}^{{#1}}} %real projective spaces

\newcommand{\sk}[2]{{#1}^{({#2})}} %n-skeleton of a CW complex

\newcommand{\simplex}[1]{\left[ {#1} \right]} %Simplex

\newcommand{\commutativesquare}[8]{\begin{tikzpicture}
  \node (A) {{#1}}; 
  \node (B) [right=of A] {{#3}}; 
  \node (C) [below=of A] {{#4}}; 
  \node (D) [right=of C, below=of B] {{#6}};
  \draw[->] (A)-- node[above] {\tiny {#2}} (B); 
  \draw[->] (A)-- node [left] {\tiny {#7}} (C); 
  \draw[->] (B)-- node [right] {\tiny {#8}} (D); 
  \draw[->] (C)-- node [below] {\tiny {#5}} (D); 
\end{tikzpicture}}%Commutative square

\newcommand{\gtori}[1]{\left( \D{T}^2 \right)^{\vee {#1}}}%wedge sum of g tori

\newcommand{\cupprod}[2]{{#1}\smile {#2}}%cup product

\newcommand{\capprod}[2]{{#1} \frown  {#2}}%cap product

\newcommand{\Map}[3]{\operatorname{Map}_{{#3}} \left( {#1}, {#2} \right)}%Mapping space

\newcommand{\Loop}[1]{\Omega {#1}}%loop space

\newcommand{\Suspen}[1]{\Sigma {#1}}%suspension over a space

\newcommand{\Face}[2]{d_{{#1}}^{{#2}}}%face map of simplicial space

\newcommand{\Degen}[2]{s_{{#1}}^{{#2}}}%degeneracy map of simplicial space

\newcommand{\gsimplex}[1]{\Abs{\Delta^{#1}}}%geomtric n-simplex

\newcommand{\myprod}[3]{{#1} \times_{{#3}}{{#2}}}%fibre product

\newcommand{\Fr}[1]{\operatorname{Fr}\left({#1}\right)}%Frame bundle

\newcommand{\Grass}[3]{\operatorname{Gr}_{{#2}}\left( \D{{#3}}^{{#1}}\right)}%Grassmannian

\newcommand{\Stiefelm}[3]{V_{{#2}}\left( \D{{#3}}^{{#1}}\right)}%Stiefel Manifold

\newcommand{\oStiefelm}[3]{V^o_{{#2}}\left( \D{{#3}}^{{#1}}\right)}%orthonormal Stiefel Manifold

\newcommand{\homotopygrp}[2]{\pi_{{#1}} \left({#2} \right)}%homotopy group

\newcommand{\homotopymap}[2]{\pi \left[ {#1}, {#2} \right]}%homotopy classes of maps

\newcommand{\cohomology}[4]{H^{{#3}}_{\operatorname{{#4}}}\left({#1}\mbox{;} \ {#2} \right)}%cohomology group

\newcommand{\deRham}[2]{H^{{#1}}_{\operatorname{dR}}\left({#2} \right)}%de Rham cohomology group

\newcommand{\Zcohomology}[2]{H^{{#2}}\left({#1}\mbox{;} \ {\D{Z}} \right)}%Integral cohomology group

\newcommand{\EG}[2]{E_{{#2}}{#1}}%universal space

\newcommand{\mydu}[3]{{#1} \sqcup_{{#3}}{{#2}}}%pushout

\newcommand{\normclosure}[1]{\ol{{#1}}^{\norm{\cdot}{}}} %norm closure

\newcommand{\weakclosure}[1]{\ol{{#1}}^{w}} %weak closure

\newcommand{\cone}[1]{\operatorname{cone}\left( {#1} \right)} %cone space

\newcommand{\cylinder}[1]{\operatorname{cyl} \left( {#1} \right)} %cylinder

\newcommand{\cwreplace}[1]{{#1}_{\operatorname{CW}}}%CW-replacement

\newcommand{\hofib}[1]{\operatorname{hofib}\left( {#1} \right)} %homotopy fibre

\newcommand{\hocofib}[1]{\operatorname{hocofib}\left( {#1} \right)} %homotopy cofibre

\newcommand{\CG}[2]{\D{CG}\left( {#1}, {#2} \right)} %complex Grassmannian

\newcommand{\ssphere}[1]{\check{\D{S}}^{{#1}}} %simplicial sphere

\newcommand{\AHSS}[3]{\operatorname{AHSS}\left( {#1} \right)^{{#2}}_{{#3}}} %sophisticated Atiyah-Hirzebruch

%CALCULUS AND ANALYSIS-----------------------------
\newcommand{\norm}[2]{\left\lVert{#1}\right\rVert_{#2}} %norm of a vector

\newcommand{\Dif}[2]{\frac{d{#1}}{d{#2}}} %derivative

\newcommand{\dif}[2]{\frac{\partial {#1}}{\partial {#2}}} %partial derivative

\newcommand{\Interval}[4]{ \left#1 {#2}, {#3} \right#4} %interval

\newcommand{\grad}[1]{\operatorname{grad}\left({#1}\right)} %gradient

\newcommand{\oball}[2]{B \left( {#1}, {#2} \right)}%open ball

\newcommand{\cball}[2]{\ol{B} \left( {#1}, {#2} \right)}%closed ball

\newcommand{\Lp}[2]{L^{{#1}} \left( {#2} \right)}%Lp space

\newcommand{\lp}[2]{\ell^{{#1}} \left( {#2} \right)}%Lp space

\newcommand{\orcom}[1]{{#1}^{\perp}}%orthogonal complement

\newcommand{\myint}[4]{\int_{{#3}}^{{#4}} {#1} \ d{{#2}}}%integration

\newcommand{\normop}[1]{\norm{{#1}}{op}}%operator norm

\newcommand{\normHS}[1]{\norm{{#1}}{\operatorname{HS}}}%Hilbert-Schmidt norm

\newcommand{\supp}[1]{\operatorname{supp}\left( {#1} \right)}%support of function

\newcommand{\Fred}[1]{\operatorname{Fred}\left( {#1} \right)}%Fredholm operators

\newcommand{\ind}[1]{\operatorname{ind}\left( {#1} \right)}%classical index

\newcommand{\Calk}[1]{\operatorname{Calk}\left( {#1} \right)}%Calking algebra

%LIE THEORY----------------------------------------
\newcommand{\Lie}[1]{\mathfrak{{#1}}} %Lie algebra

\newcommand{\commutator}[2]{\left[ {#1}, {#2} \right]}%commutator

%DIFFERENTIAL GEOMETRY----------------------------
\newcommand{\christof}[3]{\Gamma_{{#1} \hspace{0.1em} {#3}}^{\hspace{0.3em {#2}}}} %Christoffel symbol

%NUMBER THEORY------------------------------------
\newcommand{\MOD}[3]{{#1} \equiv {#2} \ \left(\operatorname{mod} \  {#3} \right)}

\newcommand{\zmodp}[1]{\D{Z}/{#1}\D{Z}} %Modulo p integers

\newcommand{\sign}[1]{\operatorname{sign}\left( {#1}\right)}%sign function

\newcommand{\mygcd}[1]{\gcd\left( {#1} \right)} % GCD

%PHYSICS----------------------------------
\newcommand{\quantumev}[1]{\left\langle {#1} \right\rangle}%quantum expected value

%RESEARCH PAPER-----------------------------------
\newcommand{\TR}[2]{\operatorname{TR}^{#1}_{#2}} %equivariant homotopy group

\newcommand{\borelH}[1]{\operatorname{H}^{{\tiny \operatorname{Borel}}}_{#1}} %Borel homology

\newcommand{\simplexcat}[1]{\Delta \downarrow {#1}} %simplex category for a simplicial set X

\newcommand{\GJreal}[1]{\Abs{{#1}}_{\operatorname{GJ}}} %Goerss-Jardine realisation for a simplicial set

\newcommand{\externalprod}[2]{{#1} \widetilde{\times} {#2}} %external product of two bi-simplicial sets

\newcommand{\fullreal}[1]{{\Abs{#1}}_{\operatorname{full}}} %full realisation of a bi-simplicial set

\newcommand{\diagreal}[1]{\Abs{{#1}}_{\operatorname{diag}}} %diagonal realisation of bi-simplicial set

\newcommand{\cofib}[2]{{#1} \rightarrowtail {#2}} %cofibration

\newcommand{\simp}[1]{\operatorname{simp}\left({#1} \right)}%Waldhausen's simp functor

\newcommand{\cofseq}[3]{{#1} \rightarrowtail {#2} \twoheadrightarrow {#3}}%Cofibration sequence

\newcommand{\THH}[2]{\operatorname{THH}\left( {#1}\right)_{{#2}}} %Topological Hochschild Homology

\newcommand{\assem}[1]{\alpha_{{#1}}}%assembly map

\newcommand{\Wh}[2]{\operatorname{Wh}_{{#2}}\left( {#1} \right)}%Whitehead group

\newcommand{\Zariskicohomology}[3]{H^{{#3}}_{\textrm{\tiny Zariski}}\left({#1} \mbox{;} \ {#2} \right)}%Zariski cohomology group

\newcommand{\etalecohomology}[3]{H^{{#3}}_{\textrm{\tiny \'{e}t}}\left({#1} \mbox{;} \ {#2} \right)}%etale cohomology group

\newcommand{\trivialcofib}{\mycal{C} \cap \mycal{W}} %trivial cofibration

\newcommand{\trivialfib}{\mycal{F} \cap \mycal{W}} %trivial fibration

\newcommand{\RamPM}[1]{\dul{P}\left( {#1} \right)}%Ramras' Category of Projective Modules

\newcommand{\Qcon}[1]{Q \left({#1} \right)} %Q-construction

\newcommand{\admmor}[5]{\begin{tikzpicture}
  \node (A) {${#1}$}; 
  \node (B) [right= of A] {${#2}$};
  \node (C) [right= of B] {${#3}$};
  \draw[->>] (B)--node[above] {\small ${#4}$} (A);
  \draw[>->] (B)--node[above] {\small ${#5}$} (C);
\end{tikzpicture}} %morphisms in Q-construction

\newcommand{\Lodayf}[4]{f^{{#1},{#2}}_{{#3},{#4}}}%Loday's f map

\newcommand{\BGL}[1]{BGL \left({#1}\right)} %Classifying space of GL

\newcommand{\BGLp}[1]{\fun{BGL}{{#1}}^{+}} %Plus construction

\newcommand{\Lodaym}[4]{\gamma^{{#1},{#2}}_{{#3},{#4}}}%Loday's multiplication map

\newcommand{\Lodaymh}[4]{\widehat{\gamma}^{{#1},{#2}}_{{#3},{#4}}}%Loday's multiplication map on smash product

\newcommand{\KDL}[1]{K^{\operatorname{DL}}\left( {#1} \right)}%Davis-Luck K-theory spectrum

\newcommand{\Orcat}[1]{\operatorname{Or}\left( {#1} \right)}%orbit category

\newcommand{\actgroupoid}[2]{{#1}\mathsmaller{\int} {#2}}%action groupoid, need \usepackage{relsize}

\newcommand{\twist}[2]{\operatorname{twist}_{{#1},{#2}}} %twist map

\newcommand{\Lodayproda}[2]{ {#1} \ast_{\operatorname{Loday}}{#2}} %Loday product \ast

\newcommand{\Lodayprodb}[2]{ {#1}  \bigstar  {#2}} %Loday product \ast

\newcommand{\KGW}[1]{\mathbb{K}^{\operatorname{GW}}_{{#1}}}%Gersten-Wagoner K-theory spectrum

\newcommand{\Kfree}[1]{\mathbb{K}^{\operatorname{free}}_{{#1}}}%Free K-theory spectrum

\newcommand{\HH}[2]{\operatorname{{\it HH}}_{{#1}} \left( {#2} \right)} %Hochschild homology 

\newcommand{\Stsym}[2]{ \left\{ {#1}, {#2} \right\}_{\mathrm{St}}}%Steinberg Symbol

\newcommand{\Ncyc}[2]{N^{\operatorname{cyc}}_{{#2}} \left( {#1} \right)} %cyclic bar construction

\newcommand{\Lodaya}{\alpha_{\operatorname{\tiny Loday}}} %Loday assembly

\newcommand{\Walda}{\alpha_{\operatorname{\tiny Wald}}} %Waldhausen assembly

\newcommand{\Lodayp}{\gamma_{\operatorname{\tiny Loday}}} %Loday pairing

\newcommand{\Waldp}{\gamma_{\operatorname{\tiny Wald}}} %Waldhausen pairing

\newcommand{\Weibelp}{\gamma_{\operatorname{\tiny Weibel}}} %Weibel pairing

\newcommand{\freep}{\gamma_{\operatorname{\tiny free}}} %pairing for free modules

\newcommand{\WWa}{\alpha_{\operatorname{\tiny WW}}} %Weiss-Williams assembly

\newcommand{\KQ}[1]{\mathbb{K}^{Q}_{{#1}}}% K-theory spectrum in terms of Q-construction (do not confuse with the double Q-construction)

\newcommand{\kgw}[1]{\Bbbk^{\operatorname{gw}}_{{#1}}}%Gersten-Wagoner K-theory spectrum without the K0-factor

\newcommand{\WhG}[2]{\operatorname{Wh}_{{#2}} \left( {#1} \right)}

\newcommand{\KKfree}[1]{K^{\operatorname{free}}_{{#1}}}%Free K-theory space

\newcommand{\KPW}[1]{\mathbb{K}^{\operatorname{PW}}_{{#1}}}%Pedersen-Weibel K-theory spectrum

\newcommand{\Ksmc}[1]{K^{\Box}_{{#1}}}%K-theory space of a symmetric monoidal category

\newcommand{\freea}{\alpha_{\operatorname{\tiny free}}} %free assembly

\newcommand{\Kproj}[1]{\mathbb{K}^{\operatorname{proj}}_{{#1}}}%Idempotent K-theory spectrum

\newcommand{\projp}{\gamma_{\operatorname{\tiny proj}}} %pairing for projective modules

\newcommand{\proja}{\alpha_{\operatorname{\tiny proj}}} %projective assembly

\newcommand{\freestar}{\star_{\operatorname{free}}} %the multiplication map with respect to \freep

\newcommand{\Kahlerdiff}[3]{\Omega^{{#1}}_{\left. {#2} \middle| {#3} \right.}} %Kahler differentials

\newcommand{\naivep}{\gamma_{\operatorname{\tiny naive}}} %naive pairing

\newcommand{\naivestar}{\star_{\operatorname{naive}}} %the multiplication map with respect to \naivep

\newcommand{\naivea}{\alpha_{\operatorname{\tiny naive}}} %naive assembly

%STATISTICS-------------------------------
\newcommand{\Var}[1]{\operatorname{Var}\left( {#1} \right)} % variance

\newcommand{\binomdist}[1]{\operatorname{Binomial}\left( {#1}\right)} % Binomial distribution

\newcommand{\negbinomdist}[1]{\operatorname{NegBinomial}\left( {#1} \right)} % Negative Binomial distribution

\newcommand{\normaldist}[1]{\operatorname{Normal}\left( {#1} \right)} % Normal distribution

\newcommand{\poissondist}[1]{\operatorname{Poisson}\left( {#1} \right)} % Poisson distribution

\newcommand{\uniformdist}[1]{\operatorname{Uniform}\left( {#1} \right)} % Poisson distribution

\newcommand{\geometricdist}[1]{\operatorname{Geometric}\left( {#1} \right)} % Geometric distribution

\newcommand{\conditbar}[2]{ \left. {#1} \middle| {#2} \right.} % conditional bar

\newcommand{\gammadist}[1]{\operatorname{Gamma}\left( {#1} \right)} % Gamma distribution

\newcommand{\betadist}[1]{\operatorname{Beta}\left( {#1} \right)} % Beta distribution

%\END{COMMAND}

\makeatletter
\tikzset{join/.code=\tikzset{after node path={%
\ifx\tikzchainprevious\pgfutil@empty\else(\tikzchainprevious)%
edge[every join]#1(\tikzchaincurrent)\fi}}}

\makeatother

%\tikzset{>=stealth',every on chain/.append style={join},
%        every join/.style={->}}

\newlength{\parindentsave}\setlength{\parindentsave}{\parindent}

\everymath{\displaystyle}

\numberwithin{equation}{subsection} 

\let\emptyset\varnothing

\hypersetup{colorlinks,citecolor=blue,linkcolor=blue}

\declaretheorem[numberwithin=section, shaded={rulecolor=black,
rulewidth=0.5pt, bgcolor={rgb}{1,1,1}}]{Theorem}

%\doublespacing

\setcounter{tocdepth}{4}

\begin{document}
\maketitle

\tableofcontents

\newpage
\section{Problem 1.1}

For each of the following experiments, describe the sample space.

\begin{enumerate}[label = (\alph*), leftmargin=*]
    \item Toss a coin four times.
    \item Count the number of insect-damaged leaves on a plant.
    \item Measure the lifetime (in hours) of a particular brand of light bulb.
    \item Record the weights of 10-day-old rats.
    \item Observe the proportion of defectives in a shipment of electronic components.
\end{enumerate}~\\

\begin{proof}[Solution]~\\

\begin{enumerate}[label= (\alph*), leftmargin=*]
    \item $S = \SET{(x_1, x_2, x_3, x_4)}{\mbox{$x_i$ is either head or tail.}}$
    \item $S = \D{N} \cup \SETT{0} \subseteq \D{Z}$
    \item $S = [0, \infty) \subseteq \D{R}$
    \item $S = (0, \infty) \subseteq \D{R}$
    \item $S = [0, 1] \subseteq \D{R}$
    
\end{enumerate}

\end{proof}

\newpage
\section{Problem 1.2} 

Verify the following identities.

\begin{enumerate}[label = (\alph*), leftmargin=*]
    \item $A - B = A - (A \cap B) = A \cap B^c$
    \item $B = (B \cap A) \cup (B \cap A^c)$
    \item $B - A = B \cap A^c$
    \item $A \cup B = A \cup (B \cap A^c)$
    
\end{enumerate}~\\

\begin{proof}[Solution]~\\

\begin{enumerate}[label = (\alph*), leftmargin=*]
    \item 
    
    \begin{align*}
        A - (A \cap B) &= \SET{x \in A}{x \not\in A \cap B} \\
                       &= \SET{x \in A}{\mbox{$x \not\in A$ and $x \not\in B$}} \\
                       &= \SET{x \in A}{x \not\in B} \\
                       &= A - B \\
                       &= \SET{x \in A}{x \not\in B} \\
                       &= \SET{x \in A}{x \in B^c} \\
                       &= \SET{x}{\mbox{$x \in A$ and $x \in B^c$}} \\
                       &= A \cap B^c
    \end{align*}
    
    \item
    
    \begin{align*}
        (B \cap A) \cup (B \cap A^c) &= \SET{x}{\mbox{$x \in B \cap A$ or $x \in B \cap A^c$}} \\
        &= \SET{x}{\left(\mbox{$x \in B$ and $x \in A$} \right) \ \mbox{or} \ \left(\mbox{$x \in B$ and $x \in A^c$} \right)} \\
        &= \SET{x}{x \in B \ \mbox{and} \left( \mbox{either $x \in A$ or $x \in A^c$} \right)} \\
        &= \SET{x}{x \in B \ \mbox{and} \ x \in A \cup A^c} \\
        &= \SET{x}{x \in B} \\
        &= B
    \end{align*}
    
    \item
    
    \begin{align*}
        B \cap A^c &= \SET{x}{\mbox{$x \in B$ and $x \in A^c$}} \\
                   &= \SET{x}{\mbox{$x \in B$ and $x \not\in A$}} \\
                   &= B-A
    \end{align*}
    
    \item
    
    \begin{align*}
        A \cup (B \cap A^c) &= \SET{x}{\mbox{$x \in A$ or $x \in B \cap A^c$}} \\
                            &= \SET{x}{x \in A \ \mbox{or} \ \left(\mbox{$x \in B$ and $x \in A^c$} \right)} \\
                            &= \SET{x}{\left( x \in A \ \mbox{or} \ x \in B \right) \ \mbox{and} \ \left( x \in A \ \mbox{or} \ x \in A^c \right)} \\
                            &= \SET{x}{\mbox{$x \in A$ or $x \in B$}} \\
                            &= A \cup B
    \end{align*}
\end{enumerate}
\end{proof}

\newpage
\section{Problem 1.3}

Finish the proof of \cite[Theorem 1.1.4 on page 3]{Berger-Casella}. For any events $A$, $B$, and $C$ defined on a sample space $S$, show that

\begin{enumerate}[label= (\alph*), leftmargin=*]
    \item $A \cup B = B \cup A$ and $A \cap B = B \cap A$.
    \item $A \cup (B \cup C) = (A \cup B) \cup C$ and $A \cap (B \cap C) = (A \cap B) \cap C$.
    \item $(A \cup B)^c = A^c \cap B^c$ and $(A \cap B)^c = A^c \cup B^c$.
\end{enumerate}~\\

\begin{proof}[Solution] We prove the statement for union. The case for intersection is analogous.
\begin{enumerate}[label=(\alph*),leftmargin=*]
    \item
    
    \begin{align*}
        A \cup B &= \SET{x}{\mbox{$x \in A$ or $x \in B$}} \\
                 &= \SET{x}{\mbox{$x \in B$ or $x \in A$}} \\
                 &= B \cup A
    \end{align*}
    
    \item
    
    \begin{align*}
        A \cup (B \cup C) &= \SET{x}{\mbox{$x \in A$ or $x \in B \cup C$}} \\
                          &= \SET{x}{x \in A \ \mbox{or} \ \left(x \in B \ \mbox{or} \ x \in C \right)} \\
                          &= \SET{x}{\left( x \in A \ \mbox{or} \ x \in B \right) \ \mbox{or} \ x \in C} \\
                          &= (A \cup B) \cup C
    \end{align*}
    
    \item
    
    \begin{align*}
        A^c \cap B^c &= \SET{x}{\mbox{$x \in A^c$ and $x \in B^c$}} \\
                     &= \SET{x}{\mbox{$x \not\in A$ and $x \not\in B$}} \\
                     &= \SET{x}{\mbox{not $\left( x \in A \ \mbox{or} \ x \in B \right)$}} \\
                     &= \SET{x}{\mbox{not $x \in A \cup B$}} \\
                     &= \left(A \cup B \right)^c
    \end{align*}
\end{enumerate}
\end{proof}

\newpage
\section{Problem 1.4}

For events $A$ and $B$, find formulas for the probabilities of the following events in terms of the quantities $P(A)$, $P(B)$, and $P(A \cap B)$.

\begin{enumerate}[label=(\alph*),leftmargin=*]
    \item either $A$ or $B$ or both
    \item either $A$ or $B$ but not both
    \item at least one of $A$ or $B$
    \item at most one of $A$ or $B$
\end{enumerate}~\\

\begin{proof}[Solution]~\\

\begin{enumerate}[label=(\alph*),leftmargin=*]
    \item 
    
    \begin{align*}
        \fun{P}{A \cup B \cup (A \cap B)} &= \fun{P}{A \cup B} \\
                                          &= P(A) + P(B) - P(A \cap B)
    \end{align*}
    
    \item
    
    \begin{align*}
        \fun{P}{(A \cap B^c) \cup (B \cap A^c)} &= \fun{P}{ \left(A \cup (B \cap A^c) \right) \cap \left(B^c \cup (B \cap A^c) \right)} \\
        &= \fun{P}{\left(A \cup B \right) \cap \left(B^c \cup A^c \right)} \\
        &= \fun{P}{\left( A \cup B \right) \cap \left( A \cap B \right)^c} \\
        &= \fun{P}{\left( A \cup B \right) - \left( A \cap B \right)} \\
        &= \fun{P}{A} + \fun{P}{B} - 2\fun{P}{A \cap B}
    \end{align*}
    
    \item
    
    \begin{align*}
        \fun{P}{A \cup B} = P(A) + P(B) - P(A \cap B)
    \end{align*}
    
    \item
    
    \begin{align*}
        P((A \cap B)^c) = 1 - P(A \cap B)
    \end{align*}
\end{enumerate}
\end{proof}

\newpage
\section{Problem 1.5} Approximately one-third of all human twins are identical (one-egg) and two-thirds are fraternal (two-egg) twins. Identical twins are necessarily the same sex, with male and female being equally likely. Among fraternal twins, approximately one-fourth are both female, one-fourth are both male, and half are one male and one female. Finally, among all U.S. births, approximately 1 in 90 is a twin birth. Define the following events:

\begin{align*}
    A &= \SETT{\mbox{a U.S. birth results in twin females}} \\
    B &= \SETT{\mbox{a U.S. birth results in identical twins}} \\
    C &= \SETT{\mbox{a U.S. birth results in twins}}
\end{align*}

\begin{enumerate}[label=(\alph*), leftmargin=*]
    \item State, in words, the event $A \cap B \cap C$.
    \item Find $P(A \cap B \cap C)$.
\end{enumerate}~\\

\begin{proof}[Solution]~\\

\begin{enumerate}[label=(\alph*),leftmargin=*]
    \item A U.S. birth results in identical twin females.
    
    \item
    
    \begin{align*}
        P(A \cap B \cap C) &= \frac{1}{90} \cdot \frac{1}{3} \cdot \frac{1}{2} \\
                           &= \frac{1}{540}
    \end{align*}
\end{enumerate}
\end{proof}

\newpage
\section{Problem 1.6} Two pennies, one with $\fun{P}{\mbox{head}} = u$ and one with $\fun{P}{\mbox{head}} = w$, are to be tossed together independently. Define

\begin{align*}
    p_0 &= \fun{P}{\mbox{0 heads occur}}, \\
    p_1 &= \fun{P}{\mbox{1 heads occur}}, \\
    p_2 &= \fun{P}{\mbox{2 heads occur}}. \\
\end{align*}
Can $u$ and $w$ be chosen such that $p_0 = p_1 = p_2$? Prove your answer.

\begin{proof}[Solution]
We cannot. 

We have

\begin{align*}
    p_0 &= (1-u)(1-w) \\
        &= 1-w-u+uw, \\
        \\
    p_1 &= u(1-w) + (1-u)w \\
        &= u+w -2uw, \\
        \\
    p_2 &= uw.
\end{align*}

If $p_0 = p_2$, then
\[ u + w = 1. \]

If $p_0 = p_1$, then
\begin{align*}
    p_0 &= p_1 \\
    1 - w - u + uw &= u + w - 2uw & \left(\mbox{if $p_0 = p_2$} \right) \\
    uw &= 1 - 2uw \\
    uw &= \frac{1}{3}
\end{align*}

Therefore, we have

\[ \left\{ \begin{array}{cc}
    u + w &= 1 \\
    uw &= \frac{1}{3}
\end{array} \right. \]

which has no solution.
\end{proof}

\newpage
\section{Problem 1.7}

Refer to the dart game of \cite[Example 1.2.7 on page 8]{Berger-Casella}. Suppose we do not assume that the probability of hitting the dart board is 1, but rather is proportional to the area of the dart board. Assume that the dart board is mounted on a wall that is hit ith probability 1, and the wall has area $A$.

\begin{enumerate}[label=(\alph*), leftmargin=*]
    \item Using the fact that the probability of hitting a region is proportional to area, construct a probability function for $\fun{P}{\mbox{scoring $i$ points}}$, $i = 0, \cdots, 5$. (No points are scored if the dart board is not hit.)
    
    \item Show that the conditional probability distribution $\fun{P}{\mbox{scoring $i$ points} \ \mid \ \mbox{board is hit}}$ is exactly the probability distribution of \cite[Example 1.2.7 on page 8]{Berger-Casella}.
\end{enumerate}~\\

\begin{proof}[Solution]

Let $r$ be the radius of the dart board.

\begin{enumerate}[label=(\alph*),leftmargin=*]
    \item \label{1.7a}
    
    \[ \fun{P}{\mbox{scoring $i$ points}} = \left\{ \begin{array}{cl}
         1 - \frac{\pi r^2}{A} & \mbox{if $i = 0$},  \\
         \frac{\pi r^2}{A} \left[ \frac{(6-i)^2 - (5-i)^2}{5^2} \right] & \mbox{if $ 1 \leq i \leq 5$}.
    \end{array} \right. \]
    
    \item
    
    \begin{align*}
        \fun{P}{\mbox{scoring $i$ points} \ \mid \ \mbox{board is hit}} &= \frac{\fun{P}{\mbox{scoring $i$ points and board is hit}}}{\fun{P}{\mbox{board is hit}}} \\
        &= \frac{\frac{\pi r^2}{A} \left[ \frac{(6-i)^2 - (5-i)^2}{5^2} \right]}{\frac{\pi r^2}{A}} & \left( \mbox{part (a)} \right) \\
        &= \frac{(6-i)^2 - (5-i)^2}{5^2}
    \end{align*}
\end{enumerate}
\end{proof}

\newpage
\section{Problem 1.8}

Again refer to the game of darts explained in \cite[Example 1.2.7 on page 8]{Berger-Casella}.

\begin{enumerate}[label=(\alph*),leftmargin=*]
    \item Derive the general formula for the probability of scoring $i$ points.
    \item Show that $\fun{P}{\mbox{scoring $i$ points}}$ is a decreasing function of $i$, that is, as the points increase, the probability of scoring them decreases.
    \item Show that $\fun{P}{\mbox{scoring $i$ points}}$ is a probability function according to the Kolmogorov Axioms.
\end{enumerate}~\\

\begin{proof} Let $P(i) = \fun{P}{\mbox{scoring $i$ points}}$.

\begin{enumerate}[label=(\alph*),leftmargin=*]
    \item By \myref{Problem 1.7}{1.7a}, we have
    
    \[ \fun{P}{i} = \left\{ \begin{array}{cl}
         1 - \frac{\pi r^2}{A} & \mbox{if $i = 0$},  \\
         \frac{\pi r^2}{A} \left[ \frac{(6-i)^2 - (5-i)^2}{5^2} \right] & \mbox{if $ 1 \leq i \leq 5$}.
    \end{array} \right. \]
    
    \item Follows immediately from part (a).
    
    \item Clearly, we have $P(i) \geq 0$ for all $i$.
    
    Next, if $S$ is the sample space, then
    
    \begin{align*}
        P(S) &= \fun{P}{\mbox{hitting the wall}} \\
             &= 1 & \left( \mbox{by definition} \right).
    \end{align*}
    
    Finally, if $i \not= j$, then
    
    \begin{align*}
        \fun{P}{\mbox{$i$ or $j$}} &= \fun{P}{\mbox{hitting $i$-th region or hitting $j$-th region}} \\
        &= \fun{P}{\mbox{hitting $i$-th region}} + \fun{P}{\mbox{hitting $j$-th region}} & \left( \mbox{regions are disjoint} \right) \\
        &= P(i) + P(j).
    \end{align*}
\end{enumerate}
\end{proof}

\newpage
\section{Problem 1.9}
Prove the general version of DeMorgan's Laws. Let $\SET{A_{\alpha}}{\alpha \in \Gamma}$ be a (possibly uncountable) collection of sets. Prove that

\begin{enumerate}[label=(\alph*),leftmargin=*]
    \item  $\left( \bigcup_{\alpha} A_{\alpha} \right)^c = \bigcap_{\alpha} A^c_{\alpha}$.
    
    \item $\left( \bigcap_{\alpha} A_{\alpha} \right)^c = \bigcup_{\alpha} A^c_{\alpha}$.
\end{enumerate}~\\

\begin{proof}[Solution] We prove part (a) here, part (b) is analogous.

\begin{align*}
    \left( \bigcup_{\alpha} A_{\alpha} \right)^c &= \SET{x}{x \not\in \bigcup_{\alpha} A_{\alpha}} \\
    &= \SET{x}{\mbox{$x \not\in A_{\alpha}$ for all $\alpha \in \Gamma$}} \\
    &= \SET{x}{\mbox{$x \in A_{\alpha}^c$ for all $\alpha \in \Gamma$}} \\
    &= \bigcap_{\alpha} A_{\alpha}^c
\end{align*}
\end{proof}

\newpage
\section{Problem 1.10}

Formulate and prove a version of DeMorgan's Laws that applies to a finite collection of sets $A_1, \cdots, A_n$.

\begin{proof}[Solution]
$\left( \bigcup_{i=1}^n A_i \right)^c = \bigcap_{i=1}^n A_i^c$ and $\left( \bigcap_{i=1}^n A_i \right)^c = \bigcup_{i=1}^n A_i^c$
\end{proof}

\newpage
\section{Problem 1.11}

Let $S$ be a sample space.

\begin{enumerate}[label=(\alph*),leftmargin=*]
    \item Show that the collection $\mycal{B} = \SETT{\emptyset, S}$ is a sigma algebra.
    
    \item Let $\mycal{B} = \SETT{\mbox{all subsets of $S$, including $S$ itself}}$. Show that $\mycal{B}$ is a sigma algebra.
    
    \item Show that the intersection of two sigma algebras is a sigma algebra.
\end{enumerate}~\\

\begin{proof}[Solution] Refer to \cite[Definition 1.2.1 on page 6]{Berger-Casella} for the definition of sigma algebra.

\begin{enumerate}[label=(\alph*),leftmargin=*]
    \item Trivial.
    \item Trivial.
    \item The first two conditions are trivial. We show the intersection $\mycal{B}_1 \cap \mycal{B}_2$ of two sigma algebras is closed under countable unions.
    
    If $A_1, A_2, \cdots \in \mycal{B}_1 \cap \mycal{B}_2$, then $A_i \in \mycal{B}_j$ for all $i$ and $j$. So $\bigcup_{i = 1}^{\infty} A_i \in \mycal{B}_j$ for all $j$ by definition of an sigma algebra.
\end{enumerate}
\end{proof}

\newpage
\section{Problem 1.12}

It was noted in \cite[page 9]{Berger-Casella} that statisticians who follow the deFinetti school do not accept the Axiom of Countable Additivity, instead adhering to the Axiom of Finite Additivity.

\begin{enumerate}[label=(\alph*),leftmargin=*]
    \item Show that the Axiom of Countable Additivity implies Finite Additivity.
    \item Alghtouh, by iteself, the Axiom of Finite Additivity does not imply Countable Additivity, suppose we suplement it with the following. Let $A_1 \supseteq A_2 \supseteq \cdots \supseteq A_n \supseteq \cdots$ be an infinite sequence of nested sets whose limit is the empty set, which we denote by $A_n \downarrow \emptyset$. Consider the following:
    \begin{center}
        {\bf Axiom of Countinuty:} If $A_n \downarrow \emptyset$, then $P(A_n) \rightarrow 0$.
    \end{center}
    Prove that the Axiom of Continuity and the Axiom of Finite Additivity imply Countable Additivity.
\end{enumerate}~\\

\begin{proof}[Solution]~\\
\begin{enumerate}[label=(\alph*),leftmargin=*]
    \item If $A_1, A_2$ are two disjoint sets in the sigma algebra $\mycal{B}$, then put $A_i := \emptyset$ for $i \geq 3$. Then
    
    \begin{align*}
        \fun{P}{\bigcup_{i = 1}^{\infty} A_i} &= \sum_{i=1}^{\infty} \fun{P}{A_i} & \left(\mbox{Countable Additivity} \right) \\
        &= \fun{P}{A_1} + \fun{P}{A_2} + \sum_{i=3}^{\infty} \fun{P}{A_i} \\
        &= \fun{P}{A_1} + \fun{P}{A_2}.
    \end{align*}
    
    \item If $\SETT{B_i}_{i=1}^{\infty} \subseteq \mycal{B}$ is a sequence of pairwise disjoint subsets from the sigma algebra, define
    \begin{align*}
        A &:= \bigcup_{i=1}^{\infty} B_i, \\
        A_j &:= A - \bigcup_{i=1}^j B_j \ \mbox{for all $j$}.
    \end{align*}
    Then we have $\bigcup_{j=1}^{\infty} A_j = A$ and $A_j \downarrow \emptyset$. Therefore,
    
    \begin{align*}
        \fun{P}{\bigcup_{i=1}^{\infty} B_i} &= \fun{P}{\bigcup_{i=1}^n B_i \cup \bigcup_{i \geq n+1} B_i} \\
                                            &= \sum_{i=1}^{n}\fun{P}{B_i} + \fun{P}{\bigcup_{i \geq n + 1} B_i} & \left( \mbox{Finite Addivity} \right) \\
                                            &= \sum_{i=1}^{n}\fun{P}{B_i} + \fun{P}{A_n}.
    \end{align*}
    The result follows from taking $n \rightarrow \infty$.
\end{enumerate}
\end{proof}

\newpage
\section{Problem 1.13}

If $\fun{P}{A} = \frac{1}{3}$ and $\fun{P}{B^c} = \frac{1}{4}$, can $A$ and $B$ be disjoint? Explain.

\begin{proof}[Solution]
They are not disjoint. We have

\begin{align*}
    1 &\geq \fun{P}{A \cup B} \\
      &= \fun{P}{A} + \fun{P}{B} - \fun{P}{A \cap B} \\
      &= \frac{1}{3} + \frac{3}{4} - \fun{P}{A \cap B}
\end{align*}
which gives $\fun{P}{A \cap B} > 0$.
\end{proof}

\newpage
\section{Problem 1.14}

Suppose that a sample space $S$ has $n$ elements. Prove that the number of subsets that can be formed from the elements of $S$ is $2^n$.

\begin{proof}[Solution]
There are $\binom{n}{k}$ subsets with exactly $0 \leq k \leq n$ elements. The result follows immediately from

\[ \sum_{k=0}^n \binom{n}{k} = (1+1)^n = 2^n. \]
\end{proof}

\newpage
\section{Problem 1.15}

Finish the proof of \cite[Theorem 1.2.14 on page 13]{Berger-Casella}. Use the result established for $k = 2$ as the basis of an induction argument.

\begin{proof}[Solution] The base case $k = 2$ is completed on \cite[page 14]{Berger-Casella}. Assume the statement is true for some integer $k \geq 2$.

When there are $ k + 1$ separate tasks, the inductive hypothesis says there are $\prod_{i=1}^k n_k$ ways to complete the first $k$ tasks. If there are $n_{k+1}$ ways to complete the last task, then the base case says there are $\prod_{i=1}^{k+1} n_k$ ways to complete the $k+1$ tasks.

Therefore, the claim follows from induction.
\end{proof}

\newpage
\section{Problem 1.16}

How many different sets of initials can be formed if every person has one surname and 
\begin{enumerate}[label=(\alph*),leftmargin=*]
    \item exactly two given names?
    \item either one or two given names?
    \item either one or two or three given names?
\end{enumerate}~\\

\begin{proof}[Solution]~\\
\begin{enumerate}[label=(\alph*),leftmargin=*]
    \item $26 \cdot 26 \cdot 26 = 26^3$
    \item $26(26 + 26^2) = 26^2 + 26^3$
    \item $26(26 + 26^2 + 26^3)$
\end{enumerate}
\end{proof}

\newpage
\section{Problem 1.17}

In the game of dominoes, each piece is marked with two numbers. The pieces are symmetrical so that the number pair is not ordered (so, for example, $(2,6) = (6,2)$). How many different pieces can be formed using the numbers $1, 2, \cdots, n$?

\begin{proof}[Solution]
If the first number is $k$, then there are $n-k$ choices for the second number. So there number of pairs is given by

\begin{align*}
    \sum_{k=1}^n n-k = \sum_{k=1}^n k = \frac{n(n+1)}{2}.
\end{align*}
\end{proof}

\newpage
\section{Problem 1.18}

If $n$ balls are placed at random into $n$ cells, find the probability that exactly one cell remains empty.

\begin{proof}[Solution]
In order the have exactly one cell empty, we must have two balls placed at the same cell.

There are total of $n^n$ placements. There are $\binom{n}{2}$ ways to choose the two balls to be placed at the same cell. There are $\binom{n}{1}$ ways to choose to cell to hold two balls. Therefore, the answer is

\[ \frac{\binom{n}{2} \binom{n}{1}}{n^n}. \]
\end{proof}

\newpage
\section{Problem 1.19}

If a multivariate function has continuous partial derivatives, the order in which the derivatives are calculated does not matter. Thus, for example, the function $f(x,y)$ of two variables has equal third partials

\[ \frac{\partial^3}{\partial x^2 \partial y} f(x,y) = \frac{\partial^3}{\partial y \partial x^2} f(x,y). \]

\begin{enumerate}[label=(\alph*),leftmargin=*]
    \item How many fourth partial derivatives does a function of three variables have?
    \item Prove that a function of $n$ variables has $\binom{n+r-1}{r}$ $r$-th partial derivatives.
\end{enumerate}~\\

\begin{proof}[Solution]~\\

\begin{enumerate}[label=(\alph*),leftmargin=*]
    \item By part (b), there are $\binom{3 + 4 - 1}{4} = 15$ derivatives.
    \item We are choosing $k$ variables from $n$ with replacement. The number of combination with repetition allowed is $\binom{n+k - 1}{k}$.
\end{enumerate}
\end{proof}

\newpage
\section{Problem 1.20}

My telephone rings 12 times each week, the calls being randomly distributed among the 7 days. What is the probability that I get at least one call each day?

\begin{proof}[Solution]
Let $A_i$ be the number of ways to distribute 12 phone calls among $i$ days in a week. Then

\[ A_i = \binom{7}{i} i^{12}, \]
and $A_7 = 7^{12}$ is the total number of ways to distribute 12 phone calls in a week. Now,

\begin{align*}
    \mbox{at least one call each day} &= \mbox{total number} - \mbox{no calls on 1 day} + \\
    & \ \ \ \ \mbox{no calls on 2 days} - \mbox{no calls on 3 days} + \\
    & \ \ \ \ \mbox{no calls on 4 days} - \mbox{no calls on 5 days} + \\
    & \ \ \ \ \mbox{no calls on 6 days} \\
    &= 7^{12} + \sum_{i=1} (-1)^i A_i \\
    &= 3162075840.
\end{align*}

Therefore, the required probability is

\[ \frac{3162075840}{7^{12}} \approx 0.2285. \]
\end{proof}

\newpage
\section{Problem 1.21}
A closet contains $n$ pairs of shoes. If $2r$ shoes are chosen at random $(2r < n)$, what is the probability that there will be no matching pair in the sample?

\begin{proof}[Solution]
To have no matching pair, one can only choose at most one shoe from each pair. There are $\binom{n}{2r}$ ways to choose $2r$ pairs from $n$ pairs. For each pair, we then have to choose left one or right. So there are $\binom{n}{2r} 2^{2r}$ non-matching shoes. The required probability is:

\[ \frac{\binom{n}{2r} 2^{2r}}{\binom{2n}{2r}}. \]
\end{proof}

\newpage
\section{Problem 1.22}

\begin{enumerate}[label=(\alph*),leftmargin=*]
    \item In a draft lottery containing the 66 days of the year (including February 29), what is the probability that the first 180 days drawn (without replacement) are evenly distributed among the 12 months?
    \item What is the probability that the first 30 days drawn contain none from September?
\end{enumerate}~\\

\begin{proof}[Solution]~\\

\begin{enumerate}[label=(\alph*),leftmargin=*]
    \item We have to pick 15 days from each month:
    \[ 7 \binom{31}{15} \cdot 4 \binom{30}{15} \cdot \binom{29}{15}. \]
    So the required probability is:
    \[ \frac{28 \binom{31}{15}\binom{30}{15}\binom{29}{15}}{\binom{366}{180}}. \]
    
    \item The required probability is
    
    \[ \frac{\binom{366-30}{30}}{\binom{366}{30}} = \frac{\binom{336}{30}}{\binom{366}{30}}. \]
\end{enumerate}
\end{proof}

\newpage
\section{Problem 1.23}

Two people each toss a fair coin $n$ times. Find the probability that they will toss the same number of heads.

\begin{proof}[Solution]

We begin with the following identity

\begin{equation}
    \sum_{k=0}^n \binom{n}{k}^2 = \binom{2n}{n}.
\end{equation}
To prove this, first note that
\[ \binom{n+m}{r} = \sum_{k=0}^{r} \binom{n}{k} \binom{m}{r-k}. \]
Then we have

\begin{align*}
    \binom{2n}{n} &= \sum_{k=0}^n \binom{n}{k} \binom{n}{n-k} \\
                  &= \sum_{k=0}^n \binom{n}{k}^2.
\end{align*}

Now, the probability for $A$ to get $k$ heads out $n$ tosses is $\binom{n}{k} \left(\frac{1}{2} \right)^n$. Therefore, the required probability is

\begin{align*}
    \sum_{k=0}^n \left[ \binom{n}{k} \left(\frac{1}{2} \right)^n \right]^2 &= \sum_{k=0}^n \binom{n}{k}^2 \left(\frac{1}{4} \right)^n \\
    &= \binom{2n}{n} \left( \frac{1}{4} \right)^n.
\end{align*}
\end{proof}

\newpage
\section{Problem 1.24}

Two players, $A$ and $B$, alternately and independently flip a coin and the first player to obtain a head wins. Assume player $A$ flips first.

\begin{enumerate}[label=(\alph*),leftmargin=*]
    \item If the coin is fair, what is the probability that $A$ wins?
    \item Suppose that $\fun{P}{\mbox{head}} = p$, not necessarily $\frac{1}{2}$. What is the probability that $A$ wins?
    \item Show that for all $p$, $0 < p < 1$, $\fun{P}{\mbox{$A$ wins}} > \frac{1}{2}$. (Hint: Try to write $\fun{P}{\mbox{$A$ wins}}$ in terms of the events $E_1, E_2, \cdots$, where $E_i = \SETT{\mbox{head first appears on $i$th toss}}$.)
\end{enumerate}~\\

\begin{proof}[Solution]~\\

\begin{enumerate}[label=(\alph*),leftmargin=*]
    \item 
    
    \begin{align*}
        \fun{P}{\mbox{$A$ wins}} &= \sum_{i=1}^{\infty} \fun{P}{\mbox{$A$ wins at $i$-th toss}} \\
        &= \frac{1}{2} + \left(\frac{1}{2} \right)^2 \frac{1}{2} + \left(\frac{1}{2} \right)^4 \frac{1}{2} + \cdots \\
        &= \frac{2}{3}
    \end{align*}
    
    \item If $\fun{P}{\mbox{head}} = p$, then
    
    \begin{align*}
        \fun{P}{\mbox{$A$ wins at $i$-th toss}} &= (1-p)^{2i-2}p
    \end{align*}
    Thus,
    
    \begin{align*}
        \fun{P}{\mbox{$A$ wins}} &= \sum_{i=1}^{\infty} (1-p)^{2i-2}p \\
        &= \frac{p}{1-(1-p)^2}
    \end{align*}
    
    \item
    
    \begin{align*}
        & 0 < p < 1 \\
        \Rightarrow & -p^2 < 0 \\
        \Rightarrow & 1-1+2p-p^2 < 2p \\
        \Rightarrow & 1-(1-p)^2 < 2p \\
        \Rightarrow & \frac{p}{1-(1-p)^2} > \frac{1}{2}
    \end{align*}
\end{enumerate}
\end{proof}

\newpage
\section{Problem 1.25}

The Smiths have two children. At least one of them is a boy. What is the probability that both children are boys? 

\begin{proof}[Solution]
We know

\begin{align*}
    \fun{P}{\mbox{at least one child is a boy}} &= \frac{3}{4}, \\
    \fun{P}{\mbox{both children are boys}, \mbox{\mbox{at least one child is a boy}}} &= \frac{3}{4} \cdot \frac{1}{3} \\
                &= \frac{1}{4}.
\end{align*}
Therefore,

\begin{align*}
    \fun{P}{\mbox{both children are boys} \mid \mbox{\mbox{at least one child is a boy}}} &= \frac{\frac{1}{4}}{\frac{3}{4}} \\
                      &= \frac{1}{3}.
\end{align*}
\end{proof}

\newpage
\section{Problem 1.26} 

A fair die is cast until a 6 appears. What is the probability that it must be cast more than five times?

\begin{proof}[Solution]
Let $E_i$ be the event that 6 appears at the $i$-th cast. Then

\[ \fun{P}{E_i} = \left( \frac{5}{6} \right)^{i-1} \left( \frac{1}{6} \right). \]
The required probability is

\begin{align*}
    1 - \sum_{i=1}^5 \fun{P}{E_i} = \frac{3125}{7776}.
\end{align*}
\end{proof}

\newpage
\section{Problem 1.27}

Verify the following identities for $n \geq 2$.

\begin{enumerate}[label=(\alph*),leftmargin=*]
    \item $\sum_{k=0}^n (-1)^k \binom{n}{k} = 0$
    \item $\sum_{k=1}^n k \binom{n}{k} = n2^{n-1}$
    \item $\sum_{k=1}^n (-1)^{k+1} k \binom{n}{k} = 0$
\end{enumerate}~\\

\begin{proof}[Solution]~\\
\begin{enumerate}[label=(\alph*),leftmargin=*]
    \item Look at the binomial expansion for $(x-1)^n$. Put $x = 1$ and the result follows.
    
    \item Look at the binomial expansion for $(x+1)^n$. Take derivative on both sides and the result follows from putting $x = 1$.
    
    \item 
    
    \begin{align*}
        \sum_{k=1}^{n} (-1)^{k+1} k \binom{n}{k} &= \sum_{k=1}^n (-1)^{k+1} n \binom{n-1}{k-1} \\
        &= n \left[ \sum_{k=1}^n (-1)^{k+1} \binom{n-1}{k-1} \right] \\
        &= n \left[ \sum_{k=0}^{n-1} (-1)^k \binom{n-1}{k} \right] \\
        &= 0 & \mbox{by part (a)}.
    \end{align*}
\end{enumerate}
\end{proof}

\newpage
\section{Problem 1.28}

Prove

\[ \lim_{n \rightarrow \infty} \frac{n!}{n^{n+ \frac{1}{2}} e^{-n}} = \mbox{constant}. \]

(See \cite[page 40]{Berger-Casella} for the exact problem statement.)

\begin{proof}[Solution]
The average of the two intergrals

\begin{align*}
 \int_0^n \log(x) dx, \ \ \int_1^{n+1} \log(x) dx
\end{align*}
is given by

\[ \frac{(n+1)\log(n+1) + n \log(n)}{2} \]
\end{proof}

\newpage
\nocite{*}
\printbibliography

\end{document}