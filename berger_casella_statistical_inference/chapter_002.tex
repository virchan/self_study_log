\documentclass[12pt,letterpaper,reqno]{amsart}


\usepackage{lipsum} %separate the paragraphs of the dummy text into TEX-paragraphs.

\usepackage{tikz} %commutative diagrams

\usepackage{setspace} %Provides support for setting the spacing between lines in a document.

\usepackage{amsmath,amssymb} %provides miscellaneous enhancements for improving the information structure and printed output of documents containing mathematical formulas. e.g., \DeclareMathOperator, \sin, \lim

\usepackage{enumitem} %additional options for \enumerate

\usepackage{subfig} %provides support for the manipulation and reference of small or ‘sub’ figures and tables within a single figure or table environment.

\usepackage{framed} %box, shaded, put a left line around a region.

\usepackage{etoolbox} %provides LATEX frontends to some of the new primitives provided by e-TEX as well as some generic tools which are not strictly related to e-TEX but match the profile of this package.

\usepackage{bm} %Access bold symbols in maths mode

\usepackage{mdframed} %Framed environments that can split at page boundaries

\usepackage{mathrsfs} %Support for using RSFS fonts in maths

\usepackage{centernot} %Centred \not command

\usepackage{thmtools} %Extensions to theorem environments

\usepackage{thm-restate} %Restate Theorem

\usepackage{fancyvrb} %Sophisticated verbatim text

\usepackage{dsfont} %?

\usepackage{bbm} %Blackboard variants of Computer Modern fonts.

\usepackage{breqn} %Automatic line breaking of displayed equations

\usepackage{adjustbox} %The package provides several macros to adjust boxed content.

\usepackage{tabularx} %The package defines an environment tabularx, an extension of tabular which has an additional column designator, X, which creates a paragraph-like column whose width automatically expands so that the declared width of the environment is filled. (Two X columns together share out the available space between them, and so on.)

\usepackage{calligra} %Calligraphic font

\usepackage[
    backend=bibtex, 
    natbib=true, 
    bibstyle=verbose, citestyle=verbose,    % bibstyle extensively modifed below
    doi=true, url=true,                     % excluded from citations below
    citecounter=true, citetracker=true,
    block=space, 
    backref=false, backrefstyle=two,
    abbreviate=false,
    isbn=true, maxbibnames=4, maxcitenames=4,
    style=alphabetic
]{biblatex} %Adding Reference to your paper.

\usepackage[margin=1in]{geometry} %edit margin of your document

\usepackage[makeindex]{imakeidx} %making index

\usepackage{empheq} %The package provides a visual markup extension to amsmath.

\usepackage{longtable} %allow longtable

\usepackage[mathscr]{eucal}%change mathscr

\usepackage{xpatch} %fix skipbelow in mdframe

\usepackage{standalone}  %need this to include the tikz picture created in different files.

\usepackage{wasysym}

\usepackage{bashful}

\usepackage{hyperref,cleveref} %Extensive support for hypertext in LATEX (ALWAYS LOAD LAST!) %this loads the packages


\makeindex

\author{Virgil Chan}
\title{Casella-Berger \\ Statistical Inference Solution: \\ Chapter 2}
\date{August 1, 2022}


\usetikzlibrary{patterns,positioning,arrows,chains,matrix,positioning,scopes} %options for tikzpicture

\addbibresource{bibliography.bib} %put your reference here


\makeatletter
\patchcmd{\@maketitle}
  {\ifx\@empty\@dedicatory}
  {\ifx\@empty\@date \else {\vskip3ex \centering\footnotesize\@date\par\vskip1ex}\fi
   \ifx\@empty\@dedicatory}
  {}{}
\patchcmd{\@adminfootnotes}
  {\ifx\@empty\@date\else \@footnotetext{\@setdate}\fi}
  {}{}{}
\makeatother

\makeatletter
\xpatchcmd{\endmdframed}
  {\aftergroup\endmdf@trivlist\color@endgroup}
  {\endmdf@trivlist\color@endgroup\@doendpe}
  {}{}
\makeatother
%\topsep
\newtheoremstyle{break}
  {\topsep}% Space above
  {\topsep}% Space below
  {\it}% Body font
  {}% Indent amount
  {\bfseries}% Theorem head font
  {}% Punctuation after theorem head
  {\topsep}% Space after theorem head, ' ', or \newline
  {\thmname{#1}\thmnumber{ #2} \thmnote{(#3)}}% Theorem head spec (can be left empty, meaning `normal')

\mdfdefinestyle{box}{
     linecolor=white,
    skipabove=\topsep,
    skipbelow=\topsep,
    innerbottommargin=\topsep,
}  %

\theoremstyle{break}
\newmdtheoremenv[style=box]{theorem}{Theorem}[section]
\newmdtheoremenv[style=box]{lemma}[theorem]{Lemma}
\newmdtheoremenv[style=box]{proposition}[theorem]{Proposition}
\newmdtheoremenv[style=box]{corollary}[theorem]{Corollary}
\newmdtheoremenv[style=box]{definition}[theorem]{Definition}
\newmdtheoremenv[style=box]{addendum}[theorem]{Addendum}
\newmdtheoremenv[style=box]{conjecture}[theorem]{Conjecture}
\newmdtheoremenv[style=box]{question}[theorem]{Question}
\newmdtheoremenv[style=box]{condit}[theorem]{Condition}

\makeatletter
\def\th@plain{%
  \thm@notefont{}% same as heading font
  \normalfont % body font
}
\def\th@definition{%
  \thm@notefont{}% same as heading font
  \normalfont % body font
}
\makeatother

\newtheoremstyle{exampstyle}
{\topsep} % Space above
{\topsep} % Space below
{} % Body font
{} % Indent amount
{\bfseries} % Theorem head font
{} % Punctuation after theorem head
{\topsep} % Space after theorem head
{\thmname{#1}\thmnumber{ #2} \thmnote{(#3)}} % Theorem head spec (can be left empty, meaning `normal')

\theoremstyle{exampstyle}
\newtheorem{example}[theorem]{Example}
\newtheorem{remark}[theorem]{Remark}


\pdfpagewidth 8.5in
\pdfpageheight 11in

    
%\BEGIN{COMMAND}

%SHORT HAND NOTATION-------------------------------
\newcommand{\D}[1]{\mathbb{#1}} %short hand mathbb command

\newcommand{\Dm}[1]{\mathbbm{#1}} %mathbb for numbers

\newcommand{\mycal}[1]{\mathcal{{#1}}} %short hand mathcal

\newcommand{\scr}[1]{\mathscr{{#1}}} %short hand mathscr

\newcommand{\ve}{\varepsilon} %short hand epsilon

\newcommand\numberthis{\addtocounter{equation}{1}\tag{\theequation}}%label equation in \align*

\newcommand{\dund}[1]{\underline{\underline{{#1}}}}%Double underline

\newcommand{\Frechet}{Fr\'{e}chet}%Frechet space

\newcommand{\ol}[1]{\overline{{#1}}}%overline

\newcommand{\ul}[1]{\underline{{#1}}}%underline

\newcommand{\dul}[1]{\ul{\ul{{#1}}}}%double underline

\newcommand{\Cech}{\v{C}ech} %Cech

\newcommand{\Nother}{N\"{o}ther} %Nother

\newcommand{\Notherian}{N\"{o}ther} %Notherian

\newcommand{\wt}[1]{\widetilde{{#1}}} %short hand for \widetilde

\newcommand{\fun}[2]{{#1}\! \left( {#2} \right)}%short hand for function

\newcommand{\on}[1]{\operatorname{{#1}}} %short hand for \operatorname

\newcommand{\Kp}[1]{K^+_{{#1}}} %short hand for plus-construction

\newcommand{\Knp}[1]{K_{{#1}}} %short hand for before plus-construction

\newcommand{\btimes}{\boxtimes} %short hand for box tensor

\newcommand{\bplus}{\boxplus} %short hand for box sum

\newcommand{\myline}{\par\noindent\rule{\textwidth}{0.4pt}} %horizontal line

\newcommand{\wh}[1]{\widehat{{#1}}} %short hand for \widetilde

\newcommand{\myref}[2]{\hyperref[{#2}]{{#1}~\ref{#2}}} %short hand for \hyperref

\newcommand{\myeqref}[2]{\hyperref[{#2}]{{#1}~\eqref{#2}}} %short hand for \hyperref for equations

%SET THEORY----------------------------------------
\newcommand{\SET}[2]{\left\{ {#1} \ \middle| \ {#2} \right\}} %short hand notation for set

\newcommand{\SETT}[1]{\left\{ {#1} \right\}} %short hand notation for set without constraint

\newcommand{\floor}[1]{\left \lfloor{#1}\right \rfloor } %floor function

\newcommand{\Rng}[1]{\operatorname{Rng}\left(#1\right)} %range of a function

\newcommand{\id}[1]{\operatorname{id}_{{#1}}} %identity map

\newcommand\restr[2]{{% we make the whole thing an ordinary symbol
  \left.\kern-\nulldelimiterspace % automatically resize the bar with \right
  #1 % the function
  \vphantom{\big|} % pretend it's a little taller at normal size
  \right|_{#2} % this is the delimiter
  }} %restriction of a function

\newcommand{\Abs}[1]{\left\lvert{#1}\right\rvert} %absolute value

\newcommand{\nonnegint}{\D{N} \cup \left\{ 0 \right\}} %Non-negative integers

\newcommand{\Image}[1]{\operatorname{im}\left( {#1} \right)} %Image of a map

\newcommand{\Proj}[1]{\operatorname{proj}_{{#1}}}%projection map

\newcommand{\ev}[1]{\operatorname{ev}_{{#1}}}%Evaluation map

\makeatletter
\newbox\xrat@below
\newbox\xrat@above
\newcommand{\xrightarrowtail}[2][]{%
  \setbox\xrat@below=\hbox{\ensuremath{\scriptstyle #1}}%
  \setbox\xrat@above=\hbox{\ensuremath{\scriptstyle #2}}%
  \pgfmathsetlengthmacro{\xrat@len}{max(\wd\xrat@below,\wd\xrat@above)+.6em}%
  \mathrel{\tikz [>->,baseline=-.75ex]
                 \draw (0,0) -- node[below=-2pt] {\box\xrat@below}
                                node[above=-2pt] {\box\xrat@above}
                       (\xrat@len,0) ;}}
\makeatother %xrightarrowtail

\makeatletter
\newbox\xrat@below
\newbox\xrat@above
\newcommand{\xtwoheadrightarrow}[2][]{%
  \setbox\xrat@below=\hbox{\ensuremath{\scriptstyle #1}}%
  \setbox\xrat@above=\hbox{\ensuremath{\scriptstyle #2}}%
  \pgfmathsetlengthmacro{\xrat@len}{max(\wd\xrat@below,\wd\xrat@above)+.6em}%
  \mathrel{\tikz [->>,baseline=-.75ex]
                 \draw (0,0) -- node[below=-2pt] {\box\xrat@below}
                                node[above=-2pt] {\box\xrat@above}
                       (\xrat@len,0) ;}}
\makeatother %xtwoheadrightarrow

\newcommand{\dom}[1]{\operatorname{dom} \left( {#1} \right)} %Domain of a function

\newcommand{\codom}[1]{\operatorname{codom} \left( {#1} \right)} %Codomain of a function

\newcommand{\sseq}{\subseteq} %short hand for \subseteq
  
%ALGEBRA-------------------------------------------
\newcommand{\Aut}[2]{\operatorname{Aut}_{{#2}}\left({#1} \right)} %automorphism group

\newcommand{\Hom}[3]{\operatorname{Hom}_{{#3}}\left({#1},{#2}\right)} %Homomorphism group

\newcommand{\Char}[1]{\operatorname{char}\left( {#1} \right)} %characteristic of a ring/field

\newcommand{\Rad}[1]{\operatorname{Rad}\left({#1}\right)} %radical

\newcommand{\Nil}[1]{\operatorname{Nil}\left({#1}\right)} %nilradical

\newcommand{\Tors}[1]{\operatorname{Tors}\left({#1} \right)} %Torsion 

\newcommand{\Gal}[2]{\operatorname{Gal}\left({#1}/{#2}\right)} %Galois group

\newcommand{\pid}[1]{\left\langle {#1} \right\rangle} %Principal Ideal

\newcommand{\polynomial}[5]{\sum_{{#1}={#2}}^{{#3}} {#4}_{{#1}} {{#5}}^{{#1}}} %Polynomial

\newcommand{\mydeg}[1]{\deg \left( {#1} \right)} %Degree

\newcommand{\GL}[1]{GL \left({#1}\right)} %General linear group

\newcommand{\abelian}[1]{{#1}^{ab}} %Abelianisation

\newcommand{\Orb}[2]{\operatorname{Orb}_{{#1}}\left( {#2} \right)} %orbit of group action

\newcommand{\Stab}[2]{\operatorname{Stab}_{{#1}}\left( {#2} \right)} %stabiliser of group action

\newcommand{\GROUP}[2]{\left\langle {#1} \ \middle| \ {#2} \right\rangle} %group presentation

\newcommand{\ideal}{\trianglelefteq}%normal subgroup symbol

\newcommand{\properideal}{\triangleleft}%proper normal subgroup symbol

\newcommand{\groupaction}[3]{{#1} \rotatebox[origin=c]{180}{$\circlearrowright$}_{{#3}} {#2}}%group action

\newcommand{\SL}[1]{SL \left({#1}\right)} %special linear group

\newcommand{\iso}[1]{\operatorname{iso}\left( {#1} \right)}%Group of isomorphisms

\newcommand{\Gr}[1]{\operatorname{Gr}\left( {#1} \right)}%associated graded group

\newcommand{\bigboxplus}{
  \mathop{
    \vphantom{\bigoplus} 
    \mathchoice
      {\vcenter{\hbox{\resizebox{\widthof{$\displaystyle\bigoplus$}}{!}{$\boxplus$}}}}
      {\vcenter{\hbox{\resizebox{\widthof{$\bigoplus$}}{!}{$\boxplus$}}}}
      {\vcenter{\hbox{\resizebox{\widthof{$\scriptstyle\oplus$}}{!}{$\boxplus$}}}}
      {\vcenter{\hbox{\resizebox{\widthof{$\scriptscriptstyle\oplus$}}{!}{$\boxplus$}}}}
  }\displaylimits 
} %internal direct sum

\newcommand{\bideg}[1]{\text{bi-deg}\left( {#1} \right)}%bi-degree

\newcommand{\Stgroup}[1]{\operatorname{St}\left({#1} \right)} %Steinberg group

\newcommand{\groupp}[2]{\left\langle {#1} \ \middle| \ {#2} \right\rangle} %group presentation

%LINEAR ALGEBRA------------------------------------
\newcommand{\rank}[1]{\operatorname{{rank}}\left({#1}\right)} %rank

\newcommand{\diag}[1]{\operatorname{diag}\left({#1} \right)} %diagonal matrix

\newcommand{\Ker}[1]{\operatorname{ker}\left({#1} \right)} %Kernel of a map

\newcommand{\Tr}[1]{\operatorname{trace}\left({#1}\right)} %trace

\newcommand{\mat}[4]{\left[ \begin{array}{cc}
{#1} & {#2}  \\
{#3} & {#4}
\end{array} \right]} %2x2 matrix

\newcommand{\Range}[1]{\operatorname{Rng}\left( {#1} \right)} %Range of a map

\newcommand{\Span}[1]{\operatorname{span}\left( {#1} \right)} %span of a set

\newcommand{\Null}[1]{\operatorname{null}\left({#1}\right)} %nullity

\newcommand{\innerprod}[3]{\left\langle {#1}, {#2} \right\rangle_{{#3}}} %Inner product

\newcommand{\colvec}[2]{\left[ \begin{array}{c}
{#1} \\
\vdots \\
{#2}
\end{array} \right]}%column vector

\newcommand{\Dim}[2]{\dim_{{#2}} \left( {#1} \right)}%dimension

\newcommand{\Det}[1]{\det \left( {#1} \right)}%determinant

\newcommand{\coDim}[2]{\operatorname{codim}_{{#2}} \left( {#1} \right)}%codimension

%COMMUTATIVE ALGEBRA AND ALGEBRAIC GEOMETRY--------
\newcommand{\td}[1]{\operatorname{tr.deg.}\left( {#1} \right)} %transcendence degree

\newcommand{\Spec}[1]{\operatorname{Spec}\left({#1}\right)} %spectrum of a ring

\newcommand{\height}{\operatorname{height}} %height

\newcommand{\mSpec}[1]{\operatorname{mSpec}\left( {#1}\right)} %maximal ideals of a ring

\newcommand{\length}[1]{\operatorname{length}\left({#1} \right)} %length

\newcommand{\Pic}[1]{\operatorname{Pic}\left({#1}\right)}  %Piccard group

\newcommand{\Idem}[1]{\operatorname{Idem}\left( {#1} \right)}%Idempotent

\newcommand{\sections}[1]{\Gamma\left({#1} \right)}%sections of sheaf/bundle

\newcommand{\kerpre}[1]{\operatorname{ker}_{\operatorname{presheaf}}\left( {#1} \right)}%presheaf kernel

\newcommand{\cokerpre}[1]{\operatorname{coker}_{\operatorname{presheaf}}\left( {#1} \right)}%presheaf cokernel

\newcommand{\RES}[2]{\operatorname{res}_{{#1},{#2}}}%restriction of a presheaf

%Complex Analysis----------------------------------

\newcommand{\expi}[2]{e^{\frac{{#1}}{{#2}}}} %complex exponential with imaginary fraction exponent

\newcommand{\Res}[2]{\operatorname{Res} \left[ {#1},{#2} \right]} %residue

\newcommand{\myRe}[1]{\operatorname{Re}\left({#1} \right)} %Real part

\newcommand{\myIm}[1]{\operatorname{Im}\left({#1} \right)} %Imaginary part

\newcommand{\winding}[2]{\fun{\operatorname{Ind}_{{#1}}}{{#2}}}%winding number of a curve

\newcommand{\polydisc}[1]{\D{D}^{{#1}}_{\operatorname{poly}}}%poly-disc

\newcommand{\distbound}[1]{\partial^{\operatorname{dist}}\polydisc{{#1}}}%distinguished boundary

\newcommand{\Blaschke}[1]{\fun{\operatorname{Blaschke}}{{#1}}}%Blaschke product

%HOMOLOGICAL ALGEBRA AND CATEGORY THEORY-----------
\newcommand{\Mor}[2]{\operatorname{mor}_{{#2}}\left({#1}\right)} %Morphism class

\newcommand{\cat}[1]{\normalfont{\mathbf{#1}}} %notation for category

\newcommand{\Tor}[3]{\operatorname{Tor}_{{#3}}\left( {#1}, {#2} \right)} %Tor functor
\newcommand{\catset}{\operatorname{\scr{S}ets}} %category of sets

\newcommand{\Ring}{\operatorname{\scr{R}ings}} %category of rings

\newcommand{\Obj}[1]{\operatorname{obj}\left({#1} \right)} %Object class of category

\newcommand{\Ext}[2]{\operatorname{Ext}\left( {#1}, {#2} \right)} %Ext-functor

\newcommand{\coker}[1]{\operatorname{{coker}}\left( {#1} \right)} %cokernel

\makeatletter
\newcommand{\colim@}[2]{%
  \vtop{\m@th\ialign{##\cr
    \hfil$#1\operator@font colim$\hfil\cr
    \noalign{\nointerlineskip\kern1.5\ex@}#2\cr
    \noalign{\nointerlineskip\kern-\ex@}\cr}}%
}
\newcommand{\colim}[1]{%
  \mathop{\mathpalette\colim@{}}_{{#1}}
} %colimit

\renewcommand{\varprojlim}{%
  \mathop{\mathpalette\varlim@{\leftarrowfill@\scriptscriptstyle}}\nmlimits@
}
\renewcommand{\varinjlim}{%
  \mathop{\mathpalette\varlim@{\rightarrowfill@\scriptscriptstyle}}\nmlimits@
} %limit

\newcommand{\hocolim@}[2]{%
  \vtop{\m@th\ialign{##\cr
    \hfil$#1\operator@font hocolim$\hfil\cr
    \noalign{\nointerlineskip\kern1.5\ex@}#2\cr
    \noalign{\nointerlineskip\kern-\ex@}\cr}}%
}
\newcommand{\hocolim}[1]{%
  \mathop{\mathpalette\hocolim@{}}_{{#1}}
}%homotopy colimit

\newcommand{\holim@}[2]{%
  \vtop{\m@th\ialign{##\cr
    \hfil$#1\operator@font holim$\hfil\cr
    \noalign{\nointerlineskip\kern1.5\ex@}#2\cr
    \noalign{\nointerlineskip\kern-\ex@}\cr}}%
}
\newcommand{\holim}[1]{%
  \mathop{\mathpalette\holim@{}}_{{#1}}
}%homotopy limit

\newcommand{\tensor}[3]{{#1} \otimes_{{#3}} {#2}}%tensor product

\newcommand{\Eq}[1]{\operatorname{Eq} \left( {#1} \right)}%Equaliser

\newcommand{\coEq}[1]{\operatorname{coEq} \left( {#1} \right)}%Coequaliser

\newcommand{\catfgProj}[1]{\operatorname{\scr{P}roj}^{\operatorname{fg}}_{{#1}}} %category of finitely generated projective R-modules

\newcommand{\catmodule}[1]{\operatorname{\scr{M}odule}_{{#1}}} %category of R-modules

\newcommand{\catspace}{\operatorname{\scr{S}paces}} %category of spaces

\newcommand{\catspectra}{\operatorname{\scr{S}pectra}} %category of spectra

\newcommand{\catab}{\operatorname{\scr{A}belian}} %category of Abelian groups

\newcommand{\catringoid}{\operatorname{\scr{R}ingoids}} %category of ringoids

\newcommand{\catgroup}{\operatorname{\scr{G}roups}} %category of groups

\newcommand{\catfgfree}[1]{\operatorname{\scr{F}ree}^{\operatorname{fg}}_{{#1}}} %category of finitely generated free R-modules

\newcommand{\hormor}[1]{\operatorname{hor-mor}\left({#1}\right)} % horizontal morphism class

\newcommand{\vermor}[1]{\operatorname{ver-mor}\left({#1}\right)} % vertical morphism class

\newcommand{\bimor}[1]{\operatorname{bimor}\left({#1}\right)} % bi-morphism class

\newcommand{\catiso}[1]{\operatorname{iso}\left({#1}\right)}%category of isomorphisms

\newcommand{\SiS}[1]{\scr{S}_{{#1}}} %S-inverse-S-construction of category of isomorphisms

\newcommand{\CMA}[2]{\scr{C}_{{#2}}\left( {#1} \right)} %Pedersen-Weibel category

\newcommand{\catfinset}{\operatorname{\scr{F}in\scr{S}et}} %category of finite set

\newcommand{\catSMC}{\operatorname{\scr{S}ym\scr{M}on\scr{C}at}} %category of symmetric monoidal categories

\newcommand{\PiP}[1]{{\scr{P}_{{#1}}}} %S-inverse-S-construction of category of isomorphisms in the idempotent completion

\newcommand{\catoofree}[1]{\operatorname{\scr{F}ree}^{\mathbb{N}}_{{#1}}} %category of countably generated free R-modules

\newcommand{\catomegaspectra}{\operatorname{\Omega-\scr{S}pectra}} %category of omega-spectra

%TOPOLOGY------------------------------------------
\newcommand{\point}{\operatorname{point}} %point 

\newcommand{\Closure}[2]{\operatorname{Closure}_{{#1}}\left({#2} \right)} %Closure

\newcommand{\Int}[1]{\operatorname{Int}\left({#1} \right)} %Set of interior points

\newcommand{\Bd}[1]{\partial {#1}} %Boundary of a set

\newcommand{\sphere}[1]{\D{S}^{{#1}}} %sphere

\newcommand{\CP}[1]{\D{C}\D{P}^{{#1}}} %complex projective spaces

\newcommand{\RP}[1]{\D{R}\D{P}^{{#1}}} %real projective spaces

\newcommand{\sk}[2]{{#1}^{({#2})}} %n-skeleton of a CW complex

\newcommand{\simplex}[1]{\left[ {#1} \right]} %Simplex

\newcommand{\commutativesquare}[8]{\begin{tikzpicture}
  \node (A) {{#1}}; 
  \node (B) [right=of A] {{#3}}; 
  \node (C) [below=of A] {{#4}}; 
  \node (D) [right=of C, below=of B] {{#6}};
  \draw[->] (A)-- node[above] {\tiny {#2}} (B); 
  \draw[->] (A)-- node [left] {\tiny {#7}} (C); 
  \draw[->] (B)-- node [right] {\tiny {#8}} (D); 
  \draw[->] (C)-- node [below] {\tiny {#5}} (D); 
\end{tikzpicture}}%Commutative square

\newcommand{\gtori}[1]{\left( \D{T}^2 \right)^{\vee {#1}}}%wedge sum of g tori

\newcommand{\cupprod}[2]{{#1}\smile {#2}}%cup product

\newcommand{\capprod}[2]{{#1} \frown  {#2}}%cap product

\newcommand{\Map}[3]{\operatorname{Map}_{{#3}} \left( {#1}, {#2} \right)}%Mapping space

\newcommand{\Loop}[1]{\Omega {#1}}%loop space

\newcommand{\Suspen}[1]{\Sigma {#1}}%suspension over a space

\newcommand{\Face}[2]{d_{{#1}}^{{#2}}}%face map of simplicial space

\newcommand{\Degen}[2]{s_{{#1}}^{{#2}}}%degeneracy map of simplicial space

\newcommand{\gsimplex}[1]{\Abs{\Delta^{#1}}}%geomtric n-simplex

\newcommand{\myprod}[3]{{#1} \times_{{#3}}{{#2}}}%fibre product

\newcommand{\Fr}[1]{\operatorname{Fr}\left({#1}\right)}%Frame bundle

\newcommand{\Grass}[3]{\operatorname{Gr}_{{#2}}\left( \D{{#3}}^{{#1}}\right)}%Grassmannian

\newcommand{\Stiefelm}[3]{V_{{#2}}\left( \D{{#3}}^{{#1}}\right)}%Stiefel Manifold

\newcommand{\oStiefelm}[3]{V^o_{{#2}}\left( \D{{#3}}^{{#1}}\right)}%orthonormal Stiefel Manifold

\newcommand{\homotopygrp}[2]{\pi_{{#1}} \left({#2} \right)}%homotopy group

\newcommand{\homotopymap}[2]{\pi \left[ {#1}, {#2} \right]}%homotopy classes of maps

\newcommand{\cohomology}[4]{H^{{#3}}_{\operatorname{{#4}}}\left({#1}\mbox{;} \ {#2} \right)}%cohomology group

\newcommand{\deRham}[2]{H^{{#1}}_{\operatorname{dR}}\left({#2} \right)}%de Rham cohomology group

\newcommand{\Zcohomology}[2]{H^{{#2}}\left({#1}\mbox{;} \ {\D{Z}} \right)}%Integral cohomology group

\newcommand{\EG}[2]{E_{{#2}}{#1}}%universal space

\newcommand{\mydu}[3]{{#1} \sqcup_{{#3}}{{#2}}}%pushout

\newcommand{\normclosure}[1]{\ol{{#1}}^{\norm{\cdot}{}}} %norm closure

\newcommand{\weakclosure}[1]{\ol{{#1}}^{w}} %weak closure

\newcommand{\cone}[1]{\operatorname{cone}\left( {#1} \right)} %cone space

\newcommand{\cylinder}[1]{\operatorname{cyl} \left( {#1} \right)} %cylinder

\newcommand{\cwreplace}[1]{{#1}_{\operatorname{CW}}}%CW-replacement

\newcommand{\hofib}[1]{\operatorname{hofib}\left( {#1} \right)} %homotopy fibre

\newcommand{\hocofib}[1]{\operatorname{hocofib}\left( {#1} \right)} %homotopy cofibre

\newcommand{\CG}[2]{\D{CG}\left( {#1}, {#2} \right)} %complex Grassmannian

\newcommand{\ssphere}[1]{\check{\D{S}}^{{#1}}} %simplicial sphere

\newcommand{\AHSS}[3]{\operatorname{AHSS}\left( {#1} \right)^{{#2}}_{{#3}}} %sophisticated Atiyah-Hirzebruch

%CALCULUS AND ANALYSIS-----------------------------
\newcommand{\norm}[2]{\left\lVert{#1}\right\rVert_{#2}} %norm of a vector

\newcommand{\Dif}[2]{\frac{d{#1}}{d{#2}}} %derivative

\newcommand{\dif}[2]{\frac{\partial {#1}}{\partial {#2}}} %partial derivative

\newcommand{\Interval}[4]{ \left#1 {#2}, {#3} \right#4} %interval

\newcommand{\grad}[1]{\operatorname{grad}\left({#1}\right)} %gradient

\newcommand{\oball}[2]{B \left( {#1}, {#2} \right)}%open ball

\newcommand{\cball}[2]{\ol{B} \left( {#1}, {#2} \right)}%closed ball

\newcommand{\Lp}[2]{L^{{#1}} \left( {#2} \right)}%Lp space

\newcommand{\lp}[2]{\ell^{{#1}} \left( {#2} \right)}%Lp space

\newcommand{\orcom}[1]{{#1}^{\perp}}%orthogonal complement

\newcommand{\myint}[4]{\int_{{#3}}^{{#4}} {#1} \ d{{#2}}}%integration

\newcommand{\normop}[1]{\norm{{#1}}{op}}%operator norm

\newcommand{\normHS}[1]{\norm{{#1}}{\operatorname{HS}}}%Hilbert-Schmidt norm

\newcommand{\supp}[1]{\operatorname{supp}\left( {#1} \right)}%support of function

\newcommand{\Fred}[1]{\operatorname{Fred}\left( {#1} \right)}%Fredholm operators

\newcommand{\ind}[1]{\operatorname{ind}\left( {#1} \right)}%classical index

\newcommand{\Calk}[1]{\operatorname{Calk}\left( {#1} \right)}%Calking algebra

%LIE THEORY----------------------------------------
\newcommand{\Lie}[1]{\mathfrak{{#1}}} %Lie algebra

\newcommand{\commutator}[2]{\left[ {#1}, {#2} \right]}%commutator

%DIFFERENTIAL GEOMETRY----------------------------
\newcommand{\christof}[3]{\Gamma_{{#1} \hspace{0.1em} {#3}}^{\hspace{0.3em {#2}}}} %Christoffel symbol

%NUMBER THEORY------------------------------------
\newcommand{\MOD}[3]{{#1} \equiv {#2} \ \left(\operatorname{mod} \  {#3} \right)}

\newcommand{\zmodp}[1]{\D{Z}/{#1}\D{Z}} %Modulo p integers

\newcommand{\sign}[1]{\operatorname{sign}\left( {#1}\right)}%sign function

\newcommand{\mygcd}[1]{\gcd\left( {#1} \right)} % GCD

%PHYSICS----------------------------------
\newcommand{\quantumev}[1]{\left\langle {#1} \right\rangle}%quantum expected value

%RESEARCH PAPER-----------------------------------
\newcommand{\TR}[2]{\operatorname{TR}^{#1}_{#2}} %equivariant homotopy group

\newcommand{\borelH}[1]{\operatorname{H}^{{\tiny \operatorname{Borel}}}_{#1}} %Borel homology

\newcommand{\simplexcat}[1]{\Delta \downarrow {#1}} %simplex category for a simplicial set X

\newcommand{\GJreal}[1]{\Abs{{#1}}_{\operatorname{GJ}}} %Goerss-Jardine realisation for a simplicial set

\newcommand{\externalprod}[2]{{#1} \widetilde{\times} {#2}} %external product of two bi-simplicial sets

\newcommand{\fullreal}[1]{{\Abs{#1}}_{\operatorname{full}}} %full realisation of a bi-simplicial set

\newcommand{\diagreal}[1]{\Abs{{#1}}_{\operatorname{diag}}} %diagonal realisation of bi-simplicial set

\newcommand{\cofib}[2]{{#1} \rightarrowtail {#2}} %cofibration

\newcommand{\simp}[1]{\operatorname{simp}\left({#1} \right)}%Waldhausen's simp functor

\newcommand{\cofseq}[3]{{#1} \rightarrowtail {#2} \twoheadrightarrow {#3}}%Cofibration sequence

\newcommand{\THH}[2]{\operatorname{THH}\left( {#1}\right)_{{#2}}} %Topological Hochschild Homology

\newcommand{\assem}[1]{\alpha_{{#1}}}%assembly map

\newcommand{\Wh}[2]{\operatorname{Wh}_{{#2}}\left( {#1} \right)}%Whitehead group

\newcommand{\Zariskicohomology}[3]{H^{{#3}}_{\textrm{\tiny Zariski}}\left({#1} \mbox{;} \ {#2} \right)}%Zariski cohomology group

\newcommand{\etalecohomology}[3]{H^{{#3}}_{\textrm{\tiny \'{e}t}}\left({#1} \mbox{;} \ {#2} \right)}%etale cohomology group

\newcommand{\trivialcofib}{\mycal{C} \cap \mycal{W}} %trivial cofibration

\newcommand{\trivialfib}{\mycal{F} \cap \mycal{W}} %trivial fibration

\newcommand{\RamPM}[1]{\dul{P}\left( {#1} \right)}%Ramras' Category of Projective Modules

\newcommand{\Qcon}[1]{Q \left({#1} \right)} %Q-construction

\newcommand{\admmor}[5]{\begin{tikzpicture}
  \node (A) {${#1}$}; 
  \node (B) [right= of A] {${#2}$};
  \node (C) [right= of B] {${#3}$};
  \draw[->>] (B)--node[above] {\small ${#4}$} (A);
  \draw[>->] (B)--node[above] {\small ${#5}$} (C);
\end{tikzpicture}} %morphisms in Q-construction

\newcommand{\Lodayf}[4]{f^{{#1},{#2}}_{{#3},{#4}}}%Loday's f map

\newcommand{\BGL}[1]{BGL \left({#1}\right)} %Classifying space of GL

\newcommand{\BGLp}[1]{\fun{BGL}{{#1}}^{+}} %Plus construction

\newcommand{\Lodaym}[4]{\gamma^{{#1},{#2}}_{{#3},{#4}}}%Loday's multiplication map

\newcommand{\Lodaymh}[4]{\widehat{\gamma}^{{#1},{#2}}_{{#3},{#4}}}%Loday's multiplication map on smash product

\newcommand{\KDL}[1]{K^{\operatorname{DL}}\left( {#1} \right)}%Davis-Luck K-theory spectrum

\newcommand{\Orcat}[1]{\operatorname{Or}\left( {#1} \right)}%orbit category

\newcommand{\actgroupoid}[2]{{#1}\mathsmaller{\int} {#2}}%action groupoid, need \usepackage{relsize}

\newcommand{\twist}[2]{\operatorname{twist}_{{#1},{#2}}} %twist map

\newcommand{\Lodayproda}[2]{ {#1} \ast_{\operatorname{Loday}}{#2}} %Loday product \ast

\newcommand{\Lodayprodb}[2]{ {#1}  \bigstar  {#2}} %Loday product \ast

\newcommand{\KGW}[1]{\mathbb{K}^{\operatorname{GW}}_{{#1}}}%Gersten-Wagoner K-theory spectrum

\newcommand{\Kfree}[1]{\mathbb{K}^{\operatorname{free}}_{{#1}}}%Free K-theory spectrum

\newcommand{\HH}[2]{\operatorname{{\it HH}}_{{#1}} \left( {#2} \right)} %Hochschild homology 

\newcommand{\Stsym}[2]{ \left\{ {#1}, {#2} \right\}_{\mathrm{St}}}%Steinberg Symbol

\newcommand{\Ncyc}[2]{N^{\operatorname{cyc}}_{{#2}} \left( {#1} \right)} %cyclic bar construction

\newcommand{\Lodaya}{\alpha_{\operatorname{\tiny Loday}}} %Loday assembly

\newcommand{\Walda}{\alpha_{\operatorname{\tiny Wald}}} %Waldhausen assembly

\newcommand{\Lodayp}{\gamma_{\operatorname{\tiny Loday}}} %Loday pairing

\newcommand{\Waldp}{\gamma_{\operatorname{\tiny Wald}}} %Waldhausen pairing

\newcommand{\Weibelp}{\gamma_{\operatorname{\tiny Weibel}}} %Weibel pairing

\newcommand{\freep}{\gamma_{\operatorname{\tiny free}}} %pairing for free modules

\newcommand{\WWa}{\alpha_{\operatorname{\tiny WW}}} %Weiss-Williams assembly

\newcommand{\KQ}[1]{\mathbb{K}^{Q}_{{#1}}}% K-theory spectrum in terms of Q-construction (do not confuse with the double Q-construction)

\newcommand{\kgw}[1]{\Bbbk^{\operatorname{gw}}_{{#1}}}%Gersten-Wagoner K-theory spectrum without the K0-factor

\newcommand{\WhG}[2]{\operatorname{Wh}_{{#2}} \left( {#1} \right)}

\newcommand{\KKfree}[1]{K^{\operatorname{free}}_{{#1}}}%Free K-theory space

\newcommand{\KPW}[1]{\mathbb{K}^{\operatorname{PW}}_{{#1}}}%Pedersen-Weibel K-theory spectrum

\newcommand{\Ksmc}[1]{K^{\Box}_{{#1}}}%K-theory space of a symmetric monoidal category

\newcommand{\freea}{\alpha_{\operatorname{\tiny free}}} %free assembly

\newcommand{\Kproj}[1]{\mathbb{K}^{\operatorname{proj}}_{{#1}}}%Idempotent K-theory spectrum

\newcommand{\projp}{\gamma_{\operatorname{\tiny proj}}} %pairing for projective modules

\newcommand{\proja}{\alpha_{\operatorname{\tiny proj}}} %projective assembly

\newcommand{\freestar}{\star_{\operatorname{free}}} %the multiplication map with respect to \freep

\newcommand{\Kahlerdiff}[3]{\Omega^{{#1}}_{\left. {#2} \middle| {#3} \right.}} %Kahler differentials

\newcommand{\naivep}{\gamma_{\operatorname{\tiny naive}}} %naive pairing

\newcommand{\naivestar}{\star_{\operatorname{naive}}} %the multiplication map with respect to \naivep

\newcommand{\naivea}{\alpha_{\operatorname{\tiny naive}}} %naive assembly

%STATISTICS-------------------------------
\newcommand{\Var}[1]{\operatorname{Var}\left( {#1} \right)} % variance

\newcommand{\Cov}[1]{\operatorname{Cov}\left( {#1} \right)} % covariance

\newcommand{\binomdist}[1]{\operatorname{Binomial}\left( {#1}\right)} % Binomial distribution

\newcommand{\negbinomdist}[1]{\operatorname{NegBinomial}\left( {#1} \right)} % Negative Binomial distribution

\newcommand{\normaldist}[1]{\operatorname{Normal}\left( {#1} \right)} % Normal distribution

\newcommand{\poissondist}[1]{\operatorname{Poisson}\left( {#1} \right)} % Poisson distribution

\newcommand{\uniformdist}[1]{\operatorname{Uniform}\left( {#1} \right)} % Poisson distribution

\newcommand{\geometricdist}[1]{\operatorname{Geometric}\left( {#1} \right)} % Geometric distribution

\newcommand{\conditbar}[2]{ \left. {#1} \middle| {#2} \right.} % conditional bar

\newcommand{\gammadist}[1]{\operatorname{Gamma}\left( {#1} \right)} % Gamma distribution

\newcommand{\betadist}[1]{\operatorname{Beta}\left( {#1} \right)} % Beta distribution

\newcommand{\bernoullidist}[1]{\operatorname{Bernoulli}\left( {#1} \right)} % Bernoulli distribution

\newcommand{\Gaussianpdf}[2]{ \frac{1}{\sqrt{2\pi} {#2}} e^{-\frac{1}{2} \left( {#1} \right)^2} } 

\newcommand{\orderstatvar}[2]{{#1}_{\left( {#2} \right)}} % short-hand notation for ordered statistics

%\END{COMMAND}

\makeatletter
\tikzset{join/.code=\tikzset{after node path={%
\ifx\tikzchainprevious\pgfutil@empty\else(\tikzchainprevious)%
edge[every join]#1(\tikzchaincurrent)\fi}}}

\makeatother

%\tikzset{>=stealth',every on chain/.append style={join},
%        every join/.style={->}}

\newlength{\parindentsave}\setlength{\parindentsave}{\parindent}

\everymath{\displaystyle}

\numberwithin{equation}{subsection} 

\let\emptyset\varnothing

\hypersetup{colorlinks,citecolor=blue,linkcolor=blue}

\declaretheorem[numberwithin=section, shaded={rulecolor=black,
rulewidth=0.5pt, bgcolor={rgb}{1,1,1}}]{Theorem}

%\doublespacing

\setcounter{tocdepth}{4}

\begin{document}
\maketitle

\tableofcontents

\newpage
\section{Problem 2.1}

In each of the following find the pdf of $Y$. Show that the pdf integrates to 1.

\begin{enumerate}[label=(\alph*),leftmargin=*]
    \item $Y = X^3$ and $f_X(x) = 42x^5(1-x)$, $0<x<1$
    \item $Y = 4X+3$ and $f_X(x) = 7e^{-7x}$, $0< x < \infty$
    \item $Y = X^2$ and $f_X(x) = 30x^2(1-x)^2$, $0< x< 1$
\end{enumerate}~\\

\begin{proof}[Solution] We begin by noting all conditions of \cite[Theorem 2.1.5 on page 51]{Berger-Casella} are satisfied in each case. We leave it to the reader to verify the pdf integrates to 1.

\begin{enumerate}[label=(\alph*),leftmargin=*]
    \item Let $g(x) = x^3$ for $x \in (0,1)$, then $g^{-1}(y) = y^{\frac{1}{3}}$ for $y \in (0,1)$, and
    
    \begin{align*}
        \Abs{\Dif{}{y} g^{-1}(y)} = \frac{1}{3 y^{\frac{2}{3}}}.
    \end{align*}
    Hence,

    \begin{align*}
        \fun{f_Y}{y} &= f_X \left( g^{-1} \left( y \right) \right) \cdot \Abs{\Dif{}{y} g^{-1}(y)} \\
        &=  \left[ 42y^{\frac{5}{3}} \left( 1 -y^{\frac{1}{3}} \right) \right] \cdot \frac{1}{3 y^{\frac{2}{3}}} \\
        &= 14  \left( y - y^{\frac{4}{3}} \right)
    \end{align*}
    on $\mycal{Y} = (0,1)$.
    
    \item Let $g(x) = 4x + 3$ for $x \in (0, \infty)$, then $g^{-1}(y) = \frac{y-3}{4}$ for $y \in (3, \infty)$, and
    
    \begin{align*}
        \Abs{\Dif{}{y} g^{-1}(y)} = \frac{1}{4}.
    \end{align*}
    Hence,
    
    \begin{align*}
        \fun{f_Y}{y} &= f_X \left( g^{-1} \left( y \right) \right) \cdot \Abs{\Dif{}{y} g^{-1}(y)} \\
        &=  \frac{7}{4} e^{-\frac{-7(y-3)}{4}}
    \end{align*}
    on $\mycal{Y} = (3, \infty)$.
    
    \item Let $g(x) = x^2$ for $x \in (0,1)$, then $g^{-1}(y) = y^{\frac{1}{2}}$ for $y \in (0,1)$, and
    
    \begin{align*}
        \Abs{\Dif{}{y} g^{-1}(y)} = \frac{1}{2} y^{-\frac{1}{2}}.
    \end{align*}
    Hence,
    
    \begin{align*}
        \fun{f_Y}{y} &= f_X \left( g^{-1} \left( y \right) \right) \cdot \Abs{\Dif{}{y} g^{-1}(y)} \\
        &= 15y^{\frac{1}{2}} \left(1-y^{\frac{1}{2}} \right)^2
    \end{align*}
\end{enumerate}
\end{proof}

\newpage
\section{Problem 2.2}

In each of the following find the pdf of $Y$.

\begin{enumerate}[label=(\alph*),leftmargin=*]
    \item $Y = X^2$ and $f_X(x) = 1$, $0 < x < 1$
    \item $Y = -\log(X)$ and $X$ has pdf
    
    \[ \mbox{$f_X(x) = \frac{(n+m+1)!}{n!m!}x^n (1-x)^m$, $0<x<1$, $m$, $n$ positive integers} \]
    
    \item $Y = e^X$ and $X$ has pdf
    
    \[ \mbox{$f_X(x) = \frac{1}{\sigma^2} xe^{-\frac{(x/\sigma)^2}{2}}$, $0< x< \infty$, $\sigma^2$ a positive constant} \]
\end{enumerate}~\\

\begin{proof}[Solution]
We begin by noting all conditions of \cite[Theorem 2.1.5 on page 51]{Berger-Casella} are satisfied in each case.

\begin{enumerate}[label=(\alph*),leftmargin=*]
    \item Let $g(x) = x^2$ for $x \in (0, 1)$, then $g^{-1}(y) = y^{\frac{1}{2}}$ for $y \in (0, 1)$, and
    
    \begin{align*}
        \Abs{\Dif{}{y} g^{-1}(y)} = \frac{1}{2} y^{-\frac{1}{2}}.
    \end{align*}
    Hence,
    
    \begin{align*}
        \fun{f_Y}{y} &= f_X \left( g^{-1} \left( y \right) \right) \cdot \Abs{\Dif{}{y} g^{-1}(y)} \\
        &=  \frac{1}{2} y^{-\frac{1}{2}}
    \end{align*}
    on $\mycal{Y} = (0, 1)$.
    
    \item Let $g(x) = -\log(x)$ for $x \in (0, 1)$, then $g^{-1}(y) = e^{-y}$ for $y \in (0, \infty)$, and
    
    \begin{align*}
        \Abs{\Dif{}{y} g^{-1}(y)} = e^{-y}.
    \end{align*}
    Hence,
    
    \begin{align*}
        \fun{f_Y}{y} &= f_X \left( g^{-1} \left( y \right) \right) \cdot \Abs{\Dif{}{y} g^{-1}(y)} \\
        &=  \frac{(n+m+1)!}{n!m!}e^{-ny} (1-e^{-y})^m
    \end{align*}
    on $\mycal{Y} = (0, \infty)$.
    
    \item Let $g(x) = e^x$ for $x \in (0, \infty)$, then $g^{-1}(y) = \log(y)$ for $y \in (1, \infty)$, and
    
    \begin{align*}
        \Abs{\Dif{}{y} g^{-1}(y)} = \frac{1}{y}.
    \end{align*}
    Hence,
    
    \begin{align*}
        \fun{f_Y}{y} &= f_X \left( g^{-1} \left( y \right) \right) \cdot \Abs{\Dif{}{y} g^{-1}(y)} \\
        &=  \frac{\log(y)}{y\sigma^2}e^{-\frac{(\log(y)/\sigma)^2}{2}}
    \end{align*}
    on $\mycal{Y} = (0, \infty)$.
\end{enumerate}
\end{proof}

\newpage
\section{Problem 2.3}

Suppose $X$ has the geometric pmf $f_X(x) = \frac{1}{3} \left( \frac{2}{3} \right)^x$, $x = 0, 1, 2, \cdots$. Determine the probability distribution of $Y = X/(X+1)$. Note that here both $X$ and $Y$ are discrete random variables. To specify the probability distribution of $Y$, specify its pmf.

\begin{proof}[Solution]

\begin{align*}
    \fun{f_Y}{y} &= \fun{P}{Y = y} \\
                 &= \fun{P}{\frac{X}{X+1} = y} \\
                 &= \fun{P}{X = \frac{y}{1-y}} \\
                 &= \fun{f_X}{\frac{y}{1-y}} \\
                 &= \frac{1}{3} \left( \frac{2}{3} \right)^{\frac{y}{1-y}}
\end{align*}
on $\mycal{Y} = \SET{\frac{x}{x+1}}{\mbox{$x = \frac{1}{3} \left( \frac{2}{3} \right)^k$ for some $k \in \D{N} \cup \SETT{0}$}}$
\end{proof}

\newpage
\section{Problem 2.4}

Let $\lambda$ be a fixed positive constant, and define the function $f(x)$ by $f(x) = \frac{1}{2} \lambda e^{-\lambda x}$ if $x \geq 0$ and $f(x) = \frac{1}{2} \lambda e^{\lambda x}$ if $x < 0$.

\begin{enumerate}[label=(\alph*),leftmargin=*]
    \item Verify that $f(x)$ is a pdf.
    \item If $X$ is a random variable with pdf given by $f(x)$, find $\fun{P}{X < t}$ for all $t$. Evaluate all integrals.
    \item Find $\fun{P}{\Abs{X} < t}$ for all $t$. Evaluate all integrals.
\end{enumerate}~\\

\begin{proof}[Solution]~\\

\begin{enumerate}[label=(\alph*),leftmargin=*]
    \item Check the conditions listed on \cite[Theorem 1.6.5 on page 36]{Berger-Casella} for $f(x)$.
    
    \item
    
    \begin{align*}
        \fun{P}{X < t} &= \myint{f(x)}{x}{-\infty}{t} \\
                       &= \left\{ \begin{array}{cl}
                            \myint{\frac{1}{2} \lambda e^{\lambda x}}{x}{-\infty}{t} & \mbox{if $t < 0$}, \\
                            \myint{\frac{1}{2} \lambda e^{\lambda x}}{x}{-\infty}{0} + \myint{\frac{1}{2} \lambda e^{-\lambda x}}{x}{0}{t} & \mbox{if else},
                       \end{array} \right. \\
                       &=  \left\{ \begin{array}{cl}
                            \frac{e^{\lambda t}}{2} & \mbox{if $t < 0$}, \\
                            1 - \frac{1}{2}e^{-\lambda t} & \mbox{if else}.
                       \end{array} \right.
    \end{align*}
    
    \item
    
    \begin{align*}
        \fun{P}{\Abs{X} < t} &= \fun{P}{-t < X < t} \\
                             &= \fun{P}{X < t} - \fun{P}{X < -t} \\
                             &= \left( 1 - \frac{1}{2}e^{-\lambda t} \right) -  \frac{e^{-\lambda t}}{2} & \left(\mbox{part (b)}\right) \\
                             &= 1 - e^{-\lambda t}
    \end{align*}
\end{enumerate}
\end{proof}

\newpage
\section{Problem 2.5}

Use \cite[Theorem 2.1.8 on page 53]{Berger-Casella} to find the pdf of $Y$ in \cite[Example 2.1.2 on page 49]{Berger-Casella}. Show that the same answer is obtained by differentiating the cdf given in \cite[Equation 2.1.6 on page 49]{Berger-Casella}.

\begin{proof}[Solution]
Partition the interval $(0, 2\pi)$ into $\SETT{A_i}_{i=0}^4$, with

\begin{align*}
    A_i &= \left\{ \begin{array}{cl}
         \SETT{0} & \mbox{if $i = 0$}, \\
         \Interval{(}{\frac{(i-1)\pi}{2}}{\frac{i \pi }{2}}{)} & \mbox{if $i > 0$}.
    \end{array} \right.
\end{align*}
For each $i$, write $g_i(x) = \sin^2(x)$ on $A_i$. Then

\begin{align*}
    \fun{g^{-1}_1}{y} &= \arcsin(\sqrt{y}) \\
    \fun{g^{-1}_2}{y} &= \pi - \arcsin(\sqrt{y}) \\
    \fun{g^{-1}_3}{y} &= \pi + \arcsin(\sqrt{y}) \\
    \fun{g^{-1}_4}{y} &= 2\pi - \arcsin(\sqrt{y})
\end{align*}
Therefore,

\begin{align*}
    \fun{f_Y}{y} &= \sum_{i=1}^4 f_X \left( g_i^{-1} \left( y \right) \right) \cdot \Abs{\Dif{}{y} g_i^{-1}(y)} \\
    &= 4 \left(\frac{1}{2\pi} \right) \left[ \frac{1}{2 \sqrt{y-y^2}} \right] \\
    &= \frac{1}{\pi \sqrt{y-y^2}}
\end{align*}
on $\mycal{Y} = (0,1)$.
\end{proof}

\newpage
\section{Problem 2.6}
In each of the following find the pdf of $Y$ and show that the pdf integrates to 1.

\begin{enumerate}[label=(\alph*),leftmargin=*]
    \item $f_X(x) = \frac{1}{2} e^{-\Abs{x}}$, $-\infty < x < \infty$; $Y = \Abs{X}^3$
    \item $f_X(x) = \frac{3}{8} \left( x + 1 \right)^2$, $-1 < x < 1$; $Y = 1 - X^2$
    \item $f_X(x) = \frac{3}{8} \left( x + 1 \right)^2$, $-1 < x < 1$; $Y = 1 - X^2$ if $X \leq 0$ and $Y = 1-X$ if $X > 0$
\end{enumerate}

\begin{proof}[Solution] We note that \cite[Theorem 2.1.8 on page 53]{Berger-Casella} applies to all cases, and let readers to verify the pdf integrates to 1.

\begin{enumerate}[label=(\alph*),leftmargin=*]
    \item Parition $\Interval{(}{-\infty}{\infty}{)}$ into
    
    \begin{align*}
        A_0 &= \SETT{0} \\
        A_1 &= \Interval{(}{-\infty}{0}{)} \\
        A_2 &= \Interval{(}{0}{\infty}{)}
    \end{align*}
    and define
    
    \[ g_i(x) = \left\{ \begin{array}{cl}
         x^3 & \mbox{if $i$ even,}\\
         -x^3 & \mbox{if $i$ odd}
    \end{array} \right. \]
    on $A_i$. Then
    
    \begin{align*}
        \fun{f_Y}{y} &= \sum_{i=1}^2 f_X \left( g_i^{-1} \left( y \right) \right) \cdot \Abs{\Dif{}{y} g_i^{-1}(y)} \\
        &= \frac{1}{3} y^{-\frac{2}{3}} e^{-y^{1/3}}
    \end{align*}
    on $\mycal{Y} = (0, \infty)$.
    
    \item Partition $(-1, 1)$ into     
    \begin{align*}
        A_0 &= \SETT{0} \\
        A_1 &= \Interval{(}{-1}{0}{)} \\
        A_2 &= \Interval{(}{0}{1}{)}
    \end{align*}
    and define
    
    \[ g_i(x) = 1-x^2 \]
    on $A_i$. Then
    
        \begin{align*}
        \fun{f_Y}{y} &= \sum_{i=1}^2 f_X \left( g_i^{-1} \left( y \right) \right) \cdot \Abs{\Dif{}{y} g_i^{-1}(y)} \\
        &= \frac{3}{8} \left( \frac{1}{ \sqrt{1-y}} + \sqrt{1-y} \right)
    \end{align*}
    on $\mycal{Y} = (0, 1)$.
    
    \item Partition $(-1,1)$ just as in part (b), and define
    
    \[ g_i(x) = \left\{ \begin{array}{cl}
        1-x^2 & \mbox{on $A_1$,} \\
        1-x & \mbox{on $A_2$.}
    \end{array} \right. \]
    Then 
    \begin{align*}
        \fun{f_Y}{y} &= \sum_{i=1}^2 f_X \left( g_i^{-1} \left( y \right) \right) \cdot \Abs{\Dif{}{y} g_i^{-1}(y)} \\
        &= \frac{3}{16} \frac{1}{\sqrt{1-y}} \left( 1 - \sqrt{1-y} \right)^2 +\frac{3}{8} (2-y)^2
    \end{align*}
    on $\mycal{Y} = (0, 1)$.
\end{enumerate}
\end{proof}

\newpage
\section{Problem 2.7}

Let $X$ have pdf $f_X(x) = \frac{2}{9}(x+1)$, $-1 \leq x \leq 2$.

\begin{enumerate}[label=(\alph*),leftmargin=*]
    \item Find the pdf of $Y = X^2$. Note that \cite[Theorem 2.1.8 on page 53]{Berger-Casella} is not directly applicable in this problem.
    \item Show that \cite[Theorem 2.1.8 on page 53]{Berger-Casella} remains valid if the sets $A_0, A_1, \cdots, A_k$ contain $\mycal{X}$, and apply the extension to solve part (a) using $A_0 = \emptyset$, $A_1 = (-2,0)$, and $A_2 = (0,2)$.
\end{enumerate}~\\

\begin{proof}[Solution]~\\
\begin{enumerate}[label=(\alph*),leftmargin=*]
    \item 
    
    \begin{align*}
        \fun{P}{Y \leq y} &= \fun{P}{X^2 \leq y} \\
                          &= \left\{ \begin{array}{cl}
                               \fun{P}{-\sqrt{y} \leq X \leq \sqrt{y}} & \mbox{if $y < 1$,} \\
                               \fun{P}{-1 \leq X \leq \sqrt{y}} & \mbox{if $1 \leq y \leq 4$.}
                          \end{array} \right. \\
                          &= \left\{ \begin{array}{cl}
                               \myint{f_X(x)}{x}{-\sqrt{y}}{\sqrt{y}} & \mbox{if $y < 1$,} \\
                               \myint{f_X(x)}{x}{-1}{\sqrt{y}} & \mbox{if $1 \leq y \leq 4$.}
                          \end{array} \right. \\
                         &= \left\{ \begin{array}{cl}
                               \frac{4 \sqrt{y}}{9} & \mbox{if $y < 1$,} \\
                               \frac{1}{9} \left( 1 + \sqrt{y} \right)^2 & \mbox{if $1 \leq y \leq 4$.}
                          \end{array} \right. \\
    \end{align*}
    on $\mycal{Y} = (0,4)$.
    
    \item C.f. Problem 2.6.
\end{enumerate}
\end{proof}

\newpage
\section{Problem 2.8}

In each of the following show that the given function is a cdf and find $F^{-1}_X(y)$.

\begin{enumerate}[label=(\alph*),leftmargin=*]
    \item
    
    \[ F_X(x) = \left\{ \begin{array}{cl}
         0 & \mbox{if $x < 0$} \\
         1 - e^{-x} & \mbox{if $x \geq 0$}
    \end{array} \right. \]
    
        \item
    
    \[ F_X(x) = \left\{ \begin{array}{cl}
         e^x/2 & \mbox{if $x < 0$} \\
         1/2 & \mbox{if $0 \leq x < 1$} \\
         1-(e^{1-x}/2) & \mbox{if $1 \leq x$}
    \end{array} \right. \]
    
        \item
    
    \[ F_X(x) = \left\{ \begin{array}{cl}
         e^x/4 & \mbox{if $x < 0$} \\
         1 - (e^{-x}/4) & \mbox{if $x \geq 0$}
    \end{array} \right. \]
\end{enumerate}~\\

\begin{proof}[Solution] To show a function is cdf, we verify the conditions in \cite[Theorem 1.5.3 on page 31]{Berger-Casella}, which are routine computations.

\begin{enumerate}[label = (\alph*),leftmargin=*]
    \item 
    
    \[ F_X^{-1}(y) = -\log(1-y) \]
    
    \item 
    
    \[ F_X^{-1}(y) = \left\{ \begin{array}{cl}
         \log(2y) & \mbox{if $0 \leq y \leq \frac{1}{2}$} \\
         1 -\log(2(1-y)) & \mbox{if $\frac{1}{2} \leq y \leq 1$}
    \end{array} \right. \]
    
    \item 
    
    \[ F_X^{-1}(y) = \left\{ \begin{array}{cl}
         \log(4y) & \mbox{if $0 \leq y \leq \frac{1}{4}$} \\
         -\log(4(1-y)) & \mbox{if $\frac{1}{4} \leq y \leq 1$}
    \end{array} \right. \]
\end{enumerate}
\end{proof}

\newpage
\section{Problem 2.9}

If the random variable $X$ has pdf

\[ f(x) = \left\{ \begin{array}{cl}
     \frac{x-1}{2} & \mbox{$1<x<3$,} \\
     0 & \mbox{otherwise,}
\end{array} \right. \]
find a monotone function $u(x)$ such that the random variable $Y = u(X)$ has uniform(0,1) distribution.

\begin{proof}[Solution] This is a direct application of \cite[Theorem 2.1.10 on page 54]{Berger-Casella}. The cdf is given by

\begin{align*}
    F_X(x) &= \left\{ \begin{array}{cl}
         0 & \mbox{if $x \leq 1$} \\
         \myint{f(t)}{t}{1}{x} & \mbox{if $1 < x < 3$} \\
         1 & \mbox{if else}
    \end{array} \right. \\
    &= \left\{ \begin{array}{cl}
         0 & \mbox{if $x \leq 1$} \\
         \frac{(x-1)^2}{4} & \mbox{if $1 < x < 3$} \\
         1 & \mbox{if else}
    \end{array} \right.
\end{align*}
which is clearly monotone. So $u(x) = F_X(x)$.
\end{proof}

\newpage
\section{Problem 2.11}

Let $X$ have the standard normal pdf, $f_X(x) = (1/ \sqrt{2\pi}) e^{-x^2/2}$.

\begin{enumerate}[label=(\alph*),leftmargin=*]
    \item Find $EX^2$ directly, and then by using the pdf of $Y = X^2$ from \cite[Example 2.1.7 on page 52]{Berger-Casella} and calculating $EY$.
    
    \item Find the pdf of $Y = \Abs{X}$, and find its mean and variance.
\end{enumerate}~\\

\begin{proof}[Solution]~\\

\begin{enumerate}[label=(\alph*),leftmargin=*]
    \item First we have
    
    \begin{align*}
        EX^2 &= \myint{x^2 f_X(x)}{x}{-\infty}{\infty} & \left(\mbox{\cite[Definition 2.2.1 on page 55]{Berger-Casella}} \right) \\
        &= \myint{\frac{x^2}{\sqrt{2\pi}} e^{-x^2/2}}{x}{-\infty}{\infty} \\
        &= \frac{1}{\sqrt{2\pi}} \left[ xe^{-x^2/2} \vert^{x = \infty}_{x = -\infty} - \myint{e^{-x^2/2}}{x}{-\infty}{\infty} \right] \\
        &= 1.
    \end{align*}
    
    Secondly, by \cite[Example 2.1.7 on page 52]{Berger-Casella}, the pdf of $Y$ is given by
    
    \begin{align*}
        f_Y(y) &= \frac{1}{2 \sqrt{y}} \left[ f_X( \sqrt{y}) + f_X(-\sqrt{y}) \right] \\
        &= \frac{1}{\sqrt{2 \pi y}} e^{-y/2}.
    \end{align*}
    Therefore,
    
    \begin{align*}
        EY &= \myint{yf_Y(y)}{y}{0}{\infty} \\
           &= \myint{\sqrt{\frac{y}{2\pi}} e^{-y/2}}{y}{0}{\infty} \\
           &= 1
    \end{align*}
    
    \item Using \cite[Theorem 2.1.8 on page 53]{Berger-Casella}, $Y$ has pdf
    
    \begin{align*}
        f_Y(y) &= f_X(y) + f_X(-y) \\
               &= \sqrt{\frac{2}{\pi}} e^{-y^2/2}
    \end{align*}
    Therefore,
    
    \begin{align*}
        EY &= \myint{yf_Y(y)}{y}{0}{\infty} = \sqrt{\frac{2}{\pi}} \\
         \Var{Y} &= EY^2 - (EY)^2 = 1 - \frac{2}{\pi}
    \end{align*}
\end{enumerate}
\end{proof}

\newpage
\section{Problem 2.12} See \cite[page 77]{Berger-Casella} for the problem statement.

\begin{proof}[Solution] We know

\[ y = \underbrace{d \tan(x)}_{g(x)} \]
for $x \in (0, \pi/2)$, and

\begin{align*}
    \Dif{g^{-1}}{y} &= \Dif{}{y} \arctan \left( \frac{y}{d} \right) \\
                    &= \frac{d}{d^2 + y^2}.
\end{align*}
Therefore, \cite[Theorem 2.1.5 on page 51]{Berger-Casella} gives

\begin{align*}
    f_Y(y) &= f_X(g^{-1}(y)) \Abs{\Dif{g^{-1}}{y}} \\
           &= \frac{1}{\frac{\pi}{2} - 0} \cdot \frac{d}{d^2 + y^2} \\
           &= \frac{2d}{\pi (d^2 + y^2)}
\end{align*}
on $\mycal{Y} = (0, \infty)$, which is the Cauchy distribution. In particular, $EY = \infty$.
\end{proof}

\newpage
\section{Problem 2.13}

Consider a sequence of independent coin flips, each of which has probability $p$ of being heads. Define a random variable $X$ as the length of the run (of either heads or tails) started by the first trail. (For example, $X = 3$ if either TTTH or HHHT is observed.) Find the distribution of $X$, and find $EX$.

\begin{proof}[Solution] $X$ has pmf

\[ P(X = k) = (1-p)^kp + p^k(1-p). \]
Therefore,

\begin{align*}
    EX &= \sum_{k=1}^{\infty} k \left[ (1-p)^kp + p^k(1-p) \right] \\
       &= (1-p)p \left[ \sum_{k=1}^{\infty} k(1-p)^{k-1} + \sum_{k=1}^{\infty} kp^{k-1} \right] \\
       &= (1-p)p \left( \frac{1}{p^2} + \frac{1}{(1-p)^2} \right)
\end{align*}
\end{proof}

\newpage
\section{Probblem 2.14}

\begin{enumerate}[label=(\alph*),leftmargin=*]
    \item Let $X$ be a continuous, nonnegative random variable [$f(x) = 0$ for $x < 0$]. Show that
    
    \[ EX = \myint{[1-F_X(x)]}{x}{0}{\infty}, \]
    where $F_X(x)$ is the cdf of $X$.
    
    \item Let $X$ be a discrete random variable whose range is the nonnegative integers. Show that
    
    \[ EX = \sum_{k=0}^{\infty} (1-F_X(k)), \]
    where $F_X(k) = \fun{P}{X \leq k}$. Compare this with part (a).
\end{enumerate}~\\

\begin{proof}[Solution]~\\

\begin{enumerate}[label=(\alph*),leftmargin=*]
    \item 
    
    \begin{align*}
        \myint{[1-F_X(x)]}{x}{0}{\infty} &= \myint{P(X > x)}{x}{0}{\infty} \\
        &= \myint{\myint{f_X(y)}{y}{x}{\infty}}{x}{0}{\infty} \\
        &= \myint{\myint{f_X(y)}{x}{0}{y}}{y}{0}{\infty} \\
        &= \myint{yf_X(y)}{y}{0}{\infty} \\
        &= EX
    \end{align*}
    
    \item
    
    \begin{align*}
        EX &= \sum_{k=0}^{\infty} k \fun{P}{X = k} \\
           &= \sum_{k=1}^{\infty} \fun{P}{X=k} + \sum_{k=2}^{\infty} \fun{P}{X=k} + \sum_{k=3}^{\infty} \fun{P}{X=k} + \cdots \\
           &= \fun{P}{X > 0} + \fun{P}{X > 1} + \fun{P}{X > 2} + \cdots \\
           &= \sum_{k=0}^{\infty} 1-F_X(k)
    \end{align*}
\end{enumerate}
\end{proof}

\newpage
\section{Problem 2.18}

Show that if $X$ is a continuous random variable, then

\[ \min_a E \Abs{X-a} = E \Abs{X-m}, \]
where $m$ is the median of $X$.

\begin{proof}[Solution] The expected value of $\Abs{X-a}$ is given by

\begin{align*}
    E\Abs{X-a} &= \myint{\Abs{x-a} f_X(x)}{x}{-\infty}{\infty} \\
    &= \myint{(x-a)f_X(x)}{x}{a}{\infty} - \myint{(x-a)f_X(x)}{x}{-\infty}{a} \\
\end{align*}
Differentiate with respect to $a$ we have

\begin{align*}
    \Dif{}{a} E\Abs{X-a} &= \Dif{}{a} \left[ \myint{(x-a)f_X(x)}{x}{a}{\infty} \right] - \Dif{}{a} \left[ \myint{(x-a)f_X(x)}{x}{-\infty}{a} \right] \\
    &= \myint{\dif{}{a}\left[(x-a)f_X(x) \right]}{x}{a}{\infty} - \myint{\dif{}{a} \left[ (x-a)f_X(x) \right]}{x}{-\infty}{a} \\
    &= \myint{f_X(x)}{x}{-\infty}{a} - \myint{f_X(x)}{x}{a}{\infty} \\
    &= \fun{P}{X \leq a} - \fun{P}{X > a}.
\end{align*}
In particular,

\[ 1- 2\fun{P}{X > a} = \Dif{}{a} E\Abs{X-a} = 1 - 2\fun{P}{X \leq a}. \]
Therefore, the solution to

\[ \Dif{}{a} E\Abs{X-a} = 0 \]
is the median $m$. Moreover, $m$ is a minima because

\begin{align*}
    \left.\Dif{^2}{a^2}\right\vert_{a=m} E \Abs{X-a} = 2f_X(m) > 0.
\end{align*}
\end{proof}

\newpage
\section{Problem 2.19}

Prove that

\[ \Dif{}{a} E(X-a)^2 = 0 \iff EX = a \]
by differentiating the integral. Verify, using calculus, that $a = EX$ is indeed a minimum. List the assumptions about $F_X$ and $f_X$ are needed.

\begin{proof}[Solution]
We have

\begin{align*}
    \Dif{}{a} E(X-a)^2 &= \Dif{}{a} \myint{(x-a)^2 f_X(x)}{x}{-\infty}{\infty} \\
    &= \myint{\dif{}{a} \left[ (x-a)^2 f_X(x) \right]}{x}{-\infty}{\infty} \\
    &=-2 \myint{(x-a)f_X(x)}{x}{-\infty}{\infty} \\
    &= -2 E(X-a)
\end{align*}
Therefore,

\begin{align*}
    \Dif{}{a} E(X-a)^2 = 0 &\iff -2E(X-a) = 0 \\
                           &\iff E(X-a) = 0 \\
                           &\iff EX = a.
\end{align*}
To verify $a = EX$ is minimum, we compute the second derivative

\[ \Dif{^2}{a^2} E(X-a)^2 = 2 > 0.\]
\end{proof}

\newpage
\section{Proboelm 2.21}

Prove the ``two-way'' rule for expectations, \cite[Equation (2.2.5) on page 58]{Berger-Casella}, which says $Eg(X) = EY$ where $Y = g(X)$. Assume that $g(x)$ is a monotone function.

\begin{proof}[Solution]
\begin{align*}
    Eg(X) &= \myint{g(x) f_X(x)}{x}{\D{R}}{} \\
          &= \myint{y f_X(g^{-1}(y)) \cdot \Dif{g^{-1}}{y}}{y}{\D{R}}{} \\
          &= \myint{y f_Y(y)}{y}{\D{R}}{} \\
          &= EY
\end{align*}
\end{proof}

\newpage
\section{Problem 2.22}

Let $X$ have the pdf

\[ f(x) = \frac{4}{\beta^3 \sqrt{\pi}} x^2 e^{-x^2/ \beta^2}, \ 0<x<\infty, \ \beta > 0. \]

\begin{enumerate}[label=(\alph*),leftmargin=*]
    \item Verify that $f(x)$ is a pdf.
    \item Find $EX$ and $\Var{X}$.
\end{enumerate}~\\

\begin{proof}[Solution]~\\

\begin{enumerate}[label=(\alph*),leftmargin=*]
    \item  \cite[Theorem 1.6.5 on page 36]{Berger-Casella}.
    \item
    
    \begin{align*}
        EX 
        &= \myint{xf(x)}{x}{0}{\infty} \\
        &= \myint{\frac{4}{\beta^3 \sqrt{\pi}} x^3 e^{-x^2/ \beta^2}}{x}{0}{\infty} \\
        &= \frac{4}{\beta^3 \sqrt{\pi}} \myint{x^3 e^{-x^2/ \beta^2}}{x}{0}{\infty} \\
        &= \left( \frac{4}{\beta^3 \sqrt{\pi}} \right) \left(- \frac{\beta^2}{2} \right) \left( - \myint{2xe^{-x^2/\beta^2}}{x}{0}{\infty} \right) \\
        &= \left( \frac{4}{\beta^3 \sqrt{\pi}} \right) \left( \frac{\beta^4}{2} \right) \\
        &= \frac{2\beta}{\sqrt{\pi}}
    \end{align*}
    and similarly,
    
    \begin{align*}
        EX^2 &= \frac{3\beta^2}{2}, \\
        \Var{X} &= EX^2 - (EX)^2 \\
                &= \beta^2 \left[ \frac{3}{2} - \frac{4}{\pi} \right]
    \end{align*}
\end{enumerate}
\end{proof}

\newpage
\section{Problem 2.23}

Let $X$ have the pdf 

\[ f(x) = \frac{1}{2}(1+x), \ -1 < x < 1. \]
\begin{enumerate}[label=(\alph*),leftmargin=*]
    \item Find the pdf of $Y = X^2$.
    \item Find $EY$ and $\Var{Y}$.
\end{enumerate}~\\

\begin{proof}[Solution]~\\
\begin{enumerate}[label=(\alph*),leftmargin=*]
    \item Define $g_i(x) = x^2$ on $A_1 = (-1,0)$ and $A_2 = (0,1)$. Then
    
    \begin{align*}
        f_Y(y) &= \left[ f(-\sqrt{y}) + f(\sqrt{y}) \right] \cdot \frac{1}{2\sqrt{y}} \\
        &= \frac{1}{2\sqrt{y}}
    \end{align*}
    on $\mycal{Y} = (0,1)$.
    
    \item We have
    
    \begin{align*}
        \myint{y^n f_Y(y)}{y}{0}{1} &= \frac{1}{2} \myint{y^{n-1/2}}{y}{0}{1} \\
        &= \frac{1}{2n+1}.
    \end{align*}
    This gives
    
    \begin{align*}
        EY &= \frac{1}{3} \\
        EY^2 &= \frac{1}{5} \\
        \Var{Y} &= \frac{4}{45}
    \end{align*}
\end{enumerate}
\end{proof}

\newpage
\section{Problem 2.26}

Let $f(x)$ be a pdf and let $a$ be a number such that, for all $\ve > 0$, $f(a + \ve) = f(a - \ve)$. Such a pdf is said to be summetric about the point $a$.

\begin{enumerate}[label=(\alph*),leftmargin=*]
    \item Give three examples of symmetric pdfs.
    \item Show that if $X \sim f(x)$, symmetric, then the median of $X$ (see Exercise 2.17) is the number $a$.
    \item Show that if $X \sim f(x)$, symmetric and $EX$ exists, then $EX = a$.
    \item Show that $f(x) = e^{-x}$, $x \geq 0$, is not a symmetric pdf.
    \item Show that for the pdf in part (d), the median is less than the mean.
\end{enumerate}~\\

\begin{proof}[Solution]~\\
\begin{enumerate}[label=(\alph*),leftmargin=*]
    \item Cauchy, Normal, Uniform.
    
    \item By change of variable, we may assume $a = 0$. The statement thus becomes: the median of an even pdf is 0, which is obvious because
    
    \begin{align*}
         1 &= \myint{f(x)}{x}{-\infty}{\infty} \\
         &= \myint{f(x)}{x}{-\infty}{0} + \myint{f(x)}{x}{0}{\infty} \\
         &= 2 \myint{f(x)}{x}{-\infty}{0} \\
         &= 2 \fun{P}{X \leq 0}
    \end{align*}
    
    \item Following the same logic in part (b), the statement becomes: the expected value of an even pdf $f(x)$ is 0.
    
    This is true because the funcion $xf(x)$ is odd, hence
    
    \begin{align*}
        EX &= \myint{xf(x)}{x}{-\infty}{\infty} \\
           &= 0
    \end{align*}
    
    \item If $f(x) = e^{-x}$ were symmetric, then it would be symmetric at $x = EX$ for $X \sim f(x)$ by part (c). In particular,
    
    \begin{align*}
        EX &= \myint{xe^{-x}}{x}{0}{\infty} \\
           &= 1.
    \end{align*}
    However, a direct computation shows $1$ is not the median of $X$, contradicting part (b). Therefore it is not symmetric.
    
    \item As computed in part (d), the mean is $1$. The claim follows from the computation
    
    \begin{align*}
        \myint{f(x)}{x}{0}{\mbox{mean}} &= \myint{e^{-x}}{x}{0}{1} \\
                                        &= \frac{e-1}{e} \\
                                        &> \frac{1}{2}
    \end{align*}
\end{enumerate}
\end{proof}

\newpage
\section{Exercise 2.32}

We compute

\begin{align*}
    \left.\Dif{}{t} \right\vert_{t=0} \fun{S}{t}
    &= \left.\Dif{}{t} \right\vert_{t=0} \fun{\log}{\fun{M_X}{t}} \\
    &= \frac{\fun{\dot{M}_X}{0}}{\fun{M_X}{0}} \\
    &= EX,
\end{align*}
and

\begin{align*}
    \left.\Dif{^2}{t^2} \right\vert_{t=0} \fun{S}{t}
    &= \frac{\fun{\ddot{M}_X}{0} \fun{M_X}{0} - \fun{\dot{M}^2_X}{0}}{\fun{M^2_X}{0}} \\
    &= EX^2 - (EX)^2 \\
    &= \Var{X}.
\end{align*}

\newpage
\section{Exercise 2.33}

\begin{enumerate}[label=(\alph*),leftmargin=*]
    \item The mgf is
    
    \begin{align*}
        \fun{M_X}{t} &= \sum_{x=0}^{\infty} e^{tx} \cdot \frac{e^{-\lambda} \lambda^x}{x!} \\
        &= e^{-\lambda} \sum_{x=0}^{\infty} \frac{\left(  \lambda e^t \right)^x}{x!} \\
        &= e^{-\lambda} \cdot e^{\lambda e^t} \\
        &= e^{\lambda \left( e^t - 1 \right)}.
    \end{align*}
    The moments are
    
    \begin{align*}
        EX &= \left.\Dif{}{t}\right\vert_{t=0} M_X(t) \\
        &= \left. \lambda e^t \cdot e^{\lambda \left( e^t - 1 \right)} \right\vert_{t=0} \\
        &= \lambda, \\
        \\
        EX^2 &= \left.\Dif{^2}{t^2}\right\vert_{t=0} M_X(t) \\
        &= \left. \lambda \left( 1 + e^t \lambda \right) \cdot e^{\lambda \left( e^t - 1 \right)} \right\vert_{t=0} \\
        &= \lambda^2 + \lambda.
    \end{align*}
    Therefore,
    
    \[ \Var{X} = \lambda. \]
    
    \item The mgf is
    
    \begin{align*}
        \fun{M_X}{t} &= \sum_{x=0}^{\infty} e^{tx} p(1-p)^x \\
        &= \sum_{x=0}^{\infty} p \left[ e^t (1-p) \right]^x \\
        &= \frac{p}{1-e^t(1-p)}
    \end{align*}
    
    The moments are
    
    \begin{align*}
        EX &= \frac{1-p}{p}, \\
        \\
        EX^2 &= \frac{(2-p)(1-p)}{p^2}
    \end{align*}
    Therefore,
    
    \begin{align*}
        \Var{X} = \frac{1-p}{p^2}.
    \end{align*}
    
    \item The mgf is
    
    \begin{align*}
        \fun{M_X}{t} &= \myint{e^{tx} \cdot \frac{e^{-(x-\mu)^2/(2 \sigma^2)}}{\sqrt{2\pi} \sigma}}{x}{-\infty}{\infty} \\
        &= \myint{\frac{e^{- \left( x^2 - 2x \mu + \mu^2 - 2 \sigma^2 tx \right)/(2 \sigma^2)}}{\sqrt{2\pi} \sigma}}{x}{-\infty}{\infty} \\
        &= e^{ \frac{2\mu \sigma^2t + \sigma^4 t^2}{2 \sigma^2}} \cdot \myint{\frac{e^{-\frac{\left[ x - \left( \mu + \sigma^2 t \right) \right]^2}{2 \sigma^2}}}{\sqrt{2\pi} \sigma}}{x}{-\infty}{\infty} \\
        &= e^{\mu t + \frac{\sigma^2 t^2}{2}}.
    \end{align*}
    The moments are
    
    \begin{align*}
        EX &= \left. \Dif{}{t} \right\vert_{t=0} \fun{M_X}{t} \\
        &= \mu, \\
        \\
        EX^2 &= \left. \Dif{^2}{t^2} \right\vert_{t=0}  \fun{M_X}{t} \\
        &= \sigma^2 + \mu^2.
    \end{align*}
    Therefore,
    
    \begin{align*}
        \Var{X} &= EX^2 - (EX)^2 \\
                &= \sigma^2
    \end{align*}
\end{enumerate}

\newpage
\nocite{*}
\printbibliography

\end{document}