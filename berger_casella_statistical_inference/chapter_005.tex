\documentclass[12pt,letterpaper,reqno]{amsart}


\usepackage{lipsum} %separate the paragraphs of the dummy text into TEX-paragraphs.

\usepackage{tikz} %commutative diagrams

\usepackage{setspace} %Provides support for setting the spacing between lines in a document.

\usepackage{amsmath,amssymb} %provides miscellaneous enhancements for improving the information structure and printed output of documents containing mathematical formulas. e.g., \DeclareMathOperator, \sin, \lim

\usepackage{enumitem} %additional options for \enumerate

\usepackage{subfig} %provides support for the manipulation and reference of small or ‘sub’ figures and tables within a single figure or table environment.

\usepackage{framed} %box, shaded, put a left line around a region.

\usepackage{etoolbox} %provides LATEX frontends to some of the new primitives provided by e-TEX as well as some generic tools which are not strictly related to e-TEX but match the profile of this package.

\usepackage{bm} %Access bold symbols in maths mode

\usepackage{mdframed} %Framed environments that can split at page boundaries

\usepackage{mathrsfs} %Support for using RSFS fonts in maths

\usepackage{centernot} %Centred \not command

\usepackage{thmtools} %Extensions to theorem environments

\usepackage{thm-restate} %Restate Theorem

\usepackage{fancyvrb} %Sophisticated verbatim text

\usepackage{dsfont} %?

\usepackage{bbm} %Blackboard variants of Computer Modern fonts.

\usepackage{breqn} %Automatic line breaking of displayed equations

\usepackage{adjustbox} %The package provides several macros to adjust boxed content.

\usepackage{tabularx} %The package defines an environment tabularx, an extension of tabular which has an additional column designator, X, which creates a paragraph-like column whose width automatically expands so that the declared width of the environment is filled. (Two X columns together share out the available space between them, and so on.)

\usepackage{calligra} %Calligraphic font

\usepackage[
    backend=bibtex, 
    natbib=true, 
    bibstyle=verbose, citestyle=verbose,    % bibstyle extensively modifed below
    doi=true, url=true,                     % excluded from citations below
    citecounter=true, citetracker=true,
    block=space, 
    backref=false, backrefstyle=two,
    abbreviate=false,
    isbn=true, maxbibnames=4, maxcitenames=4,
    style=alphabetic
]{biblatex} %Adding Reference to your paper.

\usepackage[margin=1in]{geometry} %edit margin of your document

\usepackage[makeindex]{imakeidx} %making index

\usepackage{empheq} %The package provides a visual markup extension to amsmath.

\usepackage{longtable} %allow longtable

\usepackage[mathscr]{eucal}%change mathscr

\usepackage{xpatch} %fix skipbelow in mdframe

\usepackage{standalone}  %need this to include the tikz picture created in different files.

\usepackage{wasysym}

\usepackage{bashful}

\usepackage{hyperref,cleveref} %Extensive support for hypertext in LATEX (ALWAYS LOAD LAST!) %this loads the packages


\makeindex

\author{Virgil Chan}
\title{Casella-Berger \\ Statistical Inference Solution: \\ Chapter 5}
\date{August 30, 2022}


\usetikzlibrary{patterns,positioning,arrows,chains,matrix,positioning,scopes} %options for tikzpicture

\addbibresource{bibliography.bib} %put your reference here


\makeatletter
\patchcmd{\@maketitle}
  {\ifx\@empty\@dedicatory}
  {\ifx\@empty\@date \else {\vskip3ex \centering\footnotesize\@date\par\vskip1ex}\fi
   \ifx\@empty\@dedicatory}
  {}{}
\patchcmd{\@adminfootnotes}
  {\ifx\@empty\@date\else \@footnotetext{\@setdate}\fi}
  {}{}{}
\makeatother

\makeatletter
\xpatchcmd{\endmdframed}
  {\aftergroup\endmdf@trivlist\color@endgroup}
  {\endmdf@trivlist\color@endgroup\@doendpe}
  {}{}
\makeatother
%\topsep
\newtheoremstyle{break}
  {\topsep}% Space above
  {\topsep}% Space below
  {\it}% Body font
  {}% Indent amount
  {\bfseries}% Theorem head font
  {}% Punctuation after theorem head
  {\topsep}% Space after theorem head, ' ', or \newline
  {\thmname{#1}\thmnumber{ #2} \thmnote{(#3)}}% Theorem head spec (can be left empty, meaning `normal')

\mdfdefinestyle{box}{
     linecolor=white,
    skipabove=\topsep,
    skipbelow=\topsep,
    innerbottommargin=\topsep,
}  %

\theoremstyle{break}
\newmdtheoremenv[style=box]{theorem}{Theorem}[section]
\newmdtheoremenv[style=box]{lemma}[theorem]{Lemma}
\newmdtheoremenv[style=box]{proposition}[theorem]{Proposition}
\newmdtheoremenv[style=box]{corollary}[theorem]{Corollary}
\newmdtheoremenv[style=box]{definition}[theorem]{Definition}
\newmdtheoremenv[style=box]{addendum}[theorem]{Addendum}
\newmdtheoremenv[style=box]{conjecture}[theorem]{Conjecture}
\newmdtheoremenv[style=box]{question}[theorem]{Question}
\newmdtheoremenv[style=box]{condit}[theorem]{Condition}

\makeatletter
\def\th@plain{%
  \thm@notefont{}% same as heading font
  \normalfont % body font
}
\def\th@definition{%
  \thm@notefont{}% same as heading font
  \normalfont % body font
}
\makeatother

\newtheoremstyle{exampstyle}
{\topsep} % Space above
{\topsep} % Space below
{} % Body font
{} % Indent amount
{\bfseries} % Theorem head font
{} % Punctuation after theorem head
{\topsep} % Space after theorem head
{\thmname{#1}\thmnumber{ #2} \thmnote{(#3)}} % Theorem head spec (can be left empty, meaning `normal')

\theoremstyle{exampstyle}
\newtheorem{example}[theorem]{Example}
\newtheorem{remark}[theorem]{Remark}


\pdfpagewidth 8.5in
\pdfpageheight 11in

    
%\BEGIN{COMMAND}

%SHORT HAND NOTATION-------------------------------
\newcommand{\D}[1]{\mathbb{#1}} %short hand mathbb command

\newcommand{\Dm}[1]{\mathbbm{#1}} %mathbb for numbers

\newcommand{\mycal}[1]{\mathcal{{#1}}} %short hand mathcal

\newcommand{\scr}[1]{\mathscr{{#1}}} %short hand mathscr

\newcommand{\ve}{\varepsilon} %short hand epsilon

\newcommand\numberthis{\addtocounter{equation}{1}\tag{\theequation}}%label equation in \align*

\newcommand{\dund}[1]{\underline{\underline{{#1}}}}%Double underline

\newcommand{\Frechet}{Fr\'{e}chet}%Frechet space

\newcommand{\ol}[1]{\overline{{#1}}}%overline

\newcommand{\ul}[1]{\underline{{#1}}}%underline

\newcommand{\dul}[1]{\ul{\ul{{#1}}}}%double underline

\newcommand{\Cech}{\v{C}ech} %Cech

\newcommand{\Nother}{N\"{o}ther} %Nother

\newcommand{\Notherian}{N\"{o}ther} %Notherian

\newcommand{\wt}[1]{\widetilde{{#1}}} %short hand for \widetilde

\newcommand{\fun}[2]{{#1}\! \left( {#2} \right)}%short hand for function

\newcommand{\on}[1]{\operatorname{{#1}}} %short hand for \operatorname

\newcommand{\Kp}[1]{K^+_{{#1}}} %short hand for plus-construction

\newcommand{\Knp}[1]{K_{{#1}}} %short hand for before plus-construction

\newcommand{\btimes}{\boxtimes} %short hand for box tensor

\newcommand{\bplus}{\boxplus} %short hand for box sum

\newcommand{\myline}{\par\noindent\rule{\textwidth}{0.4pt}} %horizontal line

\newcommand{\wh}[1]{\widehat{{#1}}} %short hand for \widetilde

\newcommand{\myref}[2]{\hyperref[{#2}]{{#1}~\ref{#2}}} %short hand for \hyperref

\newcommand{\myeqref}[2]{\hyperref[{#2}]{{#1}~\eqref{#2}}} %short hand for \hyperref for equations

%SET THEORY----------------------------------------
\newcommand{\SET}[2]{\left\{ {#1} \ \middle| \ {#2} \right\}} %short hand notation for set

\newcommand{\SETT}[1]{\left\{ {#1} \right\}} %short hand notation for set without constraint

\newcommand{\floor}[1]{\left \lfloor{#1}\right \rfloor } %floor function

\newcommand{\Rng}[1]{\operatorname{Rng}\left(#1\right)} %range of a function

\newcommand{\id}[1]{\operatorname{id}_{{#1}}} %identity map

\newcommand\restr[2]{{% we make the whole thing an ordinary symbol
  \left.\kern-\nulldelimiterspace % automatically resize the bar with \right
  #1 % the function
  \vphantom{\big|} % pretend it's a little taller at normal size
  \right|_{#2} % this is the delimiter
  }} %restriction of a function

\newcommand{\Abs}[1]{\left\lvert{#1}\right\rvert} %absolute value

\newcommand{\nonnegint}{\D{N} \cup \left\{ 0 \right\}} %Non-negative integers

\newcommand{\Image}[1]{\operatorname{im}\left( {#1} \right)} %Image of a map

\newcommand{\Proj}[1]{\operatorname{proj}_{{#1}}}%projection map

\newcommand{\ev}[1]{\operatorname{ev}_{{#1}}}%Evaluation map

\makeatletter
\newbox\xrat@below
\newbox\xrat@above
\newcommand{\xrightarrowtail}[2][]{%
  \setbox\xrat@below=\hbox{\ensuremath{\scriptstyle #1}}%
  \setbox\xrat@above=\hbox{\ensuremath{\scriptstyle #2}}%
  \pgfmathsetlengthmacro{\xrat@len}{max(\wd\xrat@below,\wd\xrat@above)+.6em}%
  \mathrel{\tikz [>->,baseline=-.75ex]
                 \draw (0,0) -- node[below=-2pt] {\box\xrat@below}
                                node[above=-2pt] {\box\xrat@above}
                       (\xrat@len,0) ;}}
\makeatother %xrightarrowtail

\makeatletter
\newbox\xrat@below
\newbox\xrat@above
\newcommand{\xtwoheadrightarrow}[2][]{%
  \setbox\xrat@below=\hbox{\ensuremath{\scriptstyle #1}}%
  \setbox\xrat@above=\hbox{\ensuremath{\scriptstyle #2}}%
  \pgfmathsetlengthmacro{\xrat@len}{max(\wd\xrat@below,\wd\xrat@above)+.6em}%
  \mathrel{\tikz [->>,baseline=-.75ex]
                 \draw (0,0) -- node[below=-2pt] {\box\xrat@below}
                                node[above=-2pt] {\box\xrat@above}
                       (\xrat@len,0) ;}}
\makeatother %xtwoheadrightarrow

\newcommand{\dom}[1]{\operatorname{dom} \left( {#1} \right)} %Domain of a function

\newcommand{\codom}[1]{\operatorname{codom} \left( {#1} \right)} %Codomain of a function

\newcommand{\sseq}{\subseteq} %short hand for \subseteq
  
%ALGEBRA-------------------------------------------
\newcommand{\Aut}[2]{\operatorname{Aut}_{{#2}}\left({#1} \right)} %automorphism group

\newcommand{\Hom}[3]{\operatorname{Hom}_{{#3}}\left({#1},{#2}\right)} %Homomorphism group

\newcommand{\Char}[1]{\operatorname{char}\left( {#1} \right)} %characteristic of a ring/field

\newcommand{\Rad}[1]{\operatorname{Rad}\left({#1}\right)} %radical

\newcommand{\Nil}[1]{\operatorname{Nil}\left({#1}\right)} %nilradical

\newcommand{\Tors}[1]{\operatorname{Tors}\left({#1} \right)} %Torsion 

\newcommand{\Gal}[2]{\operatorname{Gal}\left({#1}/{#2}\right)} %Galois group

\newcommand{\pid}[1]{\left\langle {#1} \right\rangle} %Principal Ideal

\newcommand{\polynomial}[5]{\sum_{{#1}={#2}}^{{#3}} {#4}_{{#1}} {{#5}}^{{#1}}} %Polynomial

\newcommand{\mydeg}[1]{\deg \left( {#1} \right)} %Degree

\newcommand{\GL}[1]{GL \left({#1}\right)} %General linear group

\newcommand{\abelian}[1]{{#1}^{ab}} %Abelianisation

\newcommand{\Orb}[2]{\operatorname{Orb}_{{#1}}\left( {#2} \right)} %orbit of group action

\newcommand{\Stab}[2]{\operatorname{Stab}_{{#1}}\left( {#2} \right)} %stabiliser of group action

\newcommand{\GROUP}[2]{\left\langle {#1} \ \middle| \ {#2} \right\rangle} %group presentation

\newcommand{\ideal}{\trianglelefteq}%normal subgroup symbol

\newcommand{\properideal}{\triangleleft}%proper normal subgroup symbol

\newcommand{\groupaction}[3]{{#1} \rotatebox[origin=c]{180}{$\circlearrowright$}_{{#3}} {#2}}%group action

\newcommand{\SL}[1]{SL \left({#1}\right)} %special linear group

\newcommand{\iso}[1]{\operatorname{iso}\left( {#1} \right)}%Group of isomorphisms

\newcommand{\Gr}[1]{\operatorname{Gr}\left( {#1} \right)}%associated graded group

\newcommand{\bigboxplus}{
  \mathop{
    \vphantom{\bigoplus} 
    \mathchoice
      {\vcenter{\hbox{\resizebox{\widthof{$\displaystyle\bigoplus$}}{!}{$\boxplus$}}}}
      {\vcenter{\hbox{\resizebox{\widthof{$\bigoplus$}}{!}{$\boxplus$}}}}
      {\vcenter{\hbox{\resizebox{\widthof{$\scriptstyle\oplus$}}{!}{$\boxplus$}}}}
      {\vcenter{\hbox{\resizebox{\widthof{$\scriptscriptstyle\oplus$}}{!}{$\boxplus$}}}}
  }\displaylimits 
} %internal direct sum

\newcommand{\bideg}[1]{\text{bi-deg}\left( {#1} \right)}%bi-degree

\newcommand{\Stgroup}[1]{\operatorname{St}\left({#1} \right)} %Steinberg group

\newcommand{\groupp}[2]{\left\langle {#1} \ \middle| \ {#2} \right\rangle} %group presentation

%LINEAR ALGEBRA------------------------------------
\newcommand{\rank}[1]{\operatorname{{rank}}\left({#1}\right)} %rank

\newcommand{\diag}[1]{\operatorname{diag}\left({#1} \right)} %diagonal matrix

\newcommand{\Ker}[1]{\operatorname{ker}\left({#1} \right)} %Kernel of a map

\newcommand{\Tr}[1]{\operatorname{trace}\left({#1}\right)} %trace

\newcommand{\mat}[4]{\left[ \begin{array}{cc}
{#1} & {#2}  \\
{#3} & {#4}
\end{array} \right]} %2x2 matrix

\newcommand{\Range}[1]{\operatorname{Rng}\left( {#1} \right)} %Range of a map

\newcommand{\Span}[1]{\operatorname{span}\left( {#1} \right)} %span of a set

\newcommand{\Null}[1]{\operatorname{null}\left({#1}\right)} %nullity

\newcommand{\innerprod}[3]{\left\langle {#1}, {#2} \right\rangle_{{#3}}} %Inner product

\newcommand{\colvec}[2]{\left[ \begin{array}{c}
{#1} \\
\vdots \\
{#2}
\end{array} \right]}%column vector

\newcommand{\Dim}[2]{\dim_{{#2}} \left( {#1} \right)}%dimension

\newcommand{\Det}[1]{\det \left( {#1} \right)}%determinant

\newcommand{\coDim}[2]{\operatorname{codim}_{{#2}} \left( {#1} \right)}%codimension

%COMMUTATIVE ALGEBRA AND ALGEBRAIC GEOMETRY--------
\newcommand{\td}[1]{\operatorname{tr.deg.}\left( {#1} \right)} %transcendence degree

\newcommand{\Spec}[1]{\operatorname{Spec}\left({#1}\right)} %spectrum of a ring

\newcommand{\height}{\operatorname{height}} %height

\newcommand{\mSpec}[1]{\operatorname{mSpec}\left( {#1}\right)} %maximal ideals of a ring

\newcommand{\length}[1]{\operatorname{length}\left({#1} \right)} %length

\newcommand{\Pic}[1]{\operatorname{Pic}\left({#1}\right)}  %Piccard group

\newcommand{\Idem}[1]{\operatorname{Idem}\left( {#1} \right)}%Idempotent

\newcommand{\sections}[1]{\Gamma\left({#1} \right)}%sections of sheaf/bundle

\newcommand{\kerpre}[1]{\operatorname{ker}_{\operatorname{presheaf}}\left( {#1} \right)}%presheaf kernel

\newcommand{\cokerpre}[1]{\operatorname{coker}_{\operatorname{presheaf}}\left( {#1} \right)}%presheaf cokernel

\newcommand{\RES}[2]{\operatorname{res}_{{#1},{#2}}}%restriction of a presheaf

%Complex Analysis----------------------------------

\newcommand{\expi}[2]{e^{\frac{{#1}}{{#2}}}} %complex exponential with imaginary fraction exponent

\newcommand{\Res}[2]{\operatorname{Res} \left[ {#1},{#2} \right]} %residue

\newcommand{\myRe}[1]{\operatorname{Re}\left({#1} \right)} %Real part

\newcommand{\myIm}[1]{\operatorname{Im}\left({#1} \right)} %Imaginary part

\newcommand{\winding}[2]{\fun{\operatorname{Ind}_{{#1}}}{{#2}}}%winding number of a curve

\newcommand{\polydisc}[1]{\D{D}^{{#1}}_{\operatorname{poly}}}%poly-disc

\newcommand{\distbound}[1]{\partial^{\operatorname{dist}}\polydisc{{#1}}}%distinguished boundary

\newcommand{\Blaschke}[1]{\fun{\operatorname{Blaschke}}{{#1}}}%Blaschke product

%HOMOLOGICAL ALGEBRA AND CATEGORY THEORY-----------
\newcommand{\Mor}[2]{\operatorname{mor}_{{#2}}\left({#1}\right)} %Morphism class

\newcommand{\cat}[1]{\normalfont{\mathbf{#1}}} %notation for category

\newcommand{\Tor}[3]{\operatorname{Tor}_{{#3}}\left( {#1}, {#2} \right)} %Tor functor
\newcommand{\catset}{\operatorname{\scr{S}ets}} %category of sets

\newcommand{\Ring}{\operatorname{\scr{R}ings}} %category of rings

\newcommand{\Obj}[1]{\operatorname{obj}\left({#1} \right)} %Object class of category

\newcommand{\Ext}[2]{\operatorname{Ext}\left( {#1}, {#2} \right)} %Ext-functor

\newcommand{\coker}[1]{\operatorname{{coker}}\left( {#1} \right)} %cokernel

\makeatletter
\newcommand{\colim@}[2]{%
  \vtop{\m@th\ialign{##\cr
    \hfil$#1\operator@font colim$\hfil\cr
    \noalign{\nointerlineskip\kern1.5\ex@}#2\cr
    \noalign{\nointerlineskip\kern-\ex@}\cr}}%
}
\newcommand{\colim}[1]{%
  \mathop{\mathpalette\colim@{}}_{{#1}}
} %colimit

\renewcommand{\varprojlim}{%
  \mathop{\mathpalette\varlim@{\leftarrowfill@\scriptscriptstyle}}\nmlimits@
}
\renewcommand{\varinjlim}{%
  \mathop{\mathpalette\varlim@{\rightarrowfill@\scriptscriptstyle}}\nmlimits@
} %limit

\newcommand{\hocolim@}[2]{%
  \vtop{\m@th\ialign{##\cr
    \hfil$#1\operator@font hocolim$\hfil\cr
    \noalign{\nointerlineskip\kern1.5\ex@}#2\cr
    \noalign{\nointerlineskip\kern-\ex@}\cr}}%
}
\newcommand{\hocolim}[1]{%
  \mathop{\mathpalette\hocolim@{}}_{{#1}}
}%homotopy colimit

\newcommand{\holim@}[2]{%
  \vtop{\m@th\ialign{##\cr
    \hfil$#1\operator@font holim$\hfil\cr
    \noalign{\nointerlineskip\kern1.5\ex@}#2\cr
    \noalign{\nointerlineskip\kern-\ex@}\cr}}%
}
\newcommand{\holim}[1]{%
  \mathop{\mathpalette\holim@{}}_{{#1}}
}%homotopy limit

\newcommand{\tensor}[3]{{#1} \otimes_{{#3}} {#2}}%tensor product

\newcommand{\Eq}[1]{\operatorname{Eq} \left( {#1} \right)}%Equaliser

\newcommand{\coEq}[1]{\operatorname{coEq} \left( {#1} \right)}%Coequaliser

\newcommand{\catfgProj}[1]{\operatorname{\scr{P}roj}^{\operatorname{fg}}_{{#1}}} %category of finitely generated projective R-modules

\newcommand{\catmodule}[1]{\operatorname{\scr{M}odule}_{{#1}}} %category of R-modules

\newcommand{\catspace}{\operatorname{\scr{S}paces}} %category of spaces

\newcommand{\catspectra}{\operatorname{\scr{S}pectra}} %category of spectra

\newcommand{\catab}{\operatorname{\scr{A}belian}} %category of Abelian groups

\newcommand{\catringoid}{\operatorname{\scr{R}ingoids}} %category of ringoids

\newcommand{\catgroup}{\operatorname{\scr{G}roups}} %category of groups

\newcommand{\catfgfree}[1]{\operatorname{\scr{F}ree}^{\operatorname{fg}}_{{#1}}} %category of finitely generated free R-modules

\newcommand{\hormor}[1]{\operatorname{hor-mor}\left({#1}\right)} % horizontal morphism class

\newcommand{\vermor}[1]{\operatorname{ver-mor}\left({#1}\right)} % vertical morphism class

\newcommand{\bimor}[1]{\operatorname{bimor}\left({#1}\right)} % bi-morphism class

\newcommand{\catiso}[1]{\operatorname{iso}\left({#1}\right)}%category of isomorphisms

\newcommand{\SiS}[1]{\scr{S}_{{#1}}} %S-inverse-S-construction of category of isomorphisms

\newcommand{\CMA}[2]{\scr{C}_{{#2}}\left( {#1} \right)} %Pedersen-Weibel category

\newcommand{\catfinset}{\operatorname{\scr{F}in\scr{S}et}} %category of finite set

\newcommand{\catSMC}{\operatorname{\scr{S}ym\scr{M}on\scr{C}at}} %category of symmetric monoidal categories

\newcommand{\PiP}[1]{{\scr{P}_{{#1}}}} %S-inverse-S-construction of category of isomorphisms in the idempotent completion

\newcommand{\catoofree}[1]{\operatorname{\scr{F}ree}^{\mathbb{N}}_{{#1}}} %category of countably generated free R-modules

\newcommand{\catomegaspectra}{\operatorname{\Omega-\scr{S}pectra}} %category of omega-spectra

%TOPOLOGY------------------------------------------
\newcommand{\point}{\operatorname{point}} %point 

\newcommand{\Closure}[2]{\operatorname{Closure}_{{#1}}\left({#2} \right)} %Closure

\newcommand{\Int}[1]{\operatorname{Int}\left({#1} \right)} %Set of interior points

\newcommand{\Bd}[1]{\partial {#1}} %Boundary of a set

\newcommand{\sphere}[1]{\D{S}^{{#1}}} %sphere

\newcommand{\CP}[1]{\D{C}\D{P}^{{#1}}} %complex projective spaces

\newcommand{\RP}[1]{\D{R}\D{P}^{{#1}}} %real projective spaces

\newcommand{\sk}[2]{{#1}^{({#2})}} %n-skeleton of a CW complex

\newcommand{\simplex}[1]{\left[ {#1} \right]} %Simplex

\newcommand{\commutativesquare}[8]{\begin{tikzpicture}
  \node (A) {{#1}}; 
  \node (B) [right=of A] {{#3}}; 
  \node (C) [below=of A] {{#4}}; 
  \node (D) [right=of C, below=of B] {{#6}};
  \draw[->] (A)-- node[above] {\tiny {#2}} (B); 
  \draw[->] (A)-- node [left] {\tiny {#7}} (C); 
  \draw[->] (B)-- node [right] {\tiny {#8}} (D); 
  \draw[->] (C)-- node [below] {\tiny {#5}} (D); 
\end{tikzpicture}}%Commutative square

\newcommand{\gtori}[1]{\left( \D{T}^2 \right)^{\vee {#1}}}%wedge sum of g tori

\newcommand{\cupprod}[2]{{#1}\smile {#2}}%cup product

\newcommand{\capprod}[2]{{#1} \frown  {#2}}%cap product

\newcommand{\Map}[3]{\operatorname{Map}_{{#3}} \left( {#1}, {#2} \right)}%Mapping space

\newcommand{\Loop}[1]{\Omega {#1}}%loop space

\newcommand{\Suspen}[1]{\Sigma {#1}}%suspension over a space

\newcommand{\Face}[2]{d_{{#1}}^{{#2}}}%face map of simplicial space

\newcommand{\Degen}[2]{s_{{#1}}^{{#2}}}%degeneracy map of simplicial space

\newcommand{\gsimplex}[1]{\Abs{\Delta^{#1}}}%geomtric n-simplex

\newcommand{\myprod}[3]{{#1} \times_{{#3}}{{#2}}}%fibre product

\newcommand{\Fr}[1]{\operatorname{Fr}\left({#1}\right)}%Frame bundle

\newcommand{\Grass}[3]{\operatorname{Gr}_{{#2}}\left( \D{{#3}}^{{#1}}\right)}%Grassmannian

\newcommand{\Stiefelm}[3]{V_{{#2}}\left( \D{{#3}}^{{#1}}\right)}%Stiefel Manifold

\newcommand{\oStiefelm}[3]{V^o_{{#2}}\left( \D{{#3}}^{{#1}}\right)}%orthonormal Stiefel Manifold

\newcommand{\homotopygrp}[2]{\pi_{{#1}} \left({#2} \right)}%homotopy group

\newcommand{\homotopymap}[2]{\pi \left[ {#1}, {#2} \right]}%homotopy classes of maps

\newcommand{\cohomology}[4]{H^{{#3}}_{\operatorname{{#4}}}\left({#1}\mbox{;} \ {#2} \right)}%cohomology group

\newcommand{\deRham}[2]{H^{{#1}}_{\operatorname{dR}}\left({#2} \right)}%de Rham cohomology group

\newcommand{\Zcohomology}[2]{H^{{#2}}\left({#1}\mbox{;} \ {\D{Z}} \right)}%Integral cohomology group

\newcommand{\EG}[2]{E_{{#2}}{#1}}%universal space

\newcommand{\mydu}[3]{{#1} \sqcup_{{#3}}{{#2}}}%pushout

\newcommand{\normclosure}[1]{\ol{{#1}}^{\norm{\cdot}{}}} %norm closure

\newcommand{\weakclosure}[1]{\ol{{#1}}^{w}} %weak closure

\newcommand{\cone}[1]{\operatorname{cone}\left( {#1} \right)} %cone space

\newcommand{\cylinder}[1]{\operatorname{cyl} \left( {#1} \right)} %cylinder

\newcommand{\cwreplace}[1]{{#1}_{\operatorname{CW}}}%CW-replacement

\newcommand{\hofib}[1]{\operatorname{hofib}\left( {#1} \right)} %homotopy fibre

\newcommand{\hocofib}[1]{\operatorname{hocofib}\left( {#1} \right)} %homotopy cofibre

\newcommand{\CG}[2]{\D{CG}\left( {#1}, {#2} \right)} %complex Grassmannian

\newcommand{\ssphere}[1]{\check{\D{S}}^{{#1}}} %simplicial sphere

\newcommand{\AHSS}[3]{\operatorname{AHSS}\left( {#1} \right)^{{#2}}_{{#3}}} %sophisticated Atiyah-Hirzebruch

%CALCULUS AND ANALYSIS-----------------------------
\newcommand{\norm}[2]{\left\lVert{#1}\right\rVert_{#2}} %norm of a vector

\newcommand{\Dif}[2]{\frac{d{#1}}{d{#2}}} %derivative

\newcommand{\dif}[2]{\frac{\partial {#1}}{\partial {#2}}} %partial derivative

\newcommand{\Interval}[4]{ \left#1 {#2}, {#3} \right#4} %interval

\newcommand{\grad}[1]{\operatorname{grad}\left({#1}\right)} %gradient

\newcommand{\oball}[2]{B \left( {#1}, {#2} \right)}%open ball

\newcommand{\cball}[2]{\ol{B} \left( {#1}, {#2} \right)}%closed ball

\newcommand{\Lp}[2]{L^{{#1}} \left( {#2} \right)}%Lp space

\newcommand{\lp}[2]{\ell^{{#1}} \left( {#2} \right)}%Lp space

\newcommand{\orcom}[1]{{#1}^{\perp}}%orthogonal complement

\newcommand{\myint}[4]{\int_{{#3}}^{{#4}} {#1} \ d{{#2}}}%integration

\newcommand{\normop}[1]{\norm{{#1}}{op}}%operator norm

\newcommand{\normHS}[1]{\norm{{#1}}{\operatorname{HS}}}%Hilbert-Schmidt norm

\newcommand{\supp}[1]{\operatorname{supp}\left( {#1} \right)}%support of function

\newcommand{\Fred}[1]{\operatorname{Fred}\left( {#1} \right)}%Fredholm operators

\newcommand{\ind}[1]{\operatorname{ind}\left( {#1} \right)}%classical index

\newcommand{\Calk}[1]{\operatorname{Calk}\left( {#1} \right)}%Calking algebra

%LIE THEORY----------------------------------------
\newcommand{\Lie}[1]{\mathfrak{{#1}}} %Lie algebra

\newcommand{\commutator}[2]{\left[ {#1}, {#2} \right]}%commutator

%DIFFERENTIAL GEOMETRY----------------------------
\newcommand{\christof}[3]{\Gamma_{{#1} \hspace{0.1em} {#3}}^{\hspace{0.3em {#2}}}} %Christoffel symbol

%NUMBER THEORY------------------------------------
\newcommand{\MOD}[3]{{#1} \equiv {#2} \ \left(\operatorname{mod} \  {#3} \right)}

\newcommand{\zmodp}[1]{\D{Z}/{#1}\D{Z}} %Modulo p integers

\newcommand{\sign}[1]{\operatorname{sign}\left( {#1}\right)}%sign function

\newcommand{\mygcd}[1]{\gcd\left( {#1} \right)} % GCD

%PHYSICS----------------------------------
\newcommand{\quantumev}[1]{\left\langle {#1} \right\rangle}%quantum expected value

%RESEARCH PAPER-----------------------------------
\newcommand{\TR}[2]{\operatorname{TR}^{#1}_{#2}} %equivariant homotopy group

\newcommand{\borelH}[1]{\operatorname{H}^{{\tiny \operatorname{Borel}}}_{#1}} %Borel homology

\newcommand{\simplexcat}[1]{\Delta \downarrow {#1}} %simplex category for a simplicial set X

\newcommand{\GJreal}[1]{\Abs{{#1}}_{\operatorname{GJ}}} %Goerss-Jardine realisation for a simplicial set

\newcommand{\externalprod}[2]{{#1} \widetilde{\times} {#2}} %external product of two bi-simplicial sets

\newcommand{\fullreal}[1]{{\Abs{#1}}_{\operatorname{full}}} %full realisation of a bi-simplicial set

\newcommand{\diagreal}[1]{\Abs{{#1}}_{\operatorname{diag}}} %diagonal realisation of bi-simplicial set

\newcommand{\cofib}[2]{{#1} \rightarrowtail {#2}} %cofibration

\newcommand{\simp}[1]{\operatorname{simp}\left({#1} \right)}%Waldhausen's simp functor

\newcommand{\cofseq}[3]{{#1} \rightarrowtail {#2} \twoheadrightarrow {#3}}%Cofibration sequence

\newcommand{\THH}[2]{\operatorname{THH}\left( {#1}\right)_{{#2}}} %Topological Hochschild Homology

\newcommand{\assem}[1]{\alpha_{{#1}}}%assembly map

\newcommand{\Wh}[2]{\operatorname{Wh}_{{#2}}\left( {#1} \right)}%Whitehead group

\newcommand{\Zariskicohomology}[3]{H^{{#3}}_{\textrm{\tiny Zariski}}\left({#1} \mbox{;} \ {#2} \right)}%Zariski cohomology group

\newcommand{\etalecohomology}[3]{H^{{#3}}_{\textrm{\tiny \'{e}t}}\left({#1} \mbox{;} \ {#2} \right)}%etale cohomology group

\newcommand{\trivialcofib}{\mycal{C} \cap \mycal{W}} %trivial cofibration

\newcommand{\trivialfib}{\mycal{F} \cap \mycal{W}} %trivial fibration

\newcommand{\RamPM}[1]{\dul{P}\left( {#1} \right)}%Ramras' Category of Projective Modules

\newcommand{\Qcon}[1]{Q \left({#1} \right)} %Q-construction

\newcommand{\admmor}[5]{\begin{tikzpicture}
  \node (A) {${#1}$}; 
  \node (B) [right= of A] {${#2}$};
  \node (C) [right= of B] {${#3}$};
  \draw[->>] (B)--node[above] {\small ${#4}$} (A);
  \draw[>->] (B)--node[above] {\small ${#5}$} (C);
\end{tikzpicture}} %morphisms in Q-construction

\newcommand{\Lodayf}[4]{f^{{#1},{#2}}_{{#3},{#4}}}%Loday's f map

\newcommand{\BGL}[1]{BGL \left({#1}\right)} %Classifying space of GL

\newcommand{\BGLp}[1]{\fun{BGL}{{#1}}^{+}} %Plus construction

\newcommand{\Lodaym}[4]{\gamma^{{#1},{#2}}_{{#3},{#4}}}%Loday's multiplication map

\newcommand{\Lodaymh}[4]{\widehat{\gamma}^{{#1},{#2}}_{{#3},{#4}}}%Loday's multiplication map on smash product

\newcommand{\KDL}[1]{K^{\operatorname{DL}}\left( {#1} \right)}%Davis-Luck K-theory spectrum

\newcommand{\Orcat}[1]{\operatorname{Or}\left( {#1} \right)}%orbit category

\newcommand{\actgroupoid}[2]{{#1}\mathsmaller{\int} {#2}}%action groupoid, need \usepackage{relsize}

\newcommand{\twist}[2]{\operatorname{twist}_{{#1},{#2}}} %twist map

\newcommand{\Lodayproda}[2]{ {#1} \ast_{\operatorname{Loday}}{#2}} %Loday product \ast

\newcommand{\Lodayprodb}[2]{ {#1}  \bigstar  {#2}} %Loday product \ast

\newcommand{\KGW}[1]{\mathbb{K}^{\operatorname{GW}}_{{#1}}}%Gersten-Wagoner K-theory spectrum

\newcommand{\Kfree}[1]{\mathbb{K}^{\operatorname{free}}_{{#1}}}%Free K-theory spectrum

\newcommand{\HH}[2]{\operatorname{{\it HH}}_{{#1}} \left( {#2} \right)} %Hochschild homology 

\newcommand{\Stsym}[2]{ \left\{ {#1}, {#2} \right\}_{\mathrm{St}}}%Steinberg Symbol

\newcommand{\Ncyc}[2]{N^{\operatorname{cyc}}_{{#2}} \left( {#1} \right)} %cyclic bar construction

\newcommand{\Lodaya}{\alpha_{\operatorname{\tiny Loday}}} %Loday assembly

\newcommand{\Walda}{\alpha_{\operatorname{\tiny Wald}}} %Waldhausen assembly

\newcommand{\Lodayp}{\gamma_{\operatorname{\tiny Loday}}} %Loday pairing

\newcommand{\Waldp}{\gamma_{\operatorname{\tiny Wald}}} %Waldhausen pairing

\newcommand{\Weibelp}{\gamma_{\operatorname{\tiny Weibel}}} %Weibel pairing

\newcommand{\freep}{\gamma_{\operatorname{\tiny free}}} %pairing for free modules

\newcommand{\WWa}{\alpha_{\operatorname{\tiny WW}}} %Weiss-Williams assembly

\newcommand{\KQ}[1]{\mathbb{K}^{Q}_{{#1}}}% K-theory spectrum in terms of Q-construction (do not confuse with the double Q-construction)

\newcommand{\kgw}[1]{\Bbbk^{\operatorname{gw}}_{{#1}}}%Gersten-Wagoner K-theory spectrum without the K0-factor

\newcommand{\WhG}[2]{\operatorname{Wh}_{{#2}} \left( {#1} \right)}

\newcommand{\KKfree}[1]{K^{\operatorname{free}}_{{#1}}}%Free K-theory space

\newcommand{\KPW}[1]{\mathbb{K}^{\operatorname{PW}}_{{#1}}}%Pedersen-Weibel K-theory spectrum

\newcommand{\Ksmc}[1]{K^{\Box}_{{#1}}}%K-theory space of a symmetric monoidal category

\newcommand{\freea}{\alpha_{\operatorname{\tiny free}}} %free assembly

\newcommand{\Kproj}[1]{\mathbb{K}^{\operatorname{proj}}_{{#1}}}%Idempotent K-theory spectrum

\newcommand{\projp}{\gamma_{\operatorname{\tiny proj}}} %pairing for projective modules

\newcommand{\proja}{\alpha_{\operatorname{\tiny proj}}} %projective assembly

\newcommand{\freestar}{\star_{\operatorname{free}}} %the multiplication map with respect to \freep

\newcommand{\Kahlerdiff}[3]{\Omega^{{#1}}_{\left. {#2} \middle| {#3} \right.}} %Kahler differentials

\newcommand{\naivep}{\gamma_{\operatorname{\tiny naive}}} %naive pairing

\newcommand{\naivestar}{\star_{\operatorname{naive}}} %the multiplication map with respect to \naivep

\newcommand{\naivea}{\alpha_{\operatorname{\tiny naive}}} %naive assembly

%STATISTICS-------------------------------
\newcommand{\Var}[1]{\operatorname{Var}\left( {#1} \right)} % variance

\newcommand{\Cov}[1]{\operatorname{Cov}\left( {#1} \right)} % covariance

\newcommand{\binomdist}[1]{\operatorname{Binomial}\left( {#1}\right)} % Binomial distribution

\newcommand{\negbinomdist}[1]{\operatorname{NegBinomial}\left( {#1} \right)} % Negative Binomial distribution

\newcommand{\normaldist}[1]{\operatorname{Normal}\left( {#1} \right)} % Normal distribution

\newcommand{\poissondist}[1]{\operatorname{Poisson}\left( {#1} \right)} % Poisson distribution

\newcommand{\uniformdist}[1]{\operatorname{Uniform}\left( {#1} \right)} % Poisson distribution

\newcommand{\geometricdist}[1]{\operatorname{Geometric}\left( {#1} \right)} % Geometric distribution

\newcommand{\conditbar}[2]{ \left. {#1} \middle| {#2} \right.} % conditional bar

\newcommand{\gammadist}[1]{\operatorname{Gamma}\left( {#1} \right)} % Gamma distribution

\newcommand{\betadist}[1]{\operatorname{Beta}\left( {#1} \right)} % Beta distribution

\newcommand{\bernoullidist}[1]{\operatorname{Bernoulli}\left( {#1} \right)} % Bernoulli distribution

\newcommand{\Gaussianpdf}[2]{ \frac{1}{\sqrt{2\pi} {#2}} e^{-\frac{1}{2} \left( {#1} \right)^2} } 

\newcommand{\orderstatvar}[2]{{#1}_{\left( {#2} \right)}} % short-hand notation for ordered statistics

%\END{COMMAND}

\makeatletter
\tikzset{join/.code=\tikzset{after node path={%
\ifx\tikzchainprevious\pgfutil@empty\else(\tikzchainprevious)%
edge[every join]#1(\tikzchaincurrent)\fi}}}

\makeatother

%\tikzset{>=stealth',every on chain/.append style={join},
%        every join/.style={->}}

\newlength{\parindentsave}\setlength{\parindentsave}{\parindent}

\everymath{\displaystyle}

\numberwithin{equation}{subsection} 

\let\emptyset\varnothing

\hypersetup{colorlinks,citecolor=blue,linkcolor=blue}

\declaretheorem[numberwithin=section, shaded={rulecolor=black,
rulewidth=0.5pt, bgcolor={rgb}{1,1,1}}]{Theorem}

%\doublespacing

\setcounter{tocdepth}{4}

\begin{document}
\maketitle

\tableofcontents

\newpage
\section{Exercise 5.1}

Let $X_n$ be the number of blind people in a sample of size $n$. Then $X \sim \binomdist{n, 0.01}$. As a result,

\begin{align*}
    0.95 & \leq \fun{P}{\mbox{the sample contains a blind people}} \\
    &= \fun{P}{X > 0} \\
    &= 1 - \fun{P}{X = 0} \\
    &= 1 - 0.99^n,
\end{align*}
or equivalently,

\begin{align*}
    n &\geq  \frac{\fun{\log}{0.05}}{\fun{\log}{0.99}} \\
    &\approx 298.073.
\end{align*}
We need a sample size of at least 299 people.

\newpage
\section{Exercise 5.2}

Let $K$ be the number of years until $X_1$ is exceeded for the first time.

\begin{enumerate}[label=(\alph*),leftmargin=*]
    \item
    
    \begin{align*}
        \fun{P}{K = k} &= \fun{P}{\mbox{$X_{k} > X_1$, $X_i \leq X_1$ for all $1 < i < k$}} \\
        &= \myint{\fun{P}{\mbox{$X_1 = x$, $X_{k} > x$, $X_i \leq x$ for all $1 < i < k$}}}{x}{\D{R}}{} \\
        &= \myint{\fun{P}{\conditbar{\mbox{$X_{k} > x$, $X_i \leq x$ for all $1 < i < k$}}{X_1 = x}} \cdot \fun{P}{X_1 = x}}{x}{\D{R}}{} \\
        &= \myint{\fun{P}{X_k > x} \cdot \prod_{i=2}^{k-1} \fun{P}{X_i \leq x} \cdot \fun{P}{X_1 = x}}{x}{\D{R}}{} \\
        &= \myint{ \left[ 1 - \fun{F}{x}\right] \cdot \prod_{i=2}^{k-1} \fun{F}{x} \cdot \fun{f}{x}}{x}{\D{R}}{} \\
        &= \myint{\fun{F}{x}^{k-2} \fun{f}{x}}{x}{\D{R}}{} - \myint{\fun{F}{x}^{k-1} \fun{f}{x}}{x}{\D{R}}{} \\
        &= \left. \frac{\fun{F}{x}^{k-1}}{k-1} \right\vert_{x=-\infty}^{x = \infty} - \left. \frac{\fun{F}{x}^{k}}{k} \right\vert_{x=-\infty}^{x = \infty} \\
        &= \frac{1}{k-1} - \frac{1}{k} \\
        &= \frac{1}{k(k-1)}
    \end{align*}
    
    \item
    
    \begin{align*}
        ET &= \sum_{k} k \cdot \frac{1}{k(k-1)} \\
        &= \sum_{k} \frac{1}{k-1} \\
        &= \infty
    \end{align*}
\end{enumerate}

\newpage
\section{Exercise 5.3}

In other words,
\[ Y_i \sim \bernoullidist{p}, \]
where $p = \fun{P}{X_i > \mu}$. The probabilities are the same for all $i$ because of the iid assumption on $X_i$'s. This iid assumption, together with the definition of $Y_i$, imply that $Y_i$'s are independent as well. Therefore,

\begin{align*}
    \sum_{i=1}^n Y_i &\sim \sum_{i=1}^n \bernoullidist{p} \\
    &\sim \binomdist{n, p}
\end{align*}

\newpage
\section{Exercise 5.4}

\begin{enumerate}[label=(\alph*),leftmargin=*]
    \item Let $\sigma$ be a permutation on $k$ objects for $k \leq n$. Then
    
    \begin{align*}
        \fun{P}{X_1 = x_{\fun{\sigma}{1}}, \cdots, X_k = x_{\fun{\sigma}{k}}}
        &= \myint{\fun{P}{\conditbar{X_1 = x_{\fun{\sigma}{1}}, \cdots, X_k = x_{\fun{\sigma}{k}}}{P=p}} \cdot \fun{f_P}{p}}{p}{0}{1} \\
        &= \myint{\prod_{i=1}^k \fun{P}{\conditbar{X_i = x_{\fun{\sigma}{i}}}{P=p}} \cdot 1}{p}{0}{1} \\
        & \ \ \ \ \left( \mbox{$\conditbar{X_i}{P}$'s are iid} \right) \\
        &= \myint{\prod_{i=1}^k p^{x_{\fun{\sigma}{i}}} (1-p)^{1-x_{\fun{\sigma}{i}}}}{p}{0}{1} \\
        &= \myint{p^{\displaystyle \sum_{i=1}^k x_{\fun{\sigma}{i}}} (1-p)^{\displaystyle k - \sum_{i=1}^k x_{\fun{\sigma}{i}}}}{p}{0}{1} \\
        &= \myint{p^{\displaystyle \sum_{i=1}^k x_{i}} (1-p)^{\displaystyle k - \sum_{i=1}^k x_{i}}}{p}{0}{1} \\
        &= \fun{P}{X_1 = x_1, \cdots, X_k = x_k} \\
        &= \myint{p^t(1-p)^{k-t}}{p}{0}{1} \\
        & \ \ \ \ \left(\mbox{where $t = \sum_{i=1}^k x_i$} \right) \\
        &= \frac{\fun{\Gamma}{t+1} \cdot \fun{\Gamma}{k-t+1}}{\fun{\Gamma}{k+2}} \\
        &= \frac{t!(k-t)!}{(k+1)!}
    \end{align*}
    
    \item We compute
    
    \begin{align*}
        \prod_{i=1}^n \fun{P}{X_i = x_i} &= \prod_{i=1}^n \myint{\fun{P}{\conditbar{X_i = x_i}{P=p}} \cdot \fun{f_P}{p}}{p}{0}{1} \\
        &= \prod_{i=1}^n \myint{p^{x_i} (1-p)^{1-x_i}}{p}{0}{1},
    \end{align*}
    which is not the same as we computed in part (a) for $k=n$.
\end{enumerate}

\newpage
\section{Exercise 5.5}

Let $Y = \sum_{i=1}^n X_i$, then $Y = n \hat{X}$. Thus

\begin{align*}
    \fun{f_{\hat{X}}}{x} &= \fun{f_Y}{x^{-1}(y)} \cdot \Abs{\Dif{x}{y}} \\
    &= n \cdot \fun{f_Y}{nx}
\end{align*}

\newpage
\section{Exercise 5.6}

The problem statement has a typo: it should be (5.2.9) instead.

\begin{enumerate}[label=(\alph*),leftmargin=*]

\item Let $Z = X-Y$ and $W = X$. Then

\begin{align*}
    \fun{f_{Z,W}}{z,w} &= \fun{f_{X,Y}}{\fun{x}{z,w}, \fun{y}{z,w}} \cdot \left\vert \begin{array}{cc}
         \dif{x}{z} & \dif{x}{w} \\
         \dif{y}{z} & \dif{y}{w}
    \end{array} \right\vert \\
    &= \fun{f_{X,Y}}{w, w-z} \cdot \left\vert \begin{array}{cc}
         0 & 1 \\
         -1 & 1
    \end{array} \right\vert \\
    &= \fun{f_X}{w} \fun{f_Y}{w-z}
\end{align*}
As a result,

\begin{align*}
    \fun{f_Z}{z} &= \myint{\fun{f_{Z,W}}{z,w}}{w}{\mycal{W}}{} \\
    &= \myint{\fun{f_X}{w} \cdot \fun{f_Y}{w-z}}{w}{\mycal{W}}{}.
\end{align*}

\item Let $Z = XY$ and $W = X$. Then

\begin{align*}
    \fun{f_{Z,W}}{z,w} &= \fun{f_{X,Y}}{\fun{x}{z,w}, \fun{y}{z,w}} \cdot \left\vert \begin{array}{cc}
         \dif{x}{z} & \dif{x}{w} \\
         \dif{y}{z} & \dif{y}{w}
    \end{array} \right\vert \\
    &= \fun{f_{X,Y}}{w, \frac{z}{w}} \cdot \Abs{\frac{1}{w}} \\
    &= \Abs{\frac{1}{w}} \fun{f_X}{w} \fun{f_Y}{\frac{z}{w}}.
\end{align*}
Therefore,

\begin{align*}
    \fun{f_Z}{z} &= \myint{\fun{f_{Z,W}}{z,w}}{w}{\mycal{W}}{} \\
    &= \myint{\fun{f_{Z,W}}{z,w}}{w}{\mycal{W}}{} \\
    &= \myint{\Abs{\frac{1}{w}} \fun{f_X}{w} \fun{f_Y}{\frac{z}{w}}}{w}{\mycal{W}}{}
\end{align*}

\item Let $Z = X/Y$ and $W = X$. Then

\begin{align*}
    \fun{f_Z}{z} &= \myint{\fun{f_{Z,W}}{z,w}}{w}{\mycal{W}}{} \\
    &= \myint{\fun{f_{X,Y}}{\fun{x}{z,w}, \fun{y}{z,w}} \cdot \left\vert \begin{array}{cc}
         \dif{x}{z} & \dif{x}{w} \\
         \dif{y}{z} & \dif{y}{w}
    \end{array} \right\vert}{w}{\mycal{W}}{} \\
    &= \myint{\Abs{\frac{w}{z^2}}\fun{f_{X,Y}}{w, \frac{w}{z}}}{w}{\mycal{W}}{} \\
    &= \myint{\Abs{\frac{w}{z^2}} \fun{f_X}{w} \fun{f_Y}{\frac{w}{z}}}{w}{\mycal{W}}{}
\end{align*}
\end{enumerate}

\newpage
\section{Exercise 5.7}

\begin{enumerate}[label=(\alph*),leftmargin=*]
    \item We have
    
    \begin{align*}
        &\frac{1}{1+(w/\sigma)^2} \cdot \frac{1}{1+((z-w)/\tau)^2} \equiv \frac{Aw}{1+(w/\sigma)^2} + \frac{B}{1+(w/\sigma)^2} \\
        & \ \ \ \ \ \ \ \ \ \ \ \ \ \ \ \ \ \ \ \ \ \ \ \ \ \ \ \ \ \ \ \ \ \ \ \ \ \ \ \ \ \ \  - \frac{Cw}{1+((z-w)/\tau)^2} -  \frac{D}{1+((z-w)/\tau)^2} \\
        \iff & 1 \equiv \left[  B \left( 1 + \frac{z^2}{\tau^2} \right) - D \right] + \left[ A \left(1 + \frac{z^2}{\tau^2} \right) - B \left( \frac{2z}{\tau^2} \right) -C \right] w \\
        & \ \ \ \ \left[ A \left(\frac{-2z}{\tau^2} \right) + B \left( \frac{1}{\tau^2} \right) + D \left( \frac{-1}{\sigma^2} \right) \right] w^2 + \left[ A \left(\frac{1}{\tau^2}\right) + C \left( \frac{-1}{\sigma^2} \right) \right] w^3.
    \end{align*}
    Comparing coefficients of $w$ gives the following system
    
    \[ \left\{ \begin{array}{rcl}
         A \left(\frac{1}{\tau^2}\right) + C \left( \frac{-1}{\sigma^2} \right) &= & 0  \\
         A \left(\frac{-2z}{\tau^2} \right) + B \left( \frac{1}{\tau^2} \right) + D \left( \frac{-1}{\sigma^2} \right) &= & 0 \\
         A \left(1 + \frac{z^2}{\tau^2} \right) - B \left( \frac{2z}{\tau^2} \right) -C &= & 0 \\
         B \left( 1 + \frac{z^2}{\tau^2} \right) - D &= & 1
    \end{array} \right. \]
    of equations with unknowns $A$, $B$, $C$, $D$. Solving this gives
    
    \[ \left\{ \begin{array}{rcl}
         A &= & \frac{2z \tau^2}{\mycal{D}} \\
         B &= & \frac{\tau^2 \left( z^2 - \sigma^2 + \tau^2 \right)}{\mycal{D}} \\
         C &= & \frac{2z \sigma^2}{\mycal{D}} \\
         D &= & \frac{\sigma^2 \left( -3z^2 - \sigma^2 + \tau^2 \right)}{\mycal{D}}
    \end{array} \right. \]
    for which
    
    \[ \mycal{D} = (z^2 + \sigma^2)^2 + 2(z-\sigma)(z+\sigma)\tau^2 + \tau^4. \]
\end{enumerate}

\newpage
\section{Exercise 5.8}

\begin{enumerate}[label=(\alph*),leftmargin=*]
    \item The desired result follows from
    
    \begin{align*}
        2n \sum_{i=1}^n \left( X_i - \ol{X} \right)^2 &= 2n \sum_{i=1}^n \left( X_i - \frac{1}{n} \sum_{j=1}^n X_j \right)^2 \\
        &= 2n \sum_{i=1}^n \left[ X_i^2 - \frac{2}{n} X_i \sum_{j=1}^n X_j + \frac{1}{n^2} \left( \sum_{j=1}^n X_j \right)^2 \right] \\
        &= 2n \left[ \sum_{i=1}^n X_i^2 - \frac{2}{n} \sum_{i=1}^n \sum_{j=1}^n X_i X_J + \frac{1}{n} \left( \sum_{j=1}^n X_j \right)^2 \right] \\
        &= 2n \left[ \sum_{i=1}^n X_i^2 - \frac{1}{n} \sum_{i=1}^n \sum_{j=1}^n X_i X_J \right] \\
        &= 2n \sum_{i=1}^n X_i^2 - 2 \sum_{i=1}^n \sum_{j=1}^n X_i X_j \\
        &= \sum_{i=1} \sum_{j=1} \left( X_i - X_j \right)^2.
    \end{align*}
    
    \item We first compute
    
    \begin{align*}
        \fun{E}{S^2} &= \fun{E}{ \frac{1}{2n(n-1)} \sum_{i=1}^n \sum_{j=1}^n \left( X_i - X_j \right)^2} \\
        &= \frac{1}{2n(n-1)} \sum_{i=1}^n \sum_{j=1}^n \fun{E}{X_i - X_j}^2 \\
        &= \frac{1}{2n(n-1)} \sum_{i=1}^n \sum_{j=1}^n \fun{E}{X_i - \theta_1 - X_j + \theta_1}^2 \\
        &= \frac{1}{2n(n-1)} \sum_{i=1}^n \sum_{j=1}^n \fun{E}{X_i - \theta_1}^2 - 2 \fun{E}{\left( X_i - \theta_1 \right) \left( X_j - \theta_1 \right)} + \fun{E}{X_j - \theta_1}^2 \\
        &= \frac{1}{2n(n-1)} \sum_{i \not= j} \fun{E}{X_i - \theta_1}^2 + \fun{E}{X_j - \theta_1}^2 \\
        &= \frac{1}{2n(n-1)} \cdot n(n-1) \cdot 2\theta_2 \\
        &= \theta_2.
    \end{align*}
    Then we get
    
    \begin{align*}
        \fun{E}{S^4} &= \left[ \frac{1}{2n(n-1)} \right]^2 \sum_{i,j,k, \ell=1}^n \fun{E}{(X_i - X_j)^2 (X_k - X_{\ell})^2} \\
        &= \left[ \frac{1}{2n(n-1)} \right]^2 \sum_{i,j,k, \ell=1}^n \fun{E}{(X_i - \theta_1 -  X_j + \theta_1)^2 (X_k - \theta_1 -  X_{\ell} + \theta_1)^2} \\
        \\
        &= \left[ \frac{1}{2n(n-1)} \right]^2 \cdot \left[ \sum_{i,j,k,\ell=1}^n \fun{E}{(X_i - \theta_1)^2 (X_j - \theta_1)^2} + \fun{E}{(X_i - \theta_1)^2 (X_{\ell} - \theta_1)^2}  \right. \\
        & \hspace{4.5cm} + \fun{E}{(X_j - \theta_1)^2 (X_{k} - \theta_1)^2} + \fun{E}{(X_j - \theta_1)^2 (X_{\ell} - \theta_1)^2} \\
        \\
        & \hspace{3cm} - 2 \sum_{i,j,k,\ell = 1}^n \fun{E}{(X_i - \theta_1)^2(X_k-\theta_1)(X_{\ell}-\theta_1)} + \fun{E}{(X_j - \theta_1)^2(X_k-\theta_1)(X_{\ell}-\theta_1)} \\
        & \hspace{4.5cm} + \fun{E}{(X_k - \theta_1)^2(X_i-\theta_1)(X_{j}-\theta_1)} + \fun{E}{(X_{\ell} - \theta_1)^2(X_i-\theta_1)(X_{j}-\theta_1)} \\
        \\
        & \hspace{3cm} \left. + 4 \sum_{i,j,k,\ell=1}^n \fun{E}{(X_i-\theta_1)(X_j-\theta_1)(X_k-\theta_1)(X_{\ell}-\theta_1)} \right] \\
        \\
        &= \left[ \frac{1}{2n(n-1)} \right]^2 \cdot \left[ 4 \sum_{\substack{i \not=j \\
        k \not= \ell \\
        i = k}} E(X_i - \theta_1)^4 + 4 \sum_{\substack{i \not=j \\
        k \not= \ell \\
        i \not= k}} E(X_i - \theta_1)^2(X_j - \theta_1)^2 \right. \\
        \\
        & \hspace{4cm} -0 \\
        \\
        &\hspace{4cm} \left. +8 \sum_{\substack{i \not= j, \\
        k \not= \ell \\
        (i,j) = (k, \ell)}} E(X_i - \theta_1)^2 (X_j - \theta_1)^2  \right] \\
        &= \left[ \frac{1}{2n(n-1)} \right]^2 \cdot \left[4 n(n-1)^2\theta_4 + 4n(n-1)^3 \theta_2^2 + 8n(n-1)\theta_2^2 \right] \\
        &= \frac{1}{n} \theta_4 + \frac{n^2-2n+3}{n(n-1)} \theta_2^2.
    \end{align*}
    Therefore,
    
    \begin{align*}
        \Var{S^2} &= ES^4 - (ES^2)^2 \\
        &= \frac{1}{n} \theta_4 + \frac{n^2-2n+3}{n(n-1)} \theta_2^2 - \theta_2^2 \\
        &= \frac{1}{n} \left(\theta_4 + \frac{n-3}{n-1} \theta_2^2 \right)
    \end{align*}
    
    \item First assume $\theta_1 = 0$. Then
    
    \begin{align*}
        \Cov{\ol{X}, S^2} &= \fun{E}{\ol{X} S^2} - \fun{E}{\ol{X}} \fun{E}{S^2} \\
        &= \fun{E}{\ol{X} S^2} - \theta_1 \fun{E}{S^2} \\
        &= \fun{E}{\ol{X} S^2} \\
        &= \fun{E}{\left( \frac{1}{n} \sum_{i=1}^n X_i \right) \cdot \left(\frac{1}{2n(n-1)} \sum_{i=1}^n \sum_{j=1}^n \left( X_i - X_j \right)^2 \right)} \\
        &= \frac{1}{2n^2(n-1)} \sum_{i,j,k=1}^n \fun{E}{X_i(X_j-X_k)^2} \\
        &= \frac{1}{2n^2(n-1)} \sum_{i,j,k=1}^n \fun{E}{X_i X_j^2} - 2 \fun{E}{X_i X_j X_k} + \fun{E}{X_i X_k^2} \\
        &= \frac{1}{2n^2(n-1)} \left[ n^2 \fun{E}{X_i^3} + n^2(n-1)\fun{E}{X_i X_j^2} \right. \\
        & \hspace{3cm} -2n(n-1)(n-2)\fun{E}{X_i X_j X_k} + 3n(n-1) \fun{E}{X_i X_j^2} + n \fun{E}{X_i^3} \\
        & \hspace{3cm} \left. n^2 \fun{E}{X_i^3} + n^2(n-1) \fun{E}{X_i X_k^2} \right] \\
        &= \frac{1}{2n^2(n-1)} \left[ 2n(n-1) \fun{E}{X_i^3} \right. \\
        & \hspace{3cm} \left. + 2n(n-1)(n-3) \fun{E}{X_i X_j^2} - 2n(n-1)(n-2) \fun{E}{X_i X_j X_k} \right] \\
        &= \frac{1}{2n^2(n-1)} \cdot 2n(n-1) \fun{E}{X_i^3} \\
        &= \frac{\theta_3}{n}.
    \end{align*}
    For general $\theta_1$, we observe that
    
    \begin{align*}
        \Cov{X, Y} &= \fun{E}{(X-EX)(Y-EY)} \\
                   &= \fun{E}{(X-EX-0)(Y-EY)} \\
                   &= \fun{E}{(X-EX-E(X-EX))(Y-EY)} \\
                   &= \Cov{X-EX, Y}.
    \end{align*}
    In other words, by replacing $X_i$ by $X_i - \theta_1$, we return to the case $\theta_1 = 0$. Moreover, this replacement preserves covariance. Therefore, $\Cov{\ol{X}, S^2} = \frac{\theta_3}{n}$, and it vanishes iff $\theta_1$ vanishes.
\end{enumerate}

\newpage
\section{Exercise 5.10}

See \cite[Theorem 5.3.1 on page 218]{Berger-Casella}.

    
    \begin{align*}
        \theta_1 &= EX_i \\
                 &= \mu, \\
        \\
        \theta_2 &= ES^2 & \left(\mbox{Exercise 5.8 (b)} \right)\\
                 &= \fun{E}{\frac{\sigma^2}{n-1} \chi_{n-1}^2} \\
                 &= \sigma^2, \\
        \\
        \theta_3 &= n \Cov{\ol{X}, S^2} & \left(\mbox{Exercise 5.8 (c)} \right)\\
        &= 0, \\
        \\
        \theta_4 &= n \Var{S^2} + \frac{n-3}{n-1}\theta_2^2 & \left(\mbox{Exercise 5.8 (b)} \right)\\
        &= n \Var{\frac{\sigma^2}{n-1} \chi_{n-1}^2} + \frac{n-3}{n-1}\theta_2^2 \\
        &= \frac{2n \sigma^4}{n-1} + \frac{(n-3)\sigma^4}{n-1} \\
        &= 3 \sigma^4.
    \end{align*}
    
    
\newpage
\section{Exercise 5.12}

Firstly, we have

\begin{align*}
    \fun{E}{\Abs{\frac{1}{n} \sum_{i=1}^n X_i}} &= \fun{E}{\Abs{\ol{X}}} \\
    &= \fun{E}{\Abs{\normaldist{0, \frac{1}{n}}}} \\
    & \hspace{0.3cm} \left( \mbox{\cite[Theorem 5.3.1 on page 218]{Berger-Casella}} \right) \\
    &= \myint{\Abs{x} \cdot \frac{n}{\sqrt{2\pi}} e^{-\frac{1}{2} \left( nx \right)^2}}{x}{-\infty}{\infty} \\
    &= \myint{x \cdot \frac{n}{\sqrt{2\pi}} e^{-\frac{1}{2} \left( nx \right)^2}}{x}{0}{\infty} - \myint{x \cdot \frac{n}{\sqrt{2\pi}} e^{-\frac{1}{2} \left( nx \right)^2}}{x}{-\infty}{0} \\
    &= \frac{1}{\sqrt{2\pi}n} + \frac{1}{\sqrt{2\pi}n} \\
    &= \frac{\sqrt{\frac{2}{\pi}}}{n}.
\end{align*}
Secondly, we have

\begin{align*}
    \fun{E}{\frac{1}{n} \sum_{i=1}^n \Abs{X_i}} &= \frac{1}{n} \sum_{i=1}^n \fun{E}{\Abs{X_i}} \\
    &= \fun{E}{\Abs{X_i}} \\
    &= \fun{E}{\Abs{\normaldist{0,1}}} \\
    &= \sqrt{\frac{2}{\pi}}
\end{align*}

\newpage
\section{Exercise 5.24}

In other words, $X \sim \uniformdist{0, \theta}$. Hence, it has cdf

\[ \fun{F_{X}}{x} = \frac{x}{\theta}. \]
\cite[Theorem 5.4.6 page 230]{Berger-Casella} then gives us the joint distribution

\begin{align}
    \fun{f_{X_{(1)}, X_{(n)}}}{u, v} &= \frac{n!}{(n-2)!} \fun{f_X}{u} \fun{f_X}{v} \left[ \fun{F_X}{v} - \fun{F_X}{u} \right]^{n-2} \nonumber \\
    &= \frac{n!}{(n-2)!} \frac{(v-u)^{n-2}}{\theta^n}.
\end{align}

Now, let $Z = \frac{X_{(1)}}{X_{(n)}}$ and $W = X_{(n)}$. Then

\begin{align*}
    \fun{f_{Z,W}}{z,w} &= \fun{f_{X_{(1)}, X_{(n)}}}{x(z,w), y(z,w)} \cdot \Abs{J} \\
    &= \fun{f_{X_{(1)}, X_{(n)}}}{wz, w} \cdot \left\vert \begin{array}{cc}
         w & z  \\
         0 & 1
    \end{array} \right\vert \\
    &= \frac{n!}{(n-2)!} \frac{w^{n-1} (1-z)^{n-2}}{\theta^n}.
\end{align*}
Since the variables can be separated in the joint distribution, $Z$, $W$ are independent.

\newpage
\section{Exercise 5.25}

From \cite[page 230]{Berger-Casella}, the joint distribution of all the order statistics is given by

\begin{align}
    \fun{f_{\orderstatvar{X}{1}, \cdots, \orderstatvar{X}{1}}}{x_1 < \cdots < x_n} &= n! \prod_{i=1}^n \fun{f_X}{x_i} \nonumber \\
    &= \frac{a^n n!}{\theta^{an}} \left(\prod_{i=1}^n x_i \right)^{a-1}.
\end{align}
Define the new variables

\begin{align*}
    Y_i &= \left\{ \begin{array}{cl}
         \frac{\orderstatvar{X}{i}}{\orderstatvar{X}{i+1}} & \mbox{if $i< n$},  \\
         \orderstatvar{X}{n}& \mbox{if $i=n$}. 
    \end{array} \right.
\end{align*}
We then compute the joint distribution

\begin{align*}
    \fun{f_{Y_1, \cdots, Y_{n}}}{y_1, \cdots, y_{n}} &= \fun{f_{\orderstatvar{X}{1}, \cdots, \orderstatvar{X}{n}}}{\fun{x_1}{y_1, \cdots, y_n}, \cdots, \fun{x_n}{y_1, \cdots, y_n}} \cdot \Abs{J} \\
    &= \fun{f_{\orderstatvar{X}{1}, \cdots, \orderstatvar{X}{n}}}{\prod_{k=1}^n y_k, \cdots, \prod_{k=i}^n y_k, \cdots, y_n} \cdot \left( \prod_{i=1}^{n-1} \prod_{j=i+1}^n y_j \right) \\
    &= \left(\frac{a^n n!}{\theta^{an}} \prod_{i=1}^n y_i^{i(a-1)}\right) \left( \prod_{i=2}^n y_{i}^{i-1} \right) \\
    &= \frac{a^n n!}{\theta^{an}} \prod_{i=1}^n y_i^{ia-1}.
\end{align*}
This proves the independency.

\newpage
\section{Exercise 5.27}

\begin{enumerate}[label=(\alph*),leftmargin=*]
    \item By \cite[Theorem 5.4.4 on page 229]{Berger-Casella} and Theorem 5.4.6 on page 230, loc. cit., we have

\begin{align*}
    \fun{f_{\conditbar{i}{j}}}{\conditbar{u}{v}} &= \frac{\fun{f_{\orderstatvar{X}{i}, \orderstatvar{X}{j}}}{u,v}}{\fun{f_{\orderstatvar{X}{j}}}{v}} \\
    &= \frac{\frac{n! \fun{f_X}{u} \fun{f_X}{v} \left[ \fun{F_X}{u} \right]^{i-1} \left[ \fun{F_X}{v} - \fun{F_X}{u} \right]^{j-1-i} \left[ 1 - \fun{F_X}{v} \right]^{n-j}}{(i-1)!(j-1-i)!(n-j)!}}{\frac{n! \fun{f_X}{v} \left[ \fun{F_X}{v} \right]^{j-1} \left[ 1 - \fun{F_X}{v} \right]^{n-j}}{(j-1)!(n-j)!}} \\
    &= \frac{(j-1)! \fun{f_X}{u} \left[ \fun{F_X}{u} \right]^{i-1} \left[ \fun{F_X}{v} - \fun{F_X}{u} \right]^{j-1-i}}{(i-1)!(j-1-i)! \left[ \fun{F_X}{v} \right]^{j-1}}
\end{align*}

\item $\fun{f_{\conditbar{V}{R}}}{\conditbar{v}{r}} = \frac{1}{a-r}$.
\end{enumerate}

\newpage
\nocite{*}
\printbibliography

\end{document}