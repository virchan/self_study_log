\documentclass[12pt,letterpaper,reqno]{amsart}


\usepackage{lipsum} %separate the paragraphs of the dummy text into TEX-paragraphs.

\usepackage{tikz} %commutative diagrams

\usepackage{setspace} %Provides support for setting the spacing between lines in a document.

\usepackage{amsmath,amssymb} %provides miscellaneous enhancements for improving the information structure and printed output of documents containing mathematical formulas. e.g., \DeclareMathOperator, \sin, \lim

\usepackage{enumitem} %additional options for \enumerate

\usepackage{subfig} %provides support for the manipulation and reference of small or ‘sub’ figures and tables within a single figure or table environment.

\usepackage{framed} %box, shaded, put a left line around a region.

\usepackage{etoolbox} %provides LATEX frontends to some of the new primitives provided by e-TEX as well as some generic tools which are not strictly related to e-TEX but match the profile of this package.

\usepackage{bm} %Access bold symbols in maths mode

\usepackage{mdframed} %Framed environments that can split at page boundaries

\usepackage{mathrsfs} %Support for using RSFS fonts in maths

\usepackage{centernot} %Centred \not command

\usepackage{thmtools} %Extensions to theorem environments

\usepackage{thm-restate} %Restate Theorem

\usepackage{fancyvrb} %Sophisticated verbatim text

\usepackage{dsfont} %?

\usepackage{bbm} %Blackboard variants of Computer Modern fonts.

\usepackage{breqn} %Automatic line breaking of displayed equations

\usepackage{adjustbox} %The package provides several macros to adjust boxed content.

\usepackage{tabularx} %The package defines an environment tabularx, an extension of tabular which has an additional column designator, X, which creates a paragraph-like column whose width automatically expands so that the declared width of the environment is filled. (Two X columns together share out the available space between them, and so on.)

\usepackage{calligra} %Calligraphic font

\usepackage[
    backend=bibtex, 
    natbib=true, 
    bibstyle=verbose, citestyle=verbose,    % bibstyle extensively modifed below
    doi=true, url=true,                     % excluded from citations below
    citecounter=true, citetracker=true,
    block=space, 
    backref=false, backrefstyle=two,
    abbreviate=false,
    isbn=true, maxbibnames=4, maxcitenames=4,
    style=alphabetic
]{biblatex} %Adding Reference to your paper.

\usepackage[margin=1in]{geometry} %edit margin of your document

\usepackage[makeindex]{imakeidx} %making index

\usepackage{empheq} %The package provides a visual markup extension to amsmath.

\usepackage{longtable} %allow longtable

\usepackage[mathscr]{eucal}%change mathscr

\usepackage{xpatch} %fix skipbelow in mdframe

\usepackage{standalone}  %need this to include the tikz picture created in different files.

\usepackage{wasysym}

\usepackage{bashful}

\usepackage{hyperref,cleveref} %Extensive support for hypertext in LATEX (ALWAYS LOAD LAST!) %this loads the packages


\makeindex

\author{Virgil Chan}
\title{Casella-Berger \\ Statistical Inference Solution: \\ Chapter 4}
\date{August 9, 2022}


\usetikzlibrary{patterns,positioning,arrows,chains,matrix,positioning,scopes} %options for tikzpicture

\addbibresource{bibliography.bib} %put your reference here


\makeatletter
\patchcmd{\@maketitle}
  {\ifx\@empty\@dedicatory}
  {\ifx\@empty\@date \else {\vskip3ex \centering\footnotesize\@date\par\vskip1ex}\fi
   \ifx\@empty\@dedicatory}
  {}{}
\patchcmd{\@adminfootnotes}
  {\ifx\@empty\@date\else \@footnotetext{\@setdate}\fi}
  {}{}{}
\makeatother

\makeatletter
\xpatchcmd{\endmdframed}
  {\aftergroup\endmdf@trivlist\color@endgroup}
  {\endmdf@trivlist\color@endgroup\@doendpe}
  {}{}
\makeatother
%\topsep
\newtheoremstyle{break}
  {\topsep}% Space above
  {\topsep}% Space below
  {\it}% Body font
  {}% Indent amount
  {\bfseries}% Theorem head font
  {}% Punctuation after theorem head
  {\topsep}% Space after theorem head, ' ', or \newline
  {\thmname{#1}\thmnumber{ #2} \thmnote{(#3)}}% Theorem head spec (can be left empty, meaning `normal')

\mdfdefinestyle{box}{
     linecolor=white,
    skipabove=\topsep,
    skipbelow=\topsep,
    innerbottommargin=\topsep,
}  %

\theoremstyle{break}
\newmdtheoremenv[style=box]{theorem}{Theorem}[section]
\newmdtheoremenv[style=box]{lemma}[theorem]{Lemma}
\newmdtheoremenv[style=box]{proposition}[theorem]{Proposition}
\newmdtheoremenv[style=box]{corollary}[theorem]{Corollary}
\newmdtheoremenv[style=box]{definition}[theorem]{Definition}
\newmdtheoremenv[style=box]{addendum}[theorem]{Addendum}
\newmdtheoremenv[style=box]{conjecture}[theorem]{Conjecture}
\newmdtheoremenv[style=box]{question}[theorem]{Question}
\newmdtheoremenv[style=box]{condit}[theorem]{Condition}

\makeatletter
\def\th@plain{%
  \thm@notefont{}% same as heading font
  \normalfont % body font
}
\def\th@definition{%
  \thm@notefont{}% same as heading font
  \normalfont % body font
}
\makeatother

\newtheoremstyle{exampstyle}
{\topsep} % Space above
{\topsep} % Space below
{} % Body font
{} % Indent amount
{\bfseries} % Theorem head font
{} % Punctuation after theorem head
{\topsep} % Space after theorem head
{\thmname{#1}\thmnumber{ #2} \thmnote{(#3)}} % Theorem head spec (can be left empty, meaning `normal')

\theoremstyle{exampstyle}
\newtheorem{example}[theorem]{Example}
\newtheorem{remark}[theorem]{Remark}


\pdfpagewidth 8.5in
\pdfpageheight 11in

    
%\BEGIN{COMMAND}

%SHORT HAND NOTATION-------------------------------
\newcommand{\D}[1]{\mathbb{#1}} %short hand mathbb command

\newcommand{\Dm}[1]{\mathbbm{#1}} %mathbb for numbers

\newcommand{\mycal}[1]{\mathcal{{#1}}} %short hand mathcal

\newcommand{\scr}[1]{\mathscr{{#1}}} %short hand mathscr

\newcommand{\ve}{\varepsilon} %short hand epsilon

\newcommand\numberthis{\addtocounter{equation}{1}\tag{\theequation}}%label equation in \align*

\newcommand{\dund}[1]{\underline{\underline{{#1}}}}%Double underline

\newcommand{\Frechet}{Fr\'{e}chet}%Frechet space

\newcommand{\ol}[1]{\overline{{#1}}}%overline

\newcommand{\ul}[1]{\underline{{#1}}}%underline

\newcommand{\dul}[1]{\ul{\ul{{#1}}}}%double underline

\newcommand{\Cech}{\v{C}ech} %Cech

\newcommand{\Nother}{N\"{o}ther} %Nother

\newcommand{\Notherian}{N\"{o}ther} %Notherian

\newcommand{\wt}[1]{\widetilde{{#1}}} %short hand for \widetilde

\newcommand{\fun}[2]{{#1}\! \left( {#2} \right)}%short hand for function

\newcommand{\on}[1]{\operatorname{{#1}}} %short hand for \operatorname

\newcommand{\Kp}[1]{K^+_{{#1}}} %short hand for plus-construction

\newcommand{\Knp}[1]{K_{{#1}}} %short hand for before plus-construction

\newcommand{\btimes}{\boxtimes} %short hand for box tensor

\newcommand{\bplus}{\boxplus} %short hand for box sum

\newcommand{\myline}{\par\noindent\rule{\textwidth}{0.4pt}} %horizontal line

\newcommand{\wh}[1]{\widehat{{#1}}} %short hand for \widetilde

\newcommand{\myref}[2]{\hyperref[{#2}]{{#1}~\ref{#2}}} %short hand for \hyperref

\newcommand{\myeqref}[2]{\hyperref[{#2}]{{#1}~\eqref{#2}}} %short hand for \hyperref for equations

%SET THEORY----------------------------------------
\newcommand{\SET}[2]{\left\{ {#1} \ \middle| \ {#2} \right\}} %short hand notation for set

\newcommand{\SETT}[1]{\left\{ {#1} \right\}} %short hand notation for set without constraint

\newcommand{\floor}[1]{\left \lfloor{#1}\right \rfloor } %floor function

\newcommand{\Rng}[1]{\operatorname{Rng}\left(#1\right)} %range of a function

\newcommand{\id}[1]{\operatorname{id}_{{#1}}} %identity map

\newcommand\restr[2]{{% we make the whole thing an ordinary symbol
  \left.\kern-\nulldelimiterspace % automatically resize the bar with \right
  #1 % the function
  \vphantom{\big|} % pretend it's a little taller at normal size
  \right|_{#2} % this is the delimiter
  }} %restriction of a function

\newcommand{\Abs}[1]{\left\lvert{#1}\right\rvert} %absolute value

\newcommand{\nonnegint}{\D{N} \cup \left\{ 0 \right\}} %Non-negative integers

\newcommand{\Image}[1]{\operatorname{im}\left( {#1} \right)} %Image of a map

\newcommand{\Proj}[1]{\operatorname{proj}_{{#1}}}%projection map

\newcommand{\ev}[1]{\operatorname{ev}_{{#1}}}%Evaluation map

\makeatletter
\newbox\xrat@below
\newbox\xrat@above
\newcommand{\xrightarrowtail}[2][]{%
  \setbox\xrat@below=\hbox{\ensuremath{\scriptstyle #1}}%
  \setbox\xrat@above=\hbox{\ensuremath{\scriptstyle #2}}%
  \pgfmathsetlengthmacro{\xrat@len}{max(\wd\xrat@below,\wd\xrat@above)+.6em}%
  \mathrel{\tikz [>->,baseline=-.75ex]
                 \draw (0,0) -- node[below=-2pt] {\box\xrat@below}
                                node[above=-2pt] {\box\xrat@above}
                       (\xrat@len,0) ;}}
\makeatother %xrightarrowtail

\makeatletter
\newbox\xrat@below
\newbox\xrat@above
\newcommand{\xtwoheadrightarrow}[2][]{%
  \setbox\xrat@below=\hbox{\ensuremath{\scriptstyle #1}}%
  \setbox\xrat@above=\hbox{\ensuremath{\scriptstyle #2}}%
  \pgfmathsetlengthmacro{\xrat@len}{max(\wd\xrat@below,\wd\xrat@above)+.6em}%
  \mathrel{\tikz [->>,baseline=-.75ex]
                 \draw (0,0) -- node[below=-2pt] {\box\xrat@below}
                                node[above=-2pt] {\box\xrat@above}
                       (\xrat@len,0) ;}}
\makeatother %xtwoheadrightarrow

\newcommand{\dom}[1]{\operatorname{dom} \left( {#1} \right)} %Domain of a function

\newcommand{\codom}[1]{\operatorname{codom} \left( {#1} \right)} %Codomain of a function

\newcommand{\sseq}{\subseteq} %short hand for \subseteq
  
%ALGEBRA-------------------------------------------
\newcommand{\Aut}[2]{\operatorname{Aut}_{{#2}}\left({#1} \right)} %automorphism group

\newcommand{\Hom}[3]{\operatorname{Hom}_{{#3}}\left({#1},{#2}\right)} %Homomorphism group

\newcommand{\Char}[1]{\operatorname{char}\left( {#1} \right)} %characteristic of a ring/field

\newcommand{\Rad}[1]{\operatorname{Rad}\left({#1}\right)} %radical

\newcommand{\Nil}[1]{\operatorname{Nil}\left({#1}\right)} %nilradical

\newcommand{\Tors}[1]{\operatorname{Tors}\left({#1} \right)} %Torsion 

\newcommand{\Gal}[2]{\operatorname{Gal}\left({#1}/{#2}\right)} %Galois group

\newcommand{\pid}[1]{\left\langle {#1} \right\rangle} %Principal Ideal

\newcommand{\polynomial}[5]{\sum_{{#1}={#2}}^{{#3}} {#4}_{{#1}} {{#5}}^{{#1}}} %Polynomial

\newcommand{\mydeg}[1]{\deg \left( {#1} \right)} %Degree

\newcommand{\GL}[1]{GL \left({#1}\right)} %General linear group

\newcommand{\abelian}[1]{{#1}^{ab}} %Abelianisation

\newcommand{\Orb}[2]{\operatorname{Orb}_{{#1}}\left( {#2} \right)} %orbit of group action

\newcommand{\Stab}[2]{\operatorname{Stab}_{{#1}}\left( {#2} \right)} %stabiliser of group action

\newcommand{\GROUP}[2]{\left\langle {#1} \ \middle| \ {#2} \right\rangle} %group presentation

\newcommand{\ideal}{\trianglelefteq}%normal subgroup symbol

\newcommand{\properideal}{\triangleleft}%proper normal subgroup symbol

\newcommand{\groupaction}[3]{{#1} \rotatebox[origin=c]{180}{$\circlearrowright$}_{{#3}} {#2}}%group action

\newcommand{\SL}[1]{SL \left({#1}\right)} %special linear group

\newcommand{\iso}[1]{\operatorname{iso}\left( {#1} \right)}%Group of isomorphisms

\newcommand{\Gr}[1]{\operatorname{Gr}\left( {#1} \right)}%associated graded group

\newcommand{\bigboxplus}{
  \mathop{
    \vphantom{\bigoplus} 
    \mathchoice
      {\vcenter{\hbox{\resizebox{\widthof{$\displaystyle\bigoplus$}}{!}{$\boxplus$}}}}
      {\vcenter{\hbox{\resizebox{\widthof{$\bigoplus$}}{!}{$\boxplus$}}}}
      {\vcenter{\hbox{\resizebox{\widthof{$\scriptstyle\oplus$}}{!}{$\boxplus$}}}}
      {\vcenter{\hbox{\resizebox{\widthof{$\scriptscriptstyle\oplus$}}{!}{$\boxplus$}}}}
  }\displaylimits 
} %internal direct sum

\newcommand{\bideg}[1]{\text{bi-deg}\left( {#1} \right)}%bi-degree

\newcommand{\Stgroup}[1]{\operatorname{St}\left({#1} \right)} %Steinberg group

\newcommand{\groupp}[2]{\left\langle {#1} \ \middle| \ {#2} \right\rangle} %group presentation

%LINEAR ALGEBRA------------------------------------
\newcommand{\rank}[1]{\operatorname{{rank}}\left({#1}\right)} %rank

\newcommand{\diag}[1]{\operatorname{diag}\left({#1} \right)} %diagonal matrix

\newcommand{\Ker}[1]{\operatorname{ker}\left({#1} \right)} %Kernel of a map

\newcommand{\Tr}[1]{\operatorname{trace}\left({#1}\right)} %trace

\newcommand{\mat}[4]{\left[ \begin{array}{cc}
{#1} & {#2}  \\
{#3} & {#4}
\end{array} \right]} %2x2 matrix

\newcommand{\Range}[1]{\operatorname{Rng}\left( {#1} \right)} %Range of a map

\newcommand{\Span}[1]{\operatorname{span}\left( {#1} \right)} %span of a set

\newcommand{\Null}[1]{\operatorname{null}\left({#1}\right)} %nullity

\newcommand{\innerprod}[3]{\left\langle {#1}, {#2} \right\rangle_{{#3}}} %Inner product

\newcommand{\colvec}[2]{\left[ \begin{array}{c}
{#1} \\
\vdots \\
{#2}
\end{array} \right]}%column vector

\newcommand{\Dim}[2]{\dim_{{#2}} \left( {#1} \right)}%dimension

\newcommand{\Det}[1]{\det \left( {#1} \right)}%determinant

\newcommand{\coDim}[2]{\operatorname{codim}_{{#2}} \left( {#1} \right)}%codimension

%COMMUTATIVE ALGEBRA AND ALGEBRAIC GEOMETRY--------
\newcommand{\td}[1]{\operatorname{tr.deg.}\left( {#1} \right)} %transcendence degree

\newcommand{\Spec}[1]{\operatorname{Spec}\left({#1}\right)} %spectrum of a ring

\newcommand{\height}{\operatorname{height}} %height

\newcommand{\mSpec}[1]{\operatorname{mSpec}\left( {#1}\right)} %maximal ideals of a ring

\newcommand{\length}[1]{\operatorname{length}\left({#1} \right)} %length

\newcommand{\Pic}[1]{\operatorname{Pic}\left({#1}\right)}  %Piccard group

\newcommand{\Idem}[1]{\operatorname{Idem}\left( {#1} \right)}%Idempotent

\newcommand{\sections}[1]{\Gamma\left({#1} \right)}%sections of sheaf/bundle

\newcommand{\kerpre}[1]{\operatorname{ker}_{\operatorname{presheaf}}\left( {#1} \right)}%presheaf kernel

\newcommand{\cokerpre}[1]{\operatorname{coker}_{\operatorname{presheaf}}\left( {#1} \right)}%presheaf cokernel

\newcommand{\RES}[2]{\operatorname{res}_{{#1},{#2}}}%restriction of a presheaf

%Complex Analysis----------------------------------

\newcommand{\expi}[2]{e^{\frac{{#1}}{{#2}}}} %complex exponential with imaginary fraction exponent

\newcommand{\Res}[2]{\operatorname{Res} \left[ {#1},{#2} \right]} %residue

\newcommand{\myRe}[1]{\operatorname{Re}\left({#1} \right)} %Real part

\newcommand{\myIm}[1]{\operatorname{Im}\left({#1} \right)} %Imaginary part

\newcommand{\winding}[2]{\fun{\operatorname{Ind}_{{#1}}}{{#2}}}%winding number of a curve

\newcommand{\polydisc}[1]{\D{D}^{{#1}}_{\operatorname{poly}}}%poly-disc

\newcommand{\distbound}[1]{\partial^{\operatorname{dist}}\polydisc{{#1}}}%distinguished boundary

\newcommand{\Blaschke}[1]{\fun{\operatorname{Blaschke}}{{#1}}}%Blaschke product

%HOMOLOGICAL ALGEBRA AND CATEGORY THEORY-----------
\newcommand{\Mor}[2]{\operatorname{mor}_{{#2}}\left({#1}\right)} %Morphism class

\newcommand{\cat}[1]{\normalfont{\mathbf{#1}}} %notation for category

\newcommand{\Tor}[3]{\operatorname{Tor}_{{#3}}\left( {#1}, {#2} \right)} %Tor functor
\newcommand{\catset}{\operatorname{\scr{S}ets}} %category of sets

\newcommand{\Ring}{\operatorname{\scr{R}ings}} %category of rings

\newcommand{\Obj}[1]{\operatorname{obj}\left({#1} \right)} %Object class of category

\newcommand{\Ext}[2]{\operatorname{Ext}\left( {#1}, {#2} \right)} %Ext-functor

\newcommand{\coker}[1]{\operatorname{{coker}}\left( {#1} \right)} %cokernel

\makeatletter
\newcommand{\colim@}[2]{%
  \vtop{\m@th\ialign{##\cr
    \hfil$#1\operator@font colim$\hfil\cr
    \noalign{\nointerlineskip\kern1.5\ex@}#2\cr
    \noalign{\nointerlineskip\kern-\ex@}\cr}}%
}
\newcommand{\colim}[1]{%
  \mathop{\mathpalette\colim@{}}_{{#1}}
} %colimit

\renewcommand{\varprojlim}{%
  \mathop{\mathpalette\varlim@{\leftarrowfill@\scriptscriptstyle}}\nmlimits@
}
\renewcommand{\varinjlim}{%
  \mathop{\mathpalette\varlim@{\rightarrowfill@\scriptscriptstyle}}\nmlimits@
} %limit

\newcommand{\hocolim@}[2]{%
  \vtop{\m@th\ialign{##\cr
    \hfil$#1\operator@font hocolim$\hfil\cr
    \noalign{\nointerlineskip\kern1.5\ex@}#2\cr
    \noalign{\nointerlineskip\kern-\ex@}\cr}}%
}
\newcommand{\hocolim}[1]{%
  \mathop{\mathpalette\hocolim@{}}_{{#1}}
}%homotopy colimit

\newcommand{\holim@}[2]{%
  \vtop{\m@th\ialign{##\cr
    \hfil$#1\operator@font holim$\hfil\cr
    \noalign{\nointerlineskip\kern1.5\ex@}#2\cr
    \noalign{\nointerlineskip\kern-\ex@}\cr}}%
}
\newcommand{\holim}[1]{%
  \mathop{\mathpalette\holim@{}}_{{#1}}
}%homotopy limit

\newcommand{\tensor}[3]{{#1} \otimes_{{#3}} {#2}}%tensor product

\newcommand{\Eq}[1]{\operatorname{Eq} \left( {#1} \right)}%Equaliser

\newcommand{\coEq}[1]{\operatorname{coEq} \left( {#1} \right)}%Coequaliser

\newcommand{\catfgProj}[1]{\operatorname{\scr{P}roj}^{\operatorname{fg}}_{{#1}}} %category of finitely generated projective R-modules

\newcommand{\catmodule}[1]{\operatorname{\scr{M}odule}_{{#1}}} %category of R-modules

\newcommand{\catspace}{\operatorname{\scr{S}paces}} %category of spaces

\newcommand{\catspectra}{\operatorname{\scr{S}pectra}} %category of spectra

\newcommand{\catab}{\operatorname{\scr{A}belian}} %category of Abelian groups

\newcommand{\catringoid}{\operatorname{\scr{R}ingoids}} %category of ringoids

\newcommand{\catgroup}{\operatorname{\scr{G}roups}} %category of groups

\newcommand{\catfgfree}[1]{\operatorname{\scr{F}ree}^{\operatorname{fg}}_{{#1}}} %category of finitely generated free R-modules

\newcommand{\hormor}[1]{\operatorname{hor-mor}\left({#1}\right)} % horizontal morphism class

\newcommand{\vermor}[1]{\operatorname{ver-mor}\left({#1}\right)} % vertical morphism class

\newcommand{\bimor}[1]{\operatorname{bimor}\left({#1}\right)} % bi-morphism class

\newcommand{\catiso}[1]{\operatorname{iso}\left({#1}\right)}%category of isomorphisms

\newcommand{\SiS}[1]{\scr{S}_{{#1}}} %S-inverse-S-construction of category of isomorphisms

\newcommand{\CMA}[2]{\scr{C}_{{#2}}\left( {#1} \right)} %Pedersen-Weibel category

\newcommand{\catfinset}{\operatorname{\scr{F}in\scr{S}et}} %category of finite set

\newcommand{\catSMC}{\operatorname{\scr{S}ym\scr{M}on\scr{C}at}} %category of symmetric monoidal categories

\newcommand{\PiP}[1]{{\scr{P}_{{#1}}}} %S-inverse-S-construction of category of isomorphisms in the idempotent completion

\newcommand{\catoofree}[1]{\operatorname{\scr{F}ree}^{\mathbb{N}}_{{#1}}} %category of countably generated free R-modules

\newcommand{\catomegaspectra}{\operatorname{\Omega-\scr{S}pectra}} %category of omega-spectra

%TOPOLOGY------------------------------------------
\newcommand{\point}{\operatorname{point}} %point 

\newcommand{\Closure}[2]{\operatorname{Closure}_{{#1}}\left({#2} \right)} %Closure

\newcommand{\Int}[1]{\operatorname{Int}\left({#1} \right)} %Set of interior points

\newcommand{\Bd}[1]{\partial {#1}} %Boundary of a set

\newcommand{\sphere}[1]{\D{S}^{{#1}}} %sphere

\newcommand{\CP}[1]{\D{C}\D{P}^{{#1}}} %complex projective spaces

\newcommand{\RP}[1]{\D{R}\D{P}^{{#1}}} %real projective spaces

\newcommand{\sk}[2]{{#1}^{({#2})}} %n-skeleton of a CW complex

\newcommand{\simplex}[1]{\left[ {#1} \right]} %Simplex

\newcommand{\commutativesquare}[8]{\begin{tikzpicture}
  \node (A) {{#1}}; 
  \node (B) [right=of A] {{#3}}; 
  \node (C) [below=of A] {{#4}}; 
  \node (D) [right=of C, below=of B] {{#6}};
  \draw[->] (A)-- node[above] {\tiny {#2}} (B); 
  \draw[->] (A)-- node [left] {\tiny {#7}} (C); 
  \draw[->] (B)-- node [right] {\tiny {#8}} (D); 
  \draw[->] (C)-- node [below] {\tiny {#5}} (D); 
\end{tikzpicture}}%Commutative square

\newcommand{\gtori}[1]{\left( \D{T}^2 \right)^{\vee {#1}}}%wedge sum of g tori

\newcommand{\cupprod}[2]{{#1}\smile {#2}}%cup product

\newcommand{\capprod}[2]{{#1} \frown  {#2}}%cap product

\newcommand{\Map}[3]{\operatorname{Map}_{{#3}} \left( {#1}, {#2} \right)}%Mapping space

\newcommand{\Loop}[1]{\Omega {#1}}%loop space

\newcommand{\Suspen}[1]{\Sigma {#1}}%suspension over a space

\newcommand{\Face}[2]{d_{{#1}}^{{#2}}}%face map of simplicial space

\newcommand{\Degen}[2]{s_{{#1}}^{{#2}}}%degeneracy map of simplicial space

\newcommand{\gsimplex}[1]{\Abs{\Delta^{#1}}}%geomtric n-simplex

\newcommand{\myprod}[3]{{#1} \times_{{#3}}{{#2}}}%fibre product

\newcommand{\Fr}[1]{\operatorname{Fr}\left({#1}\right)}%Frame bundle

\newcommand{\Grass}[3]{\operatorname{Gr}_{{#2}}\left( \D{{#3}}^{{#1}}\right)}%Grassmannian

\newcommand{\Stiefelm}[3]{V_{{#2}}\left( \D{{#3}}^{{#1}}\right)}%Stiefel Manifold

\newcommand{\oStiefelm}[3]{V^o_{{#2}}\left( \D{{#3}}^{{#1}}\right)}%orthonormal Stiefel Manifold

\newcommand{\homotopygrp}[2]{\pi_{{#1}} \left({#2} \right)}%homotopy group

\newcommand{\homotopymap}[2]{\pi \left[ {#1}, {#2} \right]}%homotopy classes of maps

\newcommand{\cohomology}[4]{H^{{#3}}_{\operatorname{{#4}}}\left({#1}\mbox{;} \ {#2} \right)}%cohomology group

\newcommand{\deRham}[2]{H^{{#1}}_{\operatorname{dR}}\left({#2} \right)}%de Rham cohomology group

\newcommand{\Zcohomology}[2]{H^{{#2}}\left({#1}\mbox{;} \ {\D{Z}} \right)}%Integral cohomology group

\newcommand{\EG}[2]{E_{{#2}}{#1}}%universal space

\newcommand{\mydu}[3]{{#1} \sqcup_{{#3}}{{#2}}}%pushout

\newcommand{\normclosure}[1]{\ol{{#1}}^{\norm{\cdot}{}}} %norm closure

\newcommand{\weakclosure}[1]{\ol{{#1}}^{w}} %weak closure

\newcommand{\cone}[1]{\operatorname{cone}\left( {#1} \right)} %cone space

\newcommand{\cylinder}[1]{\operatorname{cyl} \left( {#1} \right)} %cylinder

\newcommand{\cwreplace}[1]{{#1}_{\operatorname{CW}}}%CW-replacement

\newcommand{\hofib}[1]{\operatorname{hofib}\left( {#1} \right)} %homotopy fibre

\newcommand{\hocofib}[1]{\operatorname{hocofib}\left( {#1} \right)} %homotopy cofibre

\newcommand{\CG}[2]{\D{CG}\left( {#1}, {#2} \right)} %complex Grassmannian

\newcommand{\ssphere}[1]{\check{\D{S}}^{{#1}}} %simplicial sphere

\newcommand{\AHSS}[3]{\operatorname{AHSS}\left( {#1} \right)^{{#2}}_{{#3}}} %sophisticated Atiyah-Hirzebruch

%CALCULUS AND ANALYSIS-----------------------------
\newcommand{\norm}[2]{\left\lVert{#1}\right\rVert_{#2}} %norm of a vector

\newcommand{\Dif}[2]{\frac{d{#1}}{d{#2}}} %derivative

\newcommand{\dif}[2]{\frac{\partial {#1}}{\partial {#2}}} %partial derivative

\newcommand{\Interval}[4]{ \left#1 {#2}, {#3} \right#4} %interval

\newcommand{\grad}[1]{\operatorname{grad}\left({#1}\right)} %gradient

\newcommand{\oball}[2]{B \left( {#1}, {#2} \right)}%open ball

\newcommand{\cball}[2]{\ol{B} \left( {#1}, {#2} \right)}%closed ball

\newcommand{\Lp}[2]{L^{{#1}} \left( {#2} \right)}%Lp space

\newcommand{\lp}[2]{\ell^{{#1}} \left( {#2} \right)}%Lp space

\newcommand{\orcom}[1]{{#1}^{\perp}}%orthogonal complement

\newcommand{\myint}[4]{\int_{{#3}}^{{#4}} {#1} \ d{{#2}}}%integration

\newcommand{\normop}[1]{\norm{{#1}}{op}}%operator norm

\newcommand{\normHS}[1]{\norm{{#1}}{\operatorname{HS}}}%Hilbert-Schmidt norm

\newcommand{\supp}[1]{\operatorname{supp}\left( {#1} \right)}%support of function

\newcommand{\Fred}[1]{\operatorname{Fred}\left( {#1} \right)}%Fredholm operators

\newcommand{\ind}[1]{\operatorname{ind}\left( {#1} \right)}%classical index

\newcommand{\Calk}[1]{\operatorname{Calk}\left( {#1} \right)}%Calking algebra

%LIE THEORY----------------------------------------
\newcommand{\Lie}[1]{\mathfrak{{#1}}} %Lie algebra

\newcommand{\commutator}[2]{\left[ {#1}, {#2} \right]}%commutator

%DIFFERENTIAL GEOMETRY----------------------------
\newcommand{\christof}[3]{\Gamma_{{#1} \hspace{0.1em} {#3}}^{\hspace{0.3em {#2}}}} %Christoffel symbol

%NUMBER THEORY------------------------------------
\newcommand{\MOD}[3]{{#1} \equiv {#2} \ \left(\operatorname{mod} \  {#3} \right)}

\newcommand{\zmodp}[1]{\D{Z}/{#1}\D{Z}} %Modulo p integers

\newcommand{\sign}[1]{\operatorname{sign}\left( {#1}\right)}%sign function

\newcommand{\mygcd}[1]{\gcd\left( {#1} \right)} % GCD

%PHYSICS----------------------------------
\newcommand{\quantumev}[1]{\left\langle {#1} \right\rangle}%quantum expected value

%RESEARCH PAPER-----------------------------------
\newcommand{\TR}[2]{\operatorname{TR}^{#1}_{#2}} %equivariant homotopy group

\newcommand{\borelH}[1]{\operatorname{H}^{{\tiny \operatorname{Borel}}}_{#1}} %Borel homology

\newcommand{\simplexcat}[1]{\Delta \downarrow {#1}} %simplex category for a simplicial set X

\newcommand{\GJreal}[1]{\Abs{{#1}}_{\operatorname{GJ}}} %Goerss-Jardine realisation for a simplicial set

\newcommand{\externalprod}[2]{{#1} \widetilde{\times} {#2}} %external product of two bi-simplicial sets

\newcommand{\fullreal}[1]{{\Abs{#1}}_{\operatorname{full}}} %full realisation of a bi-simplicial set

\newcommand{\diagreal}[1]{\Abs{{#1}}_{\operatorname{diag}}} %diagonal realisation of bi-simplicial set

\newcommand{\cofib}[2]{{#1} \rightarrowtail {#2}} %cofibration

\newcommand{\simp}[1]{\operatorname{simp}\left({#1} \right)}%Waldhausen's simp functor

\newcommand{\cofseq}[3]{{#1} \rightarrowtail {#2} \twoheadrightarrow {#3}}%Cofibration sequence

\newcommand{\THH}[2]{\operatorname{THH}\left( {#1}\right)_{{#2}}} %Topological Hochschild Homology

\newcommand{\assem}[1]{\alpha_{{#1}}}%assembly map

\newcommand{\Wh}[2]{\operatorname{Wh}_{{#2}}\left( {#1} \right)}%Whitehead group

\newcommand{\Zariskicohomology}[3]{H^{{#3}}_{\textrm{\tiny Zariski}}\left({#1} \mbox{;} \ {#2} \right)}%Zariski cohomology group

\newcommand{\etalecohomology}[3]{H^{{#3}}_{\textrm{\tiny \'{e}t}}\left({#1} \mbox{;} \ {#2} \right)}%etale cohomology group

\newcommand{\trivialcofib}{\mycal{C} \cap \mycal{W}} %trivial cofibration

\newcommand{\trivialfib}{\mycal{F} \cap \mycal{W}} %trivial fibration

\newcommand{\RamPM}[1]{\dul{P}\left( {#1} \right)}%Ramras' Category of Projective Modules

\newcommand{\Qcon}[1]{Q \left({#1} \right)} %Q-construction

\newcommand{\admmor}[5]{\begin{tikzpicture}
  \node (A) {${#1}$}; 
  \node (B) [right= of A] {${#2}$};
  \node (C) [right= of B] {${#3}$};
  \draw[->>] (B)--node[above] {\small ${#4}$} (A);
  \draw[>->] (B)--node[above] {\small ${#5}$} (C);
\end{tikzpicture}} %morphisms in Q-construction

\newcommand{\Lodayf}[4]{f^{{#1},{#2}}_{{#3},{#4}}}%Loday's f map

\newcommand{\BGL}[1]{BGL \left({#1}\right)} %Classifying space of GL

\newcommand{\BGLp}[1]{\fun{BGL}{{#1}}^{+}} %Plus construction

\newcommand{\Lodaym}[4]{\gamma^{{#1},{#2}}_{{#3},{#4}}}%Loday's multiplication map

\newcommand{\Lodaymh}[4]{\widehat{\gamma}^{{#1},{#2}}_{{#3},{#4}}}%Loday's multiplication map on smash product

\newcommand{\KDL}[1]{K^{\operatorname{DL}}\left( {#1} \right)}%Davis-Luck K-theory spectrum

\newcommand{\Orcat}[1]{\operatorname{Or}\left( {#1} \right)}%orbit category

\newcommand{\actgroupoid}[2]{{#1}\mathsmaller{\int} {#2}}%action groupoid, need \usepackage{relsize}

\newcommand{\twist}[2]{\operatorname{twist}_{{#1},{#2}}} %twist map

\newcommand{\Lodayproda}[2]{ {#1} \ast_{\operatorname{Loday}}{#2}} %Loday product \ast

\newcommand{\Lodayprodb}[2]{ {#1}  \bigstar  {#2}} %Loday product \ast

\newcommand{\KGW}[1]{\mathbb{K}^{\operatorname{GW}}_{{#1}}}%Gersten-Wagoner K-theory spectrum

\newcommand{\Kfree}[1]{\mathbb{K}^{\operatorname{free}}_{{#1}}}%Free K-theory spectrum

\newcommand{\HH}[2]{\operatorname{{\it HH}}_{{#1}} \left( {#2} \right)} %Hochschild homology 

\newcommand{\Stsym}[2]{ \left\{ {#1}, {#2} \right\}_{\mathrm{St}}}%Steinberg Symbol

\newcommand{\Ncyc}[2]{N^{\operatorname{cyc}}_{{#2}} \left( {#1} \right)} %cyclic bar construction

\newcommand{\Lodaya}{\alpha_{\operatorname{\tiny Loday}}} %Loday assembly

\newcommand{\Walda}{\alpha_{\operatorname{\tiny Wald}}} %Waldhausen assembly

\newcommand{\Lodayp}{\gamma_{\operatorname{\tiny Loday}}} %Loday pairing

\newcommand{\Waldp}{\gamma_{\operatorname{\tiny Wald}}} %Waldhausen pairing

\newcommand{\Weibelp}{\gamma_{\operatorname{\tiny Weibel}}} %Weibel pairing

\newcommand{\freep}{\gamma_{\operatorname{\tiny free}}} %pairing for free modules

\newcommand{\WWa}{\alpha_{\operatorname{\tiny WW}}} %Weiss-Williams assembly

\newcommand{\KQ}[1]{\mathbb{K}^{Q}_{{#1}}}% K-theory spectrum in terms of Q-construction (do not confuse with the double Q-construction)

\newcommand{\kgw}[1]{\Bbbk^{\operatorname{gw}}_{{#1}}}%Gersten-Wagoner K-theory spectrum without the K0-factor

\newcommand{\WhG}[2]{\operatorname{Wh}_{{#2}} \left( {#1} \right)}

\newcommand{\KKfree}[1]{K^{\operatorname{free}}_{{#1}}}%Free K-theory space

\newcommand{\KPW}[1]{\mathbb{K}^{\operatorname{PW}}_{{#1}}}%Pedersen-Weibel K-theory spectrum

\newcommand{\Ksmc}[1]{K^{\Box}_{{#1}}}%K-theory space of a symmetric monoidal category

\newcommand{\freea}{\alpha_{\operatorname{\tiny free}}} %free assembly

\newcommand{\Kproj}[1]{\mathbb{K}^{\operatorname{proj}}_{{#1}}}%Idempotent K-theory spectrum

\newcommand{\projp}{\gamma_{\operatorname{\tiny proj}}} %pairing for projective modules

\newcommand{\proja}{\alpha_{\operatorname{\tiny proj}}} %projective assembly

\newcommand{\freestar}{\star_{\operatorname{free}}} %the multiplication map with respect to \freep

\newcommand{\Kahlerdiff}[3]{\Omega^{{#1}}_{\left. {#2} \middle| {#3} \right.}} %Kahler differentials

\newcommand{\naivep}{\gamma_{\operatorname{\tiny naive}}} %naive pairing

\newcommand{\naivestar}{\star_{\operatorname{naive}}} %the multiplication map with respect to \naivep

\newcommand{\naivea}{\alpha_{\operatorname{\tiny naive}}} %naive assembly

%STATISTICS-------------------------------
\newcommand{\Var}[1]{\operatorname{Var}\left( {#1} \right)} % variance

\newcommand{\Cov}[1]{\operatorname{Cov}\left( {#1} \right)} % covariance

\newcommand{\binomdist}[1]{\operatorname{Binomial}\left( {#1}\right)} % Binomial distribution

\newcommand{\negbinomdist}[1]{\operatorname{NegBinomial}\left( {#1} \right)} % Negative Binomial distribution

\newcommand{\normaldist}[1]{\operatorname{Normal}\left( {#1} \right)} % Normal distribution

\newcommand{\poissondist}[1]{\operatorname{Poisson}\left( {#1} \right)} % Poisson distribution

\newcommand{\uniformdist}[1]{\operatorname{Uniform}\left( {#1} \right)} % Poisson distribution

\newcommand{\geometricdist}[1]{\operatorname{Geometric}\left( {#1} \right)} % Geometric distribution

\newcommand{\conditbar}[2]{ \left. {#1} \middle| {#2} \right.} % conditional bar

\newcommand{\gammadist}[1]{\operatorname{Gamma}\left( {#1} \right)} % Gamma distribution

\newcommand{\betadist}[1]{\operatorname{Beta}\left( {#1} \right)} % Beta distribution

\newcommand{\bernoullidist}[1]{\operatorname{Bernoulli}\left( {#1} \right)} % Bernoulli distribution

\newcommand{\Gaussianpdf}[2]{ \frac{1}{\sqrt{2\pi} {#2}} e^{-\frac{1}{2} \left( {#1} \right)^2} } 

\newcommand{\orderstatvar}[2]{{#1}_{\left( {#2} \right)}} % short-hand notation for ordered statistics

%\END{COMMAND}

\makeatletter
\tikzset{join/.code=\tikzset{after node path={%
\ifx\tikzchainprevious\pgfutil@empty\else(\tikzchainprevious)%
edge[every join]#1(\tikzchaincurrent)\fi}}}

\makeatother

%\tikzset{>=stealth',every on chain/.append style={join},
%        every join/.style={->}}

\newlength{\parindentsave}\setlength{\parindentsave}{\parindent}

\everymath{\displaystyle}

\numberwithin{equation}{subsection} 

\let\emptyset\varnothing

\hypersetup{colorlinks,citecolor=blue,linkcolor=blue}

\declaretheorem[numberwithin=section, shaded={rulecolor=black,
rulewidth=0.5pt, bgcolor={rgb}{1,1,1}}]{Theorem}

%\doublespacing

\setcounter{tocdepth}{4}

\begin{document}
\maketitle

\tableofcontents

\newpage
\section{Exercise 4.1}

\begin{enumerate}[label=(\alph*),leftmargin=*]
    \item We want to know the probability for $(X,Y)$ to land inside the circle
    
    \[ X^2 + Y^2 = 1. \]
    This circle has area $\pi$, so the probability is $\frac{\pi}{4}$.
    
    \item We want to know the probability for $(X,Y)$ to land below the line
    
    \[ 2X-Y = 0. \]
    This line divides the square into two uniform trapeziums. One of them has vertices $\left( \pm \frac{1}{2}, \pm 1 \right)$, and has area $2$. Therefore, the probability is $\frac{1}{2}.$
    
    \item The region $\Abs{X+Y} < 2$ contains the square, so the probability is 1.
\end{enumerate}

\newpage
\section{Exercise 4.4}

\begin{enumerate}[label=(\alph*),leftmargin=*]
    \item 
    
    \begin{align*}
        1 &= \myint{\myint{C(x+2y)}{x}{0}{2}}{y}{0}{1} \\
          &= 4C
    \end{align*}
    So $C = \frac{1}{4}$.
    
    \item 
    
    \begin{align*}
        \fun{f_X}{x} &= \myint{\frac{x+2y}{4}}{y}{0}{1} \\
                     &= \frac{x+1}{4}
    \end{align*}
    on $\mycal{X} = (0,2)$.
    
    \item
    
    \begin{align*}
        \fun{F_{XY}}{x,y} &= \fun{P}{X \leq x, Y \leq y} \\
        &= \left\{ \begin{array}{cl}
             0 & \mbox{if $x \leq 0$ or $y \leq 0$} \\
             1 & \mbox{if $x \geq 2$, $y \geq 1$} \\
             \myint{\myint{f(u,v)}{u}{0}{x}}{v}{0}{y} & \mbox{if else}
        \end{array} \right.
    \end{align*}
    The if else case requires some work.
    
    \begin{align*}
         \myint{\myint{f(u,v)}{u}{0}{x}}{v}{0}{y}
         &= \left\{ \begin{array}{cl}
             \myint{\myint{f(u,v)}{u}{0}{x}}{v}{0}{y}  & \mbox{if $0 < x < 2$, $0 < y < 1$} \\
             \\
             \myint{\myint{f(u,v)}{u}{0}{x}}{v}{0}{1}  & \mbox{if $0 < x < 2$, $y \geq 1$} \\
              \\
             \myint{\myint{f(u,v)}{u}{0}{2}}{v}{0}{y}  & \mbox{if $0 < x < 2$, $y \geq 1$} \\
         \end{array} \right. \\
         \\
         &= \left\{ \begin{array}{cl}
             \frac{xy(x+2y)}{8}  & \mbox{if $0 < x < 2$, $0 < y < 1$} \\
             \\
             \frac{x(x+2)}{8}  & \mbox{if $0 < x < 2$, $y \geq 1$} \\
              \\
             \frac{y(1+y)}{2}   & \mbox{if $x \geq 2$, $0 < y < 1$} \\
         \end{array} \right. 
    \end{align*}
    
    \item Let $z = g(x) = \frac{9}{(x+1)^2}$, then $g^{-1}(z) = \frac{3}{\sqrt{z}}-1$, and $\Dif{}{z} g^{-1}(z) = \frac{3}{-2 z^{\frac{3}{2}}}$. Therefore,
    
    \begin{align*}
        f_Z(z) &= f_X(g^{-1}(z)) \cdot \Abs{\Dif{}{z} g^{-1}(z)} \\
               &= \frac{9}{8z^2}
    \end{align*}
\end{enumerate}

\newpage
\section{Exercise 4.5}

\begin{enumerate}[label=(\alph*),leftmargin=*]
    \item 
    
    \begin{align*}
        \fun{P}{X > \sqrt{Y}} &= \myint{\myint{x+y}{x}{\sqrt{y}}{1}}{y}{0}{1} \\
        &= \frac{7}{20}
    \end{align*}
    
    \item
    
    \begin{align*}
        \fun{P}{X^2 < Y < X} &= \myint{\myint{2x}{y}{x^2}{x}}{x}{0}{1} \\
        &= \frac{1}{6}
    \end{align*}
\end{enumerate}

\newpage
\section{Exercise 4.6}

Let $X$ (resp. $Y$) be the arrival time of $A$ (resp. $B$), so that $X \sim \uniformdist{[0,1]} \sim Y$.

Let $T$ be the waiting time. Then

\[ T = \max \SETT{Y-X, 0}. \]
Therefore,

\begin{align*}
    \fun{P}{T < t} &= \fun{P}{Y-X < t, Y \geq X} + \fun{P}{Y < X}.
\end{align*}
The first summand represents the area inside the square $[0,1] \times [0,1]$, bounded between the lines $y = x+t$ and $y = x$. The second summand represents the area of half of the square. Thus,

\begin{align*}
    \fun{P}{T < t} &= \fun{P}{Y-X < t, Y \geq X} + \fun{P}{Y < X} \\
    &= \myint{t}{x}{0}{1-t} + \myint{1-x}{x}{1-t}{1} + \frac{1}{2} \\
    &= -\frac{t^2}{2} + t + \frac{1}{2}
\end{align*}

\newpage
\section{Exercise 4.7}

We represent the period from 8 AM to 9 AM by the closed interval $[0,1]$. Then $X \sim \uniformdist{ \left[ 0, \frac{1}{2} \right]}$, and $Y \sim \uniformdist{ \left[ \frac{2}{3}, \frac{5}{6} \right]}$. The arrival time is given by $X + Y$, with

\[ f_{X+Y}(x,y) = 12. \]
Therefore

\begin{align*}
    \fun{P}{X + Y < 1} &= \myint{f_{X+Y}(x,y)}{A}{R}{} \\
    &= 12 \left( \mbox{area of $R$} \right).
\end{align*}
The region $R$ is bounded by the functions:

\[ \left\{ \begin{array}{ccl}
     x+ y &=& 1  \\
     y &=& \frac{1}{2} \\
     y &=& \frac{5}{6} \\
     x &=& 0
\end{array} \right. \]
It is a trapezium with vertices $\left( 0, \frac{5}{6} \right), \left( \frac{1}{6}, \frac{5}{6} \right), \left( 0, \frac{2}{3} \right), \left( \frac{1}{3}, \frac{2}{3} \right)$, and has area $\frac{1}{24}$.

Therefore,

\[ \fun{P}{X+Y < 1} = 12 \cdot \frac{1}{24} = \frac{1}{2}. \]

\newpage
\section{Exercise 4.9}

\begin{align*}
    \fun{P}{a \leq X \leq b, c \leq Y \leq d} &= \fun{P}{X \leq b , c \leq Y \leq d} - \fun{P}{X \leq a, c \leq Y \leq d} \\
    &= \left[ \fun{P}{X \leq b, Y \leq d} - \fun{P}{X \leq b, Y \leq c} \right] - \\
    & \ \ \ \ \left[ \fun{P}{X \leq a, Y \leq d} - \fun{P}{X \leq a, Y \leq c} \right] \\
    &= \fun{F_{X,Y}}{b,d} - \fun{F_{X,Y}}{b,c} - \fun{F_{X,Y}}{a,d} + \fun{F_{X,Y}}{a,c} \\
    &= \left[ \fun{F_X}{b} - \fun{F_X}{a} \right] \fun{F_Y}{d} - \left[ \fun{F_X}{b} - \fun{F_X}{a} \right] \fun{F_Y}{c} \\
    &= \left[ \fun{F_X}{b} - \fun{F_X}{a} \right] \left[ \fun{F_Y}{d} - \fun{F_Y}{c} \right] \\
    &= \fun{P}{a \leq X \leq b} \fun{P}{c \leq Y \leq d}
\end{align*}

\newpage
\section{Exercise 4.10}

\begin{enumerate}[label=(\alph*),leftmargin=*]
    \item The marginal pdfs are given by
    
    \begin{align*}
        \fun{f_X}{1} &= \frac{1}{4} & \fun{f_Y}{2} &= \frac{1}{3} \\
        \fun{f_X}{2} &= \frac{1}{2} & \fun{f_Y}{3} &= \frac{1}{3} \\
        \fun{f_X}{3} &= \frac{1}{4} & \fun{f_Y}{4} &= \frac{1}{3}
    \end{align*}
    
    We see that
    
    \begin{align*}
        \fun{P}{X = 2, Y = 3} &= 0 \\
                              &\not= \frac{1}{2} \cdot \frac{1}{3} \\
                              &= \fun{f_X}{2} \fun{f_Y}{3}
    \end{align*}
    Therefore, they are dependent.
    
    \item Let $U = X$, $V = Y$, and the pair $(U,V)$ has distribution
    
    \[ \fun{f_{U,V}}{u,v} = \fun{f_U}{u} \fun{f_V}{v}. \]
\end{enumerate}

\newpage
\section{Exercise 4.11}

Both $V$ and $V$ follow negative binomial distribution:

\begin{align*}
    U &\sim \negbinomdist{1, p}, \\
    V &\sim \negbinomdist{2, p}.
\end{align*}
In particular,

\begin{align*}
    \fun{P}{V = k} = p \cdot \fun{P}{U = k-1}.
\end{align*}
This shows they are dependent.

\newpage
\section{Exercise 4.12}

Without loss of generality, say the stick is given by the interval $(0,1)$. Let $X$, $Y$ be points chosen from $(0,1)$. Then $X \sim \uniformdist{(0,1)} \sim Y$, and

\[ f_{X,Y}(x,y) = 1 \]
on $(0,1) \times (0,1)$.

By symmetry, we may assume $y > x$ first. The points $x$, $y$ divide $(0,1)$ into three pieces of length $x$, $y-x$, $1-y$ respectively. A triangle can be formed if and only if they satisfy the triangle inequality:

\[ \left\{ \begin{array}{cl}
     x+(y-x) &\geq 1-y \\
     x+(1-y) &\geq y-x \\
     (y-x)+(1-y) &\geq x
\end{array} \right. \]
or equivalently,
\[ \left\{ \begin{array}{cl}
     y &\geq \frac{1}{2} \\
     y-x &\leq \frac{1}{2} \\
     x &\leq \frac{1}{2}
\end{array} \right. \]
This is the triangle given by

\[ \left\{ \begin{array}{l}
     0 \leq x \leq \frac{1}{2}, \\
     \frac{1}{2} \leq y \leq x + \frac{1}{2}
\end{array} \right. \]
Combining with the case $x>y$, the required probability is

\begin{align*}
    2 \myint{\myint{f_{X,Y}(x,y)}{y}{\frac{1}{2}}{x + \frac{1}{2}}}{x}{0}{\frac{1}{2}}
    &= 2 \cdot \left(\mbox{area of the triangle} \right) \\
    &= \frac{1}{4}.
\end{align*}

\newpage
\section{Exercise 4.14}
\label{Exercise 4.14}

Since $X$ and $Y$ are independent, the joint distribution is given by

\[ \fun{f_{X,Y}}{x,y} = \frac{1}{2\pi} e^{-\frac{(x^2+y^2)}{2}} \]

\begin{enumerate}[label=(\alph*),leftmargin=*]
    \item 
    
    \begin{align*}
        \fun{P}{X^2+Y^2 < 1} &= \myint{\fun{f_{X,Y}}{x,y}}{A}{x^2+y^2<1}{} \\
        &= \frac{1}{2\pi} \cdot \myint{\myint{r e^{-\frac{r^2}{2}}}{r}{0}{1}}{\theta}{0}{2\pi} \\
        &= 1- e^{-\frac{1}{2}}
    \end{align*}
    
    \item Let $Y = X^2$, then
    
    \begin{align*}
        \fun{f_Y}{y} &= \fun{f_X}{\sqrt{y}} + \fun{f_X}{-\sqrt{y}} \\
        &= \frac{1}{\sqrt{2\pi y}} e^{- \frac{y}{2}} \\
    \end{align*}
    which is the pdf for $\chi_1^2$. Therefore,
    
    \begin{align*}
        \fun{P}{X^2 < 1} &= \myint{f_Y(y)}{y}{0}{1} \\
        &\approx 0.682689
    \end{align*}
\end{enumerate}

\newpage
\section{Exercise 4.15}

Let $U = X + Y$, and $V = X$. Then $U \sim \poissondist{\theta + \lambda}$; and $U$, $V$ are independent by \cite[Theorem 4.3.2 on page 158]{Berger-Casella}.

We repeat the same computation as in \cite[Example 4.3.1 on page 157]{Berger-Casella} to find the joint pdf

\[ \fun{f}{v,u} = \frac{\lambda^{u-v} \theta^v e^{-(\theta + \lambda)}}{(u-v)!v!} \]
for $(U,V)$. Therefore, the conditional distribution is given by

\begin{align*}
    \fun{f}{\conditbar{v}{u}} &= \frac{\fun{f}{v,u}}{\fun{f}{u}} \\
    &= \frac{\lambda^{u-v} \theta^v e^{-(\theta + \lambda)}}{(u-v)!v!} \cdot \frac{u!}{(\theta + \lambda)^u e^{-(\theta + \lambda)}} \\
    &= \binom{u}{v} \left( \frac{\theta}{\theta + \lambda} \right)^v \left( \frac{\lambda}{\theta + \lambda} \right)^{u-v} \\
    &\sim \binomdist{u, \frac{\theta}{\theta + \lambda}}.
\end{align*}
Likewise, $\conditbar{Y}{X+Y} \sim \binomdist{u, \frac{\lambda}{\theta + \lambda}}$.

\newpage
\section{Exercise 4.16}

Write $X \sim \geometricdist{p} \sim Y$.

\begin{enumerate}[label=(\alph*),leftmargin=*]
    \item The joint distribution is given by
    
    \begin{align*}
        \fun{f}{u,v} &= \fun{P}{U = u, \ V = v} \\
        &= \fun{P}{\fun{\min}{X, Y} = u, \ X-Y = v} \\
        &= \left\{ \begin{array}{cl}
            \fun{P}{Y=u, \ X = v+u} & \mbox{if $v \geq 0$}  \\
            \fun{P}{X=u, \ Y = u-v} & \mbox{if $v < 0$} 
        \end{array} \right. \\
        &= \left\{ \begin{array}{cl}
            (1-p)^{2u+v-2}p^2 & \mbox{if $v \geq 0$}  \\
            (1-p)^{2u-v-2}p^2 & \mbox{if $v < 0$} 
        \end{array} \right. \\
        &= (1-p)^{2u + \Abs{v}-2} p^2 \\
        &= \underbrace{\left[ (1-p)^{2u-1}p \right]}_{g(u)} \underbrace{\left[ (1-p)^{\Abs{v}-1}p \right]}_{h(v)}.
    \end{align*}
    \cite[Lemma 4.2.7 on page 153]{Berger-Casella} then says $U$, $V$ are independent.
    
    \item We begin by noting $Z$ takes values in $\D{Q}$. Therefore, we represent all possible values of $Z$ by fractions $\frac{r}{s}$ with $\mygcd{r,s} = 1$. We then compute
    
    \begin{align*}
        \fun{P}{Z = \frac{r}{s}}
        &= \fun{P}{\frac{X}{X+Y} = \frac{r}{s}} \\
        &= \sum_{n=1}^{\infty} \fun{P}{X = nr, \ X+Y = ns} \\
        &= \sum_{n=1}^{\infty} \fun{P}{X = nr, \ Y = n(s-r)} \\
        &= \sum_{n=1}^{\infty} (1-p)^{ns-2}p^2 \\
        &= \frac{(1-p)^{s-2}p^2}{1-(1-p)^{s-2}}
    \end{align*}
    
    \item
    
    \begin{align*}
        \fun{P}{X = u, \ X+Y = v} &= \fun{P}{X = u, \ Y = v-u} \\
        &= (1-p)^{v-2} p^2
    \end{align*}
\end{enumerate}

\newpage
\section{Exercise 4.17}

\begin{enumerate}[label=(\alph*),leftmargin=*]
    \item 
    
    \begin{align*}
        \fun{P}{Y = y} &= \fun{P}{y \leq X < y + 1} \\
        &= \myint{e^{-x}}{x}{y}{y+1} \\
        &= (1-e^{-1}) \left( e^{-1} \right)^y \\
        &\sim \geometricdist{e^{-1}}
    \end{align*}
    
    \item Let $Z = X-4$. We compute the cdf first.
    
    \begin{align*}
        \fun{P}{\conditbar{Z \leq z}{Y \geq 5}} &= \frac{\fun{P}{Z \leq z, \ Y \geq 5}}{\fun{P}{Y \geq 5}} \\
        &= \frac{\fun{P}{Z \leq z, \ X \geq 4}}{\fun{P}{X \geq 4}} \\
        &= \frac{\fun{P}{4 \leq X \leq z + 4}}{\fun{P}{X \geq 4}} \\
        &= \frac{e^{-4}-e^{-4-z}}{e^{-4}} \\
        &= 1- e^{-z}
    \end{align*}
    Therefore, the pdf is given by
    
    \begin{align*}
        \fun{P}{\conditbar{Z = z}{Y \geq 5}} &= \Dif{}{z} 1 - e^{-z} \\
        &= e^{-z}
    \end{align*}
    on $\mycal{Z} = [0, \infty)$.
\end{enumerate}

\newpage
\section{Exercise 4.18}
Polar coordinates.

\newpage
\section{Exercise 4.19}

\begin{enumerate}[label=(\alph*),leftmargin=*]
    \item By \cite[Theorem 4.2.14 on page 156]{Berger-Casella}, if $X \sim \normaldist{\mu_X, \sigma_X^2}$ and $Y \sim \normaldist{\mu_Y, \sigma_Y^2}$, then $X-Y \sim \normaldist{\mu_X - \mu_Y, \sigma_X^2 + \sigma_Y^2}$. In particular, when $X$ and $Y$ are both standard normal, the difference

\[ \frac{X-Y}{\sqrt{2}} \sim \normaldist{0, 1} \]
is standard normal as well. It follows from Exercise 4.14 that

\begin{align*}
    \frac{(X-Y)^2}{2} &= \left( \frac{X-Y}{\sqrt{2}} \right)^2 \\
    &\sim \left( \normaldist{0, 1} \right)^2 \\
    &\sim \chi_1^2
\end{align*}

    \item Refer to \cite[page 158]{Berger-Casella}.
    
    Define
    
    \[ \left\{ \begin{array}{cl}
         y_1 &= \frac{x_1}{x_1 + x_2}, \\
        y_2 &= x_1 + x_2,
    \end{array} \right. \]
    so that
    
    \[ \left\{ \begin{array}{cl}
        x_1 &= y_1 y_2, \\
        x_2 &= y_2 (1-y_1).
    \end{array} \right. \]
    Next, the Jacobi determinant is given by
    
    \begin{align*}
        \Abs{J} &= \left| \begin{array}{cc}
             \dif{x_1}{y_1} & \dif{x_1}{y_2} \\
             \dif{x_2}{y_1} & \dif{x_2}{y_2}
        \end{array} \right| \\
        &= \left\vert \begin{array}{cc}
             y_2 & y_1 \\
             -y_2 & 1-y_1
        \end{array} \right\vert \\
        &= \Abs{y_2}.
    \end{align*}
    
    Therefore, the joint distribution for $Y_1 = \frac{X_1}{X_1 + X_2}$ and $Y_2 = X_1 + X_2$ is given by
    
    \begin{align*}
        \fun{f_{Y_1, Y_2}}{y_1, y_2} &= \fun{f_{X_1, X_2}}{y_1 y_2,y_2 (1-y_1)} \cdot \Abs{y_2} \\
        &= \fun{f_{X_1}}{y_1y_2} \cdot \fun{f_{X_2}}{y_2(1-y_1)} \cdot \Abs{y_2} & \left( \mbox{since $X_1$ and $X_2$ are independent.} \right) \\
            &= \frac{(y_1y_2)^{\alpha_1 - 1} e^{-y_1 y_2}}{\fun{\Gamma}{\alpha_1}} \cdot \frac{(y_2(1-y_1))^{\alpha_2-1} e^{-y_2(1-y_1)}}{\fun{\Gamma}{\alpha_2}} \cdot \Abs{y_2} \\
            &= \left[ \frac{y_1^{\alpha_1 - 1} (1-y_1)^{\alpha_2 - 1}}{\fun{\Gamma}{\alpha_1} \fun{\Gamma}{\alpha_2}} \right] \cdot \left[ y_2^{\alpha_1 + \alpha_2 - 1} e^{-y_2} \right] \\
            &= \underbrace{\left[ \frac{\fun{\Gamma}{\alpha_1 + \alpha_2}}{\fun{\Gamma}{\alpha_1} \fun{\Gamma}{\alpha_2}} y_1^{\alpha_1 - 1} (1-y_1)^{\alpha_2 - 1} \right]}_{\fun{f_{Y_1}}{y_1}} \cdot \underbrace{\left[ \frac{y_2^{\alpha_1 + \alpha_2 - 1} e^{-y_2}}{\fun{\Gamma}{\alpha_1 + \alpha_2}} \right]}_{{\fun{f_{Y_2}}{y_2}}}.
    \end{align*}
    In particular, this shows $Y_1 \sim \betadist{\alpha_1, \alpha_2}$. Finding the pdf of $\frac{X_2}{X_1 + X_2} = 1 - Y_1$ is similar.
\end{enumerate}

\newpage
\section{Exercise 4.20}

We can think of the variables as Cartesian coordinates versus polar coordinates on $\D{R}^2$. The variables are related as:

\begin{align*}
    x_1 &= \sqrt{y_1} y_2, \\
    x_2 &= \pm \sqrt{y_1 - y_1 y_2^2},
\end{align*}
and we have two Jacobi matrices:

\begin{align*}
      J_{\pm} &= \left[ \begin{array}{cc}
           \frac{y_2}{2\sqrt{y_1}} & \sqrt{y_1} \\
           \frac{\pm \sqrt{y_1 - y_1 y_2^2}}{2y_1} & \mp \frac{y_1 y_2}{\sqrt{y_1 - y_1 y_2^2}}
      \end{array} \right], & 
\end{align*}
with $\Abs{J_{\pm}} = \frac{1}{2\sqrt{1 - y_2^2}}$. As a result, the joint distribution is given by

\begin{align*}
    \fun{f_{Y_1, Y_2}}{y_1, y_2} &= \left[ \fun{f_{X_1, X_2}}{\sqrt{y_1}y_2, \sqrt{y_1 - y_1 y_2^2}} + \fun{f_{X_1, X_2}}{\sqrt{y_1}y_2, -\sqrt{y_1 - y_1 y_2^2}} \right] \cdot \frac{1}{2\sqrt{1 - y_2^2}} \\
    &= \left[ \frac{1}{2 \pi \sigma^2} e^{-\frac{y_1}{2\sigma^2}} \right] \cdot \left[ \frac{1}{\sqrt{1 - y_2^2}} \right],
\end{align*}
proving $Y_1$, $Y_2$ are independent as well.

\newpage
\section{Exercise 4.21}

Write $\mycal{R} = R^2$. Then

\begin{align*}
    \fun{f_{X, Y}}{x,y} &= \fun{f_{\mycal{R}, \theta}}{\mycal{R} = x^2 + y^2, \theta = \fun{\arctan}{\frac{y}{x}}} \cdot \left\vert \begin{array}{cc}
         \dif{\mycal{R}}{x} & \dif{\mycal{R}}{dy} \\
         \dif{\theta}{x} & \dif{\theta}{dy}
    \end{array} \right\vert \\
    &= \left[ \frac{1}{2}e^{-\frac{x^2 + y^2}{2}} \right] \cdot \frac{1}{2\pi} \cdot \left\vert \begin{array}{cc}
         2x & 2y \\
         -\frac{y}{x^2 + y^2} & -\frac{x}{x^2 + y^2} \\
    \end{array} \right\vert \\
    &= \left[ \frac{1}{\sqrt{2\pi}} e^{-\frac{x^2}{2}} \right] \cdot \left[ \frac{1}{\sqrt{2\pi}} e^{-\frac{y^2}{2}} \right]
\end{align*}

\newpage
\section{Exercise 4.22}

We have

\[ \left\{ \begin{array}{cc}
     x &= \frac{u-b}{a}, \\
     y &= \frac{v-d}{c}.
\end{array} \right. \]
The Jacobi determinant is then given by

\begin{align*}
    \Abs{J} &= \Abs{\begin{array}{cc}
         \frac{1}{a} & 0  \\
         0 & \frac{1}{c}
    \end{array}} \\
    &= \frac{1}{ac}.
\end{align*}
Therefore, the result follows immediately from \cite[page 158]{Berger-Casella}.

\newpage
\section{Exercise 4.27}

Let

\[ \left\{ \begin{array}{cc}
     u &= x+y, \\
     v &= x-y
\end{array} \right. \]
Then

\begin{align*}
    \fun{f_{U,V}}{u,v} &= \fun{f_{X,Y}}{\fun{x}{u,v}, \fun{y}{u,v}} \cdot \Abs{J} \\
    &= \fun{f_X}{x(u,v)} \cdot \fun{f_Y}{y(u,v)} \cdot \Abs{J} \\
    & \ \ \ \ \left(\mbox{since $X$ and $Y$ are independent} \right) \\
    &= \fun{f_X}{x(u,v)} \cdot \fun{f_Y}{y(u,v)} \cdot \Abs{\begin{array}{cc}
         \frac{1}{2}& \frac{1}{2}  \\
         \frac{1}{2}& -\frac{1}{2}
    \end{array}} \\
    &= \frac{1}{2} \cdot \fun{f_X}{\frac{u+v}{2}} \cdot \fun{f_Y}{\frac{u-v}{2}} \\
    &= \frac{1}{4\pi \sigma^2} \fun{\exp}{-\frac{1}{2\sigma^2}\left[ \left( \frac{u+v}{2} - \mu \right)^2 + \left( \frac{u-v}{2} - \gamma \right)^2 \right]} \\
    &= \frac{1}{4\pi \sigma^2} \fun{\exp}{-\frac{1}{8\sigma^2} \left[\left[ (u+v) - 2\mu \right]^2 + \left[ (u-v) - 2\gamma \right]^2 \right]} \\
    &= \frac{1}{4\pi \sigma^2} \fun{\exp}{-\frac{1}{8\sigma^2} \left[ 2 \left[ u - (\gamma + \mu) \right]^2 - 2(\gamma + \mu)^2 + 2v^2 + 4(\gamma - \mu)v + 4\mu^2 + 4 \gamma^2 \right]} \\
    &= \frac{1}{4\pi \sigma^2} \fun{\exp}{-\frac{1}{8\sigma^2} \left[ 2 \left[ u - (\gamma + \mu) \right]^2 + 2 \left[ v - (\mu - \gamma) \right]^2 \right]} \\
    &= \underbrace{\frac{1}{ \sqrt{2\pi} \cdot \sqrt{2}\sigma} \fun{\exp}{- \frac{1}{2} \cdot \frac{\left[ u - ( \gamma + \mu) \right]^2}{2 \sigma^2}}}_{\fun{f_U}{u}} \cdot \underbrace{\frac{1}{ \sqrt{2\pi} \cdot \sqrt{2}\sigma} \fun{\exp}{- \frac{1}{2} \cdot \frac{\left[ v - ( \gamma - \mu) \right]^2}{2 \sigma^2}}}_{\fun{f_V}{v}} \\
    &\sim \normaldist{\gamma + \mu, 2\sigma^2} \cdot \normaldist{\gamma - \mu, 2\sigma^2}
\end{align*}

\newpage
\section{Exercise 4.30}

\begin{enumerate}[label=(\alph*),leftmargin=*]
    \item Firstly,
    
    \begin{align*}
        EY &= \fun{E}{\fun{E}{\conditbar{Y}{X}}} \\
        & \ \ \ \ \left( \mbox{\cite[Theorem 4.4.3 on page 164]{Berger-Casella}} \right) \\
        &= \fun{E}{\fun{E}{\normaldist{x,x^2}}} \\
        &= \fun{E}{X} \\
        &= \frac{1}{2}.
    \end{align*}
Secondly,

    \begin{align*}
        \Var{Y} &= \fun{E}{\Var{\conditbar{Y}{X}}} + \Var{\fun{E}{\conditbar{Y}{X}}} \\
        &= \fun{E}{\Var{\normaldist{x,x^2}}} + \Var{\fun{E}{\normaldist{x, x^2}}} \\
        &= \fun{E}{X^2} + \Var{X} \\
        &= 2 \Var{X} + \fun{E}{X}^2 \\
        &= \frac{5}{12}
    \end{align*}
Finally, to compute $\Cov{X,Y}$, we notice we have to deal with the random variable $XY$. Let $U = XY$, $V = X$. \cite[page 158]{Berger-Casella} gives the joint distribution

\begin{equation}
    \fun{f_{U,V}}{u,v} = \frac{1}{v} \fun{f_{X,Y}}{v, \frac{u}{v}},
\end{equation}
and (hence) the conditional distribution is given by

\begin{align}
    \fun{f_{\conditbar{U}{V}}}{\conditbar{u}{v}} &= \frac{\fun{f_{U,V}}{u,v}}{\fun{f_V}{v}} \nonumber \\
                              &= \frac{\frac{1}{v} \fun{f_{X,Y}}{v, \frac{u}{v}}}{\fun{f_X}{v}} \nonumber \\
                              &= \frac{1}{v} \fun{f_{\conditbar{Y}{X}}}{\conditbar{\frac{u}{v}}{v}}.
\end{align}
This allows us to prove the following formula for expectation:

\begin{align}
    \fun{E}{\fun{E}{\conditbar{XY}{X}}} &= \fun{E}{\fun{E}{\conditbar{U}{V}}} \nonumber \\
    & \ \ \ \ \left( \mbox{$U = XY$, $V = X$} \right) \nonumber \\
    &= \fun{E}{\myint{u \fun{f_{\conditbar{U}{V}}}{\conditbar{u}{v}}}{u}{}{}} \nonumber \\
    &= \fun{E}{\myint{\frac{u}{v} \fun{f_{\conditbar{Y}{X}}}{\conditbar{\frac{u}{v}}{v}}}{u}{}{}} \nonumber \\
    &= \fun{E}{\myint{xy \fun{f_{\conditbar{Y}{X}}}{\conditbar{y}{x}}}{y}{}{}} \nonumber \\
    & \ \ \ \ \left( \mbox{$y = \frac{u}{v}$, $x = v$, $dy = \frac{1}{v} du$} \right) \nonumber \\
    &= \fun{E}{X\fun{E}{\conditbar{Y}{X}}}.
\end{align}
As a result,

\begin{align*}
    \Cov{X,Y} &= \fun{E}{XY} - \left( EX \right) \left( EY \right) \\
    &= \fun{E}{XY} - \frac{1}{4} \\
    &= \fun{E}{\conditbar{XY}{X}} - \frac{1}{4} \\
    &= \fun{E}{X\fun{E}{\conditbar{Y}{X}}} - \frac{1}{4} \\
    &= \fun{E}{X \cdot \fun{E}{\normaldist{x, x^2}}} - \frac{1}{4} \\
    &= \fun{E}{X^2} - \frac{1}{4} \\
    &= \Var{X} + \left( EX \right)^2 - \frac{1}{4} \\
    &= \frac{1}{12}.
\end{align*}

\item Let $U = \frac{Y}{X}$, $V = X$. The Jacobi matrix is given by

\[ \left[ \begin{array}{cc}
     0 & 1 \\
     v & u \\
\end{array} \right], \]
and the joint distribution is given by

\begin{align*}
    \fun{f_{U,V}}{u,v} &= \fun{f_{X,Y}}{v, uv} \cdot \Abs{J} \\
    & \ \ \ \ \left( \mbox{\cite[page 158]{Berger-Casella}} \right) \\
    &= v \cdot \fun{f_X}{v} \cdot \fun{f_{\conditbar{Y}{X}}}{\conditbar{uv}{v}} \\
    &= v \cdot 1 \cdot \Gaussianpdf{\frac{uv-v}{v}}{v} \\
    &= \underbrace{\Gaussianpdf{u-1}{}}_{\fun{f_U}{u}} \cdot \underbrace{1}_{\fun{f_V}{v}} \\
    &\sim \normaldist{1,1} \cdot \uniformdist{0,1}.
\end{align*}
\end{enumerate}

\newpage
\nocite{*}
\printbibliography

\end{document}